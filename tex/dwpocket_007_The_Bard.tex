\chapter{The Bard}
 \index{The Bard} \index{Bard}
            

         

Sure, an adventurer’s life is all open roads and the glory of coin and combat. Those tales that are told in every farmhand-filled inn have to have some ring of truth to them, don’t they? The songs to inspire peasantry and royals alike—to sooth the savage beast or drive men to a frenzy have to come from somewhere.

         

Enter the Bard. You, with your smooth tongue and quick wit. You teller-of-tales and singer-of-songs. It takes a mere minstrel to retell a thing but a true Bard to live it. Strap on your boots, noble orator. Sharpen that hidden dagger and take up the call. Someone’s got to be there, fighting shoulder-to-shoulder with the goons and the thugs and the soon-to-be-heroes. Who better than you to write the tale of your own heroism?

         

Nobody. Get going.

       

       
\section{Names}  \index{Names} \index{Names}
       

         

           {\em Elf} : Astrafel, Daelwyn, Feliana, Damarra, Sistranalle, Pendrell, Melliandre, Dagoliir

         

           {\em Human} : Baldric, Leena, Dunwick, Willem, Edwyn, Florian, Seraphine, Quorra, Charlotte, Lily, Ramonde, Cassandra

       

       
\section{Look}  \index{Look}
       

         

Choose one for each:

         

Knowing Eyes, Fiery Eyes, Joyous Eyes

         

Fancy Hair, Wild Hair, Stylish Cap

         

Finery, Traveling Clothes, Poor Clothes

         

Fit Body, Well-fed Body, Thin Body

       

       
\section{Stats}  \index{Stats} \index{Stats}
       

         

Assign these scores to your stats:

         

17 (+2), 15 (+1), 13 (+1), 11 (+0), 9 (+0), 8 (-1)

         

You start with 6+Constitution HP.

       

       

Your base damage is d6.

       
\section{Starting Moves}  \index{Starting Moves} \index{Moves}
       

         
\startInstructionsAfterHeader
Choose a racial move:
\stopInstructionsAfterHeader
         

           
\subsection{Elf}  \index{Elf} \index{Elf}
           

When you enter an important location (your call) you can ask the GM for one fact from the history of that location.

           
\subsection{Human}  \index{Human} \index{Human}
           

When you first enter a civilized settlement someone who respects the custom of hospitality to minstrels will take you in as their guest.

         

         

           
\startInstructions
You start with these moves:
\stopInstructions
           
\subsection{Arcane Art (Cha)}  \index{Arcane Art (Cha)} \index{Arcane} \index{Art} \index{(cha)}
           

When you weave a performance into a basic spell, choose an ally and an effect:

           
\startitemize[1,packed]
             
\item Heal 1d8 damage

             
\item +1d4 forward to damage

             
\item Their mind is shaken clear of one enchantment

             
\item The next time the target is Aided, on a hit they get +2 instead of +1

           
\stopitemize
           

Then roll+Cha. On a hit, the ally gets the selected effect. On a 7-9, you also draw unwanted attention or your magic reverberates to other targets affecting them as well, GM's choice.

           
\subsection{Bardic Lore}  \index{Bardic Lore} \index{Bardic} \index{Lore}
           

Choose an area of expertise:

           
\startitemize[1,packed]
             
\item On Spells and Magicks

             
\item The Dead and Undead

             
\item Grand Histories of the Known World

             
\item A Bestiary of Creatures Unusual

             
\item The Planar Spheres

             
\item Legends of Heroes Past

             
\item Gods and Their Servants

           
\stopitemize
           

When you first encounter an important creature, location, or item (your call) covered by your Bardic Lore you can ask the GM any one question about it, the GM will answer truthfully. The GM may then ask you what tale, song, or legend you heard that information in.

           
\subsection{Charming and Open}  \index{Charming and Open} \index{Open}
           

When you speak frankly with someone, you can ask their player a question from the list below. If they answer it truthfully they then get to ask you a question from the list below.

           
\startitemize[1,packed]
             
\item Whom do you serve?

             
\item What do you wish I would do?

             
\item How can I get you to \_\_\_\_\_\_?

             
\item What are you really feeling right now?

             
\item What do you most desire?

           
\stopitemize
           
\subsection{A Port in the Storm}  \index{A Port in the Storm} \index{Port} \index{Storm}
           

When you return to a civilized settlement you've visited before, tell the GM when you were last here. They'll tell you how its changed since then.

         

       

       
\section{Alignment}  \index{Alignment} \index{Alignment}
       
\startInstructionsAfterHeader
Choose an alignment:
\stopInstructionsAfterHeader
       

         
\subsection{Good}  \index{Good} \index{Good}
         

Perform your art to aid someone else

         
\subsection{Neutral}  \index{Neutral} \index{Neutral}
         

Avoid a conflict or diffuse a tense situation

         
\subsection{Chaotic}  \index{Chaotic} \index{Chaotic}
         

Spur others to significant and unplanned decisive action

       

       
\section{Gear}  \index{Gear} \index{Gear}
       

         

Your Load is 5+Str. You have dungeon rations (5 uses, 1 weight). Choose one instrument:

         
\startitemize[1,packed]
           
\item Your father's mandolin, repaired

           
\item A fine lute, a gift from a noble

           
\item The pipes with which you courted your first love

           
\item A stolen horn

           
\item A fiddle, never before played

           
\item A songbook in a forgotten tongue

         
\stopitemize
         

Choose your clothing:

         
\startitemize[1,packed]
           
\item Leather armor  (1 armor, 1 weight)

           
\item Ostentatious clothes

         
\stopitemize
         

Choose your armament:

         
\startitemize[1,packed]
           
\item Dueling rapier (Close, Precise, 2 weight)

           
\item Worn bow (Near, 2 weight), bundle of arrows (3 ammo, 1 weight), and short sword (Close, 1 weight)

         
\stopitemize
         

Choose one:

         
\startitemize[1,packed]
           
\item Adventuring Gear (1 weight)

           
\item Bandages (0 weight)

           
\item Halfling pipeleaf (1 weight)

           
\item 3 coin

         
\stopitemize
       

       
\section{Bonds}  \index{Bonds} \index{Bonds}
       

         

Fill in the name of one of your companions in at least one:

         

This is not my first adventure with \thinrules[n=2].

         

I sang stories of \thinrules[n=2] long before I ever met them in person.

         

\thinrules[n=2] is often the butt of my jokes.

         

I am writing a ballad about the adventures of \thinrules[n=2].

         

\thinrules[n=2] trusted me with a secret.

         

\thinrules[n=2] does not trust me, and for good reason.

       

       
\section{Advanced Moves}  \index{Advanced Moves} \index{Advanced} \index{Moves}
       

         
\startInstructionsAfterHeader
When you gain a level from 2-5, choose from these moves.
\stopInstructionsAfterHeader
         
\subsection{Healing Song}  \index{Healing Song} \index{Song}
         

When you heal with Arcane Art, you heal +1d8 damage.

         
\subsection{Vicious Cacophony}  \index{Vicious Cacophony} \index{Vicious} \index{Cacophony}
         

When you grant bonus damage with Arcane Art, you grant an extra +1d4 damage.

         
\subsection{It Goes To Eleven}  \index{It Goes To Eleven} \index{Eleven}
         

When you unleash a crazed performance (a righteous lute solo, might brass blast, confusing interpretive dance) choose a target who can hear you and roll+Cha. On a 10+ the target flails in confusion dealing its damage to a creature of your choosing. On a 7–9 it deals its damage, but then takes +1d4 damage ongoing as the music invigorates it.

         
\subsection{Metal Hurlant}  \index{Metal Hurlant} \index{Metal} \index{Hurlant}
         

When you shout with great force or play a shattering note choose a target and roll+Con. On a hit the target takes 2d6 damage and is deafened for a few minutes. On a 7–9 it's out of control: the GM will choose an additional target nearby.

         
\subsection{A Little Help From My Friends}  \index{A Little Help From My Friends} \index{Friends}
         

When you successfully Aid someone you take +1 forward as well.

         
\subsection{Eldritch Tones}  \index{Eldritch Tones} \index{Eldritch} \index{Tones}
         

When you use Arcane Art, you choose two effects instead of one.

         
\subsection{Duelist's Parry}  \index{Duelist's Parry} \index{Duelist's} \index{Parry}
         

When you Hack and Slash, you take +1 armor forward.

         
\subsection{Bamboozle}  \index{Bamboozle} \index{Bamboozle}
         

When you Parley with someone, on a hit you also take +1 forward with them.

         
\subsection{Multiclass Dabbler}  \index{Multiclass Dabbler} \index{Multiclass} \index{Dabbler}
         

Get one move from another class. Treat your level as one lower for choosing the move.

         
\subsection{Multiclass Initiate}  \index{Multiclass Initiate} \index{Multiclass} \index{Initiate}
         

Get one move from another class. Treat your level as one lower for choosing the move.

         
\startInstructions
When you gain a level from 6-10, choose from these moves or the level 2-5 moves.
\stopInstructions
         
\subsection{Healing Chorus}  \index{Healing Chorus} \index{Chorus}
         

Replaces: Healing Song

         

When you heal with Arcane Art, you heal +2d8 damage.

         
\subsection{Vicious Blast}  \index{Vicious Blast} \index{Vicious} \index{Blast}
         

Replaces: Vicious Cacophony

         

When you grant bonus damage with Arcane Art, you grant an extra +1d4 damage.

         
\subsection{Unforgettable Face}  \index{Unforgettable Face} \index{Unforgettable} \index{Face}
         

When you meet someone you've met before (your call) after some time apart you take +1 forward against them.

         
\subsection{Reputation (Cha)}  \index{Reputation (Cha)} \index{Reputation} \index{(cha)}
         

When you first meet someone who's heard songs about you, roll+Cha. On a 10+, tell the GM two things they've heard about you. On a 7-9, tell the GM one thing they've heard, and the GM tells you one thing.

         
\subsection{Eldritch Chord}  \index{Eldritch Chord} \index{Eldritch} \index{Chord}
         

Replaces: Eldritch Tones

         

When you use Arcane Art, you choose two effects. You also get to choose one of those effects to double.

         
\subsection{An Ear For Magic}  \index{An Ear For Magic} \index{Ear} \index{Magic}
         

When you hear an enemy cast a spell the GM will tell you the name of the spell and its effects. Take +1 forward when acting on the answers.

         
\subsection{Devious}  \index{Devious} \index{Devious}
         

When you use Charming and Open you may also ask "How are you vulnerable to me?" Your subject may not ask this question of you.

         
\subsection{Duelist's Block}  \index{Duelist's Block} \index{Duelist's} \index{Block}
         

Replaces: Duelist's Parry

         

When you Hack and Slash, you take +2 armor forward.

         
\subsection{Con}  \index{Con} \index{Con}
         

Replaces: Bamboozle

         

When you Parley with someone, on a hit you also take +1 forward with them and get to ask their player one question which they must answer truthfully.

         
\subsection{Multiclass Master}  \index{Multiclass Master} \index{Multiclass} \index{Master}
         

Get one move from another class. Treat your level as one lower for choosing the move.

       

                
