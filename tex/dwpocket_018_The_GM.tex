\chapter{The GM}
 \index{The GM} \index{Gm}
 



This section is about the art and rules of being the Game Master or GM. There are many styles of GMing epic fantasy games with things like dragons and dungeons and brave adventurers but Dungeon World is designed for one of those styles in particular. These rules will help you run a game in that style.

 

Just because the rules are mechanical doesn't mean they're removed from the fiction of what's actually happening in the game or that you're playing to win. You'll be refereeing, adjudicating, and narrating your part the game much like you would any other game. You'll just have a framework that helps you determine what to say, at what time.

 

The GM's rules are rules, just like the rules for moves and character creation and all the rest. Just like every other rule in the game, they are designed to help you play a game of exploration and epic fantasy. You will of course be making your own rules, in the form of custom moves, but the GM's rules are as important to playing Dungeon World as the rules for rolling dice. Play with the rules as written before making any changes, and think carefully about any changes you do make.

 
\subsection{The Basics}  \index{The Basics} \index{Basics}
 

Dungeon World is built on a framework: the GM's {\bf agenda} , {\bf principles} , and {\bf moves} . The GM's agenda is what they set out to do when they sit down at the table. The principles are the guides that keep the GM focused on their agenda. The GM's moves are the concrete, moment-to-moment things the GM does to drive the game forward. The GM's moves aren't like player moves, they aren't triggered by the fiction. Instead they are actions that drive the game onward.

 

The GM's agenda, principles, and moves are rules just like damage or stats or HP. You should take the same care in altering them or ignoring them that you would with any other rule. Changing a principle may have just as much of an effect on your game as changing the Fighter's damage dice or giving the Cleric access to Wizard spells.

 
\section{Always Say}  \index{Always Say}
 

When running Dungeon World as the GM you say these things:

 
\startitemize[1,packed]

\item What the rules demand

 
\item What the adventure demands

 
\item What honesty demands

 
\item What the principles demand


\stopitemize
 

The players have it easy—they just say what their characters say, think and do. You have it a bit harder. You have to say everything else. So what do you say? Say {\bf what the rules tell you to} . If a move has triggered, yours or the players', then say what the rules tell you to say. Embellish and expand but use the rules to give you a start. The rules will always give you material to work with.

 

Say {\bf what the adventure demands} . You'll know some things before you sit down at the table. You might know where the goblins are hiding or when the reinforcements are going to arrive. If the players haven't done anything to change those things, stick with them.

 

Always {\bf be honest} . If the rules tell you to give out information, like the Spout Lore and Discern Realities moves, do it. Don't lie or give half truths; be open and honest—generous, even. The player characters have risked something to get that information just by rolling so make it worth their while. If you don't know the answer make one up or turn the question back to the players. Once you tell the players it's set in stone, no going back on it.

 

This applies in general to the players' actions, too. If they have worked to achieve something, you should give it to them fully. You're not here to fight back against the players; you're not opposed to them at all. You are playing the game with them.

 

At all times, {\bf use your principles and agenda}  as a filter or inspiration. If something falls flat it's usually because you ignored one of your principles or acted on a different agenda. If you're unsure of what you're about to say just take a moment and look at your agenda and principles to make sure you're abiding by them.

 
\section{Agenda}  \index{Agenda} \index{Agenda}
 

The GM's agenda is what they sit down at the table to do:

 
\startitemize[1,packed]

\item Make the world fantastic

 
\item Fill the characters' lives with adventure

 
\item Play to find out what happens


\stopitemize
 

Everything you say, create, and do at the table and away from the table is to accomplish these three goals and no others. Things that aren't on this list aren't your goals. You're not trying to beat the players or test their ability to solve complex traps. You're not here to give the players a chance to explore your finely crafted setting. You're most certainly not here to tell everyone a planned story.

 

That one deserves repeating: {\bf you are not here to tell everyone a planned story} . Don't ever plan a storyline. You do not know what will happen to the players' characters any more than they do. Your job is to portray a fantastic world, not provide a canned plot.

 

To that end, Dungeon World adventures {\bf never}  presume player actions. A Dungeon World adventure describes a location in motion, someplace important with creatures pursuing their own goals. As the players come into conflict with that location, it will snowball into action. You'll honestly portray the repercussions of their actions.

 

When you play this way you get to share in the fun of finding out what happens to the characters and the world around them. You're not a frustrated novelist trying to organize your unruly characters. You're a participant in a great story that's unfolding. So really—don't plan the story. The rules of the game will fight you.

 

{\bf Fill the character's life with adventure}  means helping the players create a world that's exciting and full of epic foes to battle, strange places to explore, and glorious treasure to discover. Adventurers are always caught up in some plot or world-threatening danger or another—encourage and foster that kind of action in the game.

 

The players have an agenda too, but it's probably something they'll do by default: portray their characters.

 
\section{Principles}  \index{Principles} \index{Principles}
 
\startitemize[1,packed]

\item Draw maps, leave blanks

 
\item Address the characters, not the players

 
\item Embrace the fantastic

 
\item Make a move that follows

 
\item Never speak the name of your move

 
\item Give every monster life

 
\item Name every person

 
\item Ask questions and use the answers

 
\item Be a fan of the characters

 
\item Think Dangerous

 
\item Begin and end with the fiction

 
\item Think offscreen, too


\stopitemize
 

Your principles are your guides. Often, when it's time to make a move, you'll already have an idea. Quickly run it past your principles and make sure it fits, then go with it.

 
\subsubsection{Draw maps, leave blanks}  \index{Draw maps leave blanks} \index{Draw} \index{Maps} \index{Leave} \index{Blanks}
 

Dungeon World is mostly in our imaginations, but we can actually see it when we draw a map. So, make use of maps. You won't always be drawing them yourself, but any time there's a new location draw it on a map (or make a new map for it).

 

When you draw a map, it doesn't have to be complete. Leave blanks, places that are unknown to you. As you play you'll get more ideas or the players will give you inspiration to work with.

 
\subsubsection{Address the characters, not the players}  \index{Address the characters not the players} \index{Address} \index{Characters} \index{Players}
 

Addressing the characters, not the players, means that you don't say "Whit, is Dunwick doing something about that wight?" Instead you say "Dunwick, what are you doing about the wight?" Talking this way keeps the game rooted in the fiction and not at the table. It's important to the flow of the game, too. If you talk to the players you may leave out details that are important to what moves the characters make. Since moves are always based on the actions of the character you need to think about what's happening in terms of characters—not players.

 
\subsubsection{Embrace the fantastic}  \index{Embrace the fantastic} \index{Embrace} \index{Fantastic}
 

The fantastic is the core of fantasy: magic, strange vistas, gods, demons, and abominations. The player characters already have these kind of abilities, so you should reflect them in the world.

 
\subsubsection{Make a move that follows}  \index{Make a move that follows} \index{Move}
 

When you make a move what you're actually doing is taking some element of the fiction and bringing it to bear against the characters. Your move should always follow from the fiction, and you never speak its name. Instead describe the fictional actions that take place which follow from the situation established.

 
\subsubsection{Never speak the name of your move}  \index{Never speak the name of your move} \index{Speak} \index{Move}
 

There is no quicker way to ruin the consistency of Dungeon World than to tell the players what move you're making. Your moves are prompts to you, not things you say directly.

 

You never show the players that you're picking a move from a list. You know the reason the slavers dragged off Omar was because you made the "Put someone in a spot" move, but you show it to the players as a straightforward outcome of their actions.

 
\subsubsection{Give every monster life}  \index{Give every monster life} \index{Give} \index{Monster} \index{Life}
 

Monsters are nameless hordes of creatures that stand between the players and what they want. Give each monster details that bring it to life: smells, sights, sounds. Your monsters are arrows, fired en masse at the players. Give each enough detail to make it real, but don't cry when it gets slain by intrepid adventurers.

 
\subsubsection{Name every person}  \index{Name every person} \index{Person}
 

Every person gets a name. You'll have a name list to work from on your adventure sheet, so don't worry too much about it. Anyone that the players interact with has a name. They probably have a personality and some goals or opinions too, but you can figure that out as you go. Start with a name. The rest can flow from there.

 

How do you know if someone gets a name? If you start dealing with them as an individual (not just "a member of the Knob Street gang" or "a goblin ambusher") it's time for a name.

 
\subsubsection{Ask questions and use the answers}  \index{Ask questions and use the answers} \index{Questions} \index{Answers}
 

You don't have to know everything. If you don't know, or you don't have an idea, just ask the players and use what they say.

 

The easiest question to use is "What do you do?" Whenever you make a move, end with "What do you do?" You don't even have to ask the person you made the move against. Take that chance to shift the focus elsewhere: "Rath's spell is torn apart with a flick of the mage's wand. Finnegan, that spell was aiding you. What are you doing now that it's gone?"

 
\subsubsection{Be a fan of the characters}  \index{Be a fan of the characters} \index{Fan} \index{Characters}
 

Treat the players' characters like characters you watch on TV. You want to see how things turn out for them. You're not here to make them lose, or to make them win, and definitely not to guide them to your story. You're here to portray the interesting world around them and see how interacting with that world changes everything.

 
\subsubsection{Think Dangerous}  \index{Think Dangerous} \index{Dangerous}
 

Thinking Dangerous means that everything in the world is a target. You're thinking like an evil overlord: no single life is worth anything, there is nothing sacrosanct. Everything can be put in danger, everything can be destroyed. Nothing you create is ever protected. Whenever your eye falls on something you've created, think dangerous. Think how it can be put in danger, fall apart, crumble.

 
\subsubsection{Begin and end with the fiction}  \index{Begin and end with the fiction} \index{Fiction}
 

Everything you and the players do in Dungeon World comes from and leads to fictional events. When the players make a move, they always take a fictional action to trigger it, apply the rules, and get a fictional effect. When you make a move it always comes from the fiction.

 

You can apply this to everything you say. Start with the fiction ("The ogre's axe comes sailing down into your shoulder…"), apply the rules ("…you take 12 damage…"), go back to the fiction ("…as your collar bone cracks beneath your armor. What do you do?").

 
\subsubsection{Think offscreen too}  \index{Think offscreen too} \index{Offscreen}
 

Just because you're a fan of the characters doesn't mean everything happens right in front of them. Sometimes your best move is in the next room, or another part of the dungeon, or even back in town. Make your move elsewhere and show the effects later.

 
\section{Moves}  \index{Moves} \index{Moves}
 
\startitemize[1,packed]

\item Use a monster, danger, or location move

 
\item Reveal an unwelcome truth

 
\item Show signs of doom

 
\item Deal damage

 
\item Use up their resources

 
\item Turn their move back on them

 
\item Separate them

 
\item Give an opportunity that fits a class' abilities

 
\item Show a downside to their class, race, or equipment

 
\item Offer an opportunity, with or without cost

 
\item Put someone in a spot

 
\item Tell them the requirements or consequences and ask


\stopitemize
 

Whenever everyone looks to you to see what happens choose one of these. Each move is something that occurs in the fiction of the game—they aren't code words or special terms. "Use up their resources" literally means to expend the resources of the characters.

 

Of course you don't say that to the players. You never speak the name of your move (it's one of your principles). You make it a real thing that happens to them: "As you dodge the hulking ogre's club, you slip and land hard. Your sword goes sliding away into the darkness. You think you saw where it went but the ogre is lumbering your way. What do you do?"

 

No matter what move you make always follow up with "What do you do?" Your move is a way of fulfilling your agenda—part of which is to fill the character's lives with adventure. When a spell goes wild or the floor drops out from under them adventurers react.

 
\subsection{When to Make a Move}  \index{When to Make a Move} \index{Move}
 

You make a move when everyone looks to you to find out what happens. When it's your turn to say something in the conversation you make a move. In particular, you make a soft move: a move that sets up a future move.

 

Making a soft move just means that you put events in motion, then let the players react. If they don't do anything about it you follow through with the full consequences, making another (harder) move. Showing signs of doom is your most versatile soft move since the doom you portend is a move waiting to happen.

 

Of course your moves apply when the players undertake something that's not a player move. In that case the players will say something, like "I lay my case before the king, pleading for aid," and look to you to find out what happens. Since they haven't made a move (there's no leverage to make a Parley) you just respond with a soft move of your own as setup by the fiction.

 

You also make a move when the players give you a golden opportunity. A golden opportunity is any time they ignore a threat or when they fail a roll (6-).

 

When they give you a golden opportunity, you can make your move just as hard as you like. A hard move is one that is irrevocable and immediate. The players immediately feel the consequences of the move and have to deal with them. Dealing damage is a hard move, since the damage is immediately applied.

 

Soft moves are useful to setup future harder moves. When the doom you show signs of is an onslaught of goblin arrows, if the players don't so something to get out of the way, you can follow through with damage as a hard move. Ignoring the oncoming arrows is a golden opportunity.

 
\subsection{Choosing a Move}  \index{Choosing a Move} \index{Move}
 

To choose a move, start by looking at the obvious consequences of the action that triggered it. If you already have an idea, think on it for a second to make sure it fits your agenda and principles and then do it. {\bf Let your moves snowball} . Build on the success or failure of the characters moves and on your own previous moves.

 

You can choose to save up your moves instead. Use this option sparingly, only when you're sure the consequences of their action occurred off screen and that you'll be able to come up with those consequences later. The saved move should always be used in the same physical area, such as a dungeon complex or sprawling swamp. Your players will come to expect you to make hard moves on the tail of their failed rolls—this will throw them off. Their actions will instead come back to bite them later. Be careful with it.

 
\subsection{Making your Move}  \index{Making your Move} \index{Move}
 

When making a move, keep your principles in mind. In particular, "never speak the name of your move" and "address the characters, not the players." Your moves are not mechanical actions happening around the table. They are concrete events happening to the characters in the fictional world you are describing.

 

Note that "Deal damage" is a move, but other moves may include damage as well. When an ogre flings you against a wall you take damage as surely as if he had smashed you with his fists. If a monster deals damage incidentally as part of another move, like charging past Titanius slamming her to the ground, the damage dealt is equal to half the monster's normal damage.

 

If a move causes damage not related to a monster, like a collapsing tunnel or fall into a pit, use the damage rules on page X (Blood and Guts chapter).

 

After every move you make, always ask "What do you do?" The players' characters are the stars, remember.

 
\subsubsection{Use a monster, danger, or location move}  \index{Use a monster danger or location move} \index{Monster} \index{Danger} \index{Location} \index{Move}
 

Each monster in an adventure has moves associated with it, as do many locations. A monster or location move is just a description of what that location or monster does, maybe "hurl someone away" or "bridge the planes." If a move (like Hack and Slash) says that a monster gets to make an attack, make a move with that monster.

 

The overarching dangers of the adventure also have moves associated with them. Use these moves to bring that danger into play, which may mean more monsters.

 
\subsubsection{Reveal an unwelcome truth}  \index{Reveal an unwelcome truth} \index{Reveal} \index{Unwelcome} \index{Truth}
 

An unwelcome truth is a fact the players wish wasn't true: that the room's been trapped, maybe, or that the helpful goblin is actually a spy. You never make up an unwelcome truth when making this move—you just bring one to light. Reveal to the players just how much trouble they're in.

 
\subsubsection{Show signs of doom}  \index{Show signs of doom} \index{Show} \index{Signs} \index{Doom}
 

This is one of your most versatile moves. 'Doom' is anything bad that's coming. With this move, you just show them that something's going to happen unless they do something about it. Remember to ask "What do you do?"

 
\subsubsection{Deal damage}  \index{Deal damage} \index{Deal} \index{Damage}
 

When you deal damage you choose one source of damage that's fictionally threatening a character and apply it. In combat with a lizard man? It stabs you. Running from a collapsing tunnel? Some rocks catch your ankle.

 

The amount of damage is decided by the source. In some cases, this move might involve trading damage both ways, with the player character also dealing damage.

 

Most damage is based on a dice roll. When a player takes damage, tell them what to roll, you never need to touch the dice. If the player is too cowardly to find out their own fate, they can ask another player to roll for them.

 

Dealing damage is a hard move. Use it carefully.

 
\subsubsection{Use up their resources}  \index{Use up their resources} \index{Resources}
 

Surviving in a dungeon, or anywhere dangerous, often comes down to supplies. With this move, something happens to use up some resource: weapons, armor, healing, ongoing spells. You don't always have to use it up permanently. A sword might just be flung to the other side of the room, not shattered.

 
\subsubsection{Offer an opportunity, with or without cost}  \index{Offer an opportunity with or without cost} \index{Offer} \index{Opportunity} \index{Cost}
 

Show them something they want: riches, power, glory. If you want, you can associate some cost with it too, of course.

 

Remember to lead with the fiction. You don't say "This area isn't dangerous so you can make camp here, if you're willing to take the time." You make it a solid fictional thing and say "Helferth's blessings still hang around the shattered alter. It's clearly been untouched, the goblins don't come here. It's a nice safe spot, but the chanting from the ritual chamber is getting louder. What do you do?"

 
\subsubsection{Put someone in a spot}  \index{Put someone in a spot} \index{Put} \index{Spot}
 

A spot is someplace where they have to make tough, ugly choices. Put them in the path of destruction. Put someone or something they care about in a dangerous situation. Whatever you do, just make sure they're someplace where they have to take action and then ask "What do you do?"

 
\subsubsection{Tell them the requirements or consequences and ask}  \index{Tell them the requirements or consequences and ask} \index{Requirements} \index{Consequences}
 

This move is particularly good when they've done something that's not a move, or failed a move. They can do it, sure, but they'll have to meet the requirements. Or, they can do it, but there will be consequences. Maybe they can swing across the chasm, fully armored, and leap into battle, but the rope will be stressed beyond usefulness afterwards. Maybe they can swim through the crocodilian-infested moat before being devoured, but they'll need a distraction. Of course this is made clear to the characters, not just the players: the crocodilians are slavering hungry and starved, or the rope already has dangerous give.

 
\section{Dungeon Moves}  \index{Dungeon Moves} \index{Dungeon} \index{Moves}
 

Dungeon Moves are a special subset that are used to make or alter a dungeon on the fly. Use these if your players are exploring a hostile area that you don't already have planned all the way through.

 

Map out the area being explored as you make these moves. Most of them will require you to add a new room or element to your map.

 
\startitemize[1,packed]

\item Change the environment

 
\item Point to a looming threat

 
\item Introduce a new faction or type of creature

 
\item Use a threat from an existing faction or type of creature

 
\item Make them backtrack

 
\item Present riches at a price

 
\item Present a challenge to one of the characters


\stopitemize
 

You can make these moves whenever everyone looks to you to say something, when the players present you an opportunity, or when the players miss on a roll. They're particularly well-suited for when the players look at you to find out what a new room or area is like.

 
\subsubsection{Change the environment}  \index{Change the environment} \index{Change} \index{Environment}
 

The environment is the general feel of the area the players are in: carved tunnels, warped trees, safe trails, or whatever else. This is your opportunity to introduce them to a new environment: the tunnels gradually become naturally carved, the trees are dead, or the trails are lost and the wilderness takes over. Use this move to vary the types of areas and creatures the players will face.

  

It's an opportunity for you to interject with a change in scenery and play up the themes and dangers that are to come. Snowball this move with itself over time to gradually shift the dungeon to something new and exciting by using one or two elements at a time. First the scent of brimstone fills the tunnels, then hellish sigils mar the walls, then the moans of the damned and before the players know it, they're not in a cavern at all—they're in the pit of a demon lord!

 
\subsubsection{Point to a looming threat}  \index{Point to a looming threat} \index{Point} \index{Threat}
 

If you know that something lurking and waiting for the players to stumble upon it, this move shows them the signs and clues. This move is the dragon's footprints in the mud or the slimy trail of the gelatinous cube.

 

This move means that when the players finally come face-to-face with the threat, they'll have some ideas and fear about what awaits them. Use it to build tension or, in some cases, provide hints that prove to be a surprise. It's not a wicked red dragon like the players expected, it's a wounded silver drake who needs their help.

 
\subsubsection{Introduce a new faction or type of creature}  \index{Introduce a new faction or type of creature} \index{Introduce} \index{Faction} \index{Type} \index{Creature}
 

A type of creatures is a broad grouping: orcs, goblins, lizardmen, golems, etc.

 

A faction is a group of creatures united by a similar goal. Once you introduce them you can begin to make moves and cause trouble for the players with those creatures or NPCs.

 

Introducing means giving some clear sensory evidence or substantiated information. Don't be coy, the players should have some idea what you're showing the presence of. You can, however, be subtle in your approach. No need to have the cultist overlord waving a placard and screaming in the infernal tongue every single time.

 

You don't have to warn the players about this move. A hard application of this move will snowball directly into a combat scene or ambush.

 
\subsubsection{Use a threat from an existing faction or type of creature}  \index{Use a threat from an existing faction or type of creature} \index{Threat} \index{Faction} \index{Type} \index{Creature}
 

Once the characters have some been introduced to the presence a faction or type of creature you can use monsters of that type.

 

Use the factions and types broadly. Orcs are accompanied with their hunting worgs. A mad cult probably has some undead servants or maybe a few beasts summoned from the deepest pits. This is a move that, often, you'll be making subconsciously—it's just implementing the tools you've set out for yourself in a clear and effective manner.

 
\subsubsection{Make them backtrack}  \index{Make them backtrack} \index{Backtrack}
 

Look back at the spaces you've added to the map. Is there anything useful there as yet undiscovered? Can you add a new obstacle that can only be overcome by going back there? Is there a locked door here and now whose key lies in an earlier room?

 

When backtracking, take the opportunity to show the effect the players have had on the areas they've left behind. What new threats have sprung up in their wake? What didn't they take care of that's waiting for their return?

 

Use this move the make the dungeon a living, breathing place. There is no stasis in the wake of the characters' passing. Add reinforcements, cave in walls, cause chaos. Make the dungeon evolve in the wake of the characters' actions.

 
\subsubsection{Present riches at a price}  \index{Present riches at a price} \index{Present} \index{Riches} \index{Price}
 

What do the players want? What might they give something up for?

 

Put some desirable item just out of reach. Find something they're short on: time, HP, gear, whatever. Find a way to make what they want available if they give up what they have.

 

The simplest way to use this move is the promise of gold out of the way of the main objective. Will they stop to pry the ruby eyes from the idol when they know that the virgin sacrifice looms closer and closer? Use this move and you can find out.

 
\subsubsection{Present a challenge to one of the characters}  \index{Present a challenge to one of the characters} \index{Present} \index{Challenge} \index{Characters}
 

Challenge a character by looking at what they're good at. Give the Thief a lock to pick, show the Cleric evil gods to battle against. Give the Wizard magical mysteries to investigate. Show the Fighter some skulls he can crack. Give someone a chance to shine.

 

As an alternative, challenge a character by looking at what they're bad at or what they've left unresolved. If the Bard has a long con running what steps will he take to cover it up when someone figures him out? If the Wizard has been summoning demons then what happens when word gets out?

 

This move can give a character the spotlight—even if just for a moment. Try to give everyone a chance to be the focus of play using this move from session to session.

 





 
