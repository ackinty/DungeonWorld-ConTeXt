
\enableregime[utf]
\useregime[utf]

\loadmapfile[texnansi-open-dwfonts.map]


\starttypescript[serif][vollkorn-serif][texnansi]
	\definefontsynonym[VollkornSerifRegular][texnansi-Vollkorn-Regular][encoding=texnansi]
	\definefontsynonym[VollkornSerifBold][texnansi-Vollkorn-Bold][encoding=texnansi]
	\definefontsynonym[VollkornSerifBolditalic][texnansi-Vollkorn-BoldItalic][encoding=texnansi]
	\definefontsynonym[VollkornSerifItalic][texnansi-Vollkorn-Italic][encoding=texnansi]
	\definefontsynonym[VollkornSerifSlanted][texnansi-Vollkorn-Italic][encoding=texnansi]
	\definefontsynonym[VollkornSerifBoldslanted][texnansi-Vollkorn-BoldItalic][encoding=texnansi]
	\definefontsynonym[VollkornSerifCaps][texnansi-Vollkorn-Bold][encoding=texnansi]
	\stoptypescript


\starttypescript[serif][vollkorn-serif][name]
	\usetypescript[serif,fallback]
	\definefontsynonym[Serif][VollkornSerifRegular]
	\definefontsynonym[SerifBold][VollkornSerifBold]
	\definefontsynonym[SerifBoldItalic][VollkornSerifBolditalic]
	\definefontsynonym[SerifItalic][VollkornSerifItalic]
	\definefontsynonym[SerifSlanted][VollkornSerifSlanted]
	\definefontsynonym[SerifBoldSlanted][VollkornSerifBoldslanted]
	\definefontsynonym[SerifCaps][VollkornSerifCaps]
\stoptypescript

\starttypescript[sans][sigvar-sans][ec]
	\definefontsynonym[SigvarSansRegular][texnansi-OpenSansRegular][encoding=ec]
	\definefontsynonym[SigvarSansBold][texnansi-OpenSansBold][encoding=ec]
	\definefontsynonym[SigvarSansBolditalic][texnansi-OpenSansBoldItalic][encoding=ec]
	\definefontsynonym[SigvarSansItalic][texnansi-OpenSanItalic][encoding=ec]
	\definefontsynonym[SigvarSansSlanted][texnansi-OpenSanItalic][encoding=ec]
	\definefontsynonym[SigvarSansBoldslanted][texnansi-OpenSansBoldItalic][encoding=ec]
	\definefontsynonym[SigvarSansCaps][texnansi-OpenSansBold][encoding=ec]
	\stoptypescript


\starttypescript[sans][sigvar-sans][name]
	\usetypescript[sans,fallback]
	\definefontsynonym[Sans][SigvarSansRegular]
	\definefontsynonym[SansBold][SigvarSansBold]
	\definefontsynonym[SansBoldItalic][SigvarSansBolditalic]
	\definefontsynonym[SansItalic][SigvarSansItalic]
	\definefontsynonym[SansSlanted][SigvarSansSlanted]
	\definefontsynonym[SansBoldSlanted][SigvarSansBoldslanted]
	\definefontsynonym[SansCaps][SigvarSansCaps]
\stoptypescript

\starttypescript[mono][display-mono][texnansi]
	\definefontsynonym[DisplayMonoRegular][texnansi-BOYCOTT][encoding=texnansi]
	\definefontsynonym[DisplayMonoBold][texnansi-DIRTYEGO][encoding=texnansi]
	\definefontsynonym[DisplayMonoBolditalic][texnansi-goodluck][encoding=texnansi]
	\definefontsynonym[DisplayMonoItalic][texnansi-PROPAGAN][encoding=texnansi]
	\definefontsynonym[DisplayMonoSlanted][texnansi-Deutsch][encoding=texnansi]
	\definefontsynonym[DisplayMonoBoldslanted][texnansi-Deutsch][encoding=texnansi]
	\definefontsynonym[DisplayMonoCaps][texnansi-Deutsch][encoding=texnansi]
	\stoptypescript


\starttypescript[mono][display-mono][name]
	\usetypescript[mono,fallback]
	\definefontsynonym[Mono][DisplayMonoRegular]
	\definefontsynonym[MonoBold][DisplayMonoBold]
	\definefontsynonym[MonoBoldItalic][DisplayMonoBolditalic]
	\definefontsynonym[MonoItalic][DisplayMonoItalic]
	\definefontsynonym[MonoSlanted][DisplayMonoSlanted]
	\definefontsynonym[MonoBoldSlanted][DisplayMonoBoldslanted]
	\definefontsynonym[MonoCaps][DisplayMonoCaps]
\stoptypescript



\starttypescript[fantasy]
	\definetypeface[fantasy][tt][mono][display-mono][default][encoding=texnansi]
	\definetypeface[fantasy][ss][sans][sigvar-sans][default][encoding=ec]
	\definetypeface[fantasy][rm][serif][vollkorn-serif][default][encoding=texnansi]
\stoptypescript



\usetypescript[fantasy]
\setupbodyfont[fantasy,rm,10pt]


% uncomment these line to activate microtypography

 \setupfontsynonym[Sans][handling=highquality]
 \setupfontsynonym[SansBold][handling=highquality]
 \setupfontsynonym[SansItalic][handling=highquality]
 \setupfontsynonym[SansBoldItalic][handling=highquality]
% \setupfontsynonym[SansSlanted][handling=highquality]
% \setupfontsynonym[SansBoldSlanted][handling=highquality]
% \setupfontsynonym[SansSmallCaps][handling=highquality]
 \setupfontsynonym[Serif][handling=highquality]
 \setupfontsynonym[SerifBold][handling=highquality]
 \setupfontsynonym[SerifItalic][handling=highquality]
 \setupfontsynonym[SerifBoldItalic][handling=highquality]
 \setupfontsynonym[SerifSlanted][handling=highquality]
 \setupfontsynonym[SerifBoldSlanted][handling=highquality]
 \setupfontsynonym[SerifSmallCaps][handling=highquality]

% \setupfonthandling[hz][min=0, max=90, step=5]
% \setupalign[hanging,hz,spacing]

\setupinterlinespace
[height=.72,
depth=.28,
top=1.0,
bottom=0.4,
line=2.7ex]


% definizione di una serie di sinonimi per cambiare font in maniera indiretta. Ad esempio:
% {\eulbigtitle Ciccio} -> mostrer� Ciccio reso in Immortal, corpo 18
% ----------------------------------------------------------------------------------------
\definefont[hi][MonoBold sa 3]
\definefont[hii][MonoBold sa 2]
\definefont[hiii][MonoBold sa 1.7]
\definefont[hiv][MonoBold sa 1.4]


\definefont[legacy][SansBold]
\definefont[legacyb][SansBold]
\definefont[eul][SansBold]
\definefont[eub][SansBold]
\definefont[eulb][SansBold at 12pt]
\definefont[eulgiant][SerifBold at 36pt]
\definefont[eulbig][SansBold at 12pt]
\definefont[eulbignorm][SansBold at 10pt]
\definefont[eulbigit][SansBold]
\definefont[eultitle][SansBold at 14pt]
\definefont[eulbiggie][SansBold at 72pt]
\definefont[eulbiggiex][SansBold at 24pt]
\definefont[booksubtitlefont][SansBold at 24pt]
\definefont[booktitlefont][SansBold at 48pt]


%-------- definisco dei colori
\setupcolors[state=start]
\definecolor[grigino][r=.9,g=.9,b=.9]
\definecolor[grigino2][r=1,g=.5,b=.1]
\definecolor[rossoscuro][r=0.66,g=.1,b=.1]

\def\reallynobreak{\unskip\penalty 100000}

%----------- macro per definire il titolo di un "libro" -- rulebook e sourcebook
\def\BookTitle#1
	{
		\page[odd] % titolo su una nuova pagina DISPARI
		\setupbackgrounds[rightpage][background=color,backgroundcolor=grigino]
		\setupbackgrounds[leftpage][background=color,backgroundcolor=grigino]
		\vskip 6cm
		\hairline
		
		\startalignment[flushright]
		\setupbodyfont[itc-gentium, rm, 24pt]
		{\booksubtitlefont The Shadow Of Yesterday}
		
		\setupbodyfont[itc-gentium, rm, 48pt]		
		{\booktitlefont #1}
		
		\setupbodyfont[itc-gentium, rm, 11pt]
		\hairline
		\vfill
		\stopalignment	
}

%-- una serie di macro che riutilizzer� quando far� l'indice, che mappano il sottotitolo
% direttamente con una entry dell'indice

\def\mySubSubject#1%
	{\subsubject{#1} \index{#1}}
	
\def\mySubSubSubject#1%
	{\subsubsubject{#1} \index{#1}}
	
\setupitemize[headstyle=\bf,before=\blank]

\setuphead[title] % equivalente a H1 -- Senza numero
[
	align={center,nothyphenated},
	style={\eulgiant},
	after={\blank[5*line]}
]

\definecolor[BigColorOne][r=.9,g=.7,b=.7]
\definecolor[BigColorTwo][r=.66,g=.0,b=.0]

\definefont[HugeFont][MonoCaps at 96pt]
\definefont[BigFontOne][MonoItalic sa 6] %RegularSlanted sa 3(ori)%Sans sa 3:don't work (pb accents)
\definefont[BigFontTwo][MonoCaps sa 3.5]   %%RegularSlanted
      
\def\MyChapterText#1{\rlap{\BigFontOne\BigColorOne#1}#1}
\def\HeadTitleMax#1#2%
{\rlap{\BigFontOne\BigColorOne #1}#2}

\def\DWTitle#1%
{\framed[align=center,backgroundcolor=white,color=black,background=color,width=broad,frame=no]{\BigFontTwo #1}}

\def\DWNum#1%
{\framed[align=center,width=broad,frame=no,]{\HugeFont{\BigColorTwo	 #1}}}

\def\DWCombo#1#2%
{\vbox{
#1
#2
}
}

	 
	 
\setuphead
   [chapter]
   [align={center,nothyphenated},
   page=no,
   before={\page},
   after={\column},
   command={\DWCombo},
   deepnumbercommand={\DWNum},
   deeptextcommand={\DWTitle}]
	

\setuphead[section] % equivalente a H2
[
	align={flushleft, nothyphenated, verytolerant},
	before={\testpage[10][5cm]},	
	after={\blank[2pt]},	
	number=no,
	page=no,
	style=\hii % da notare che uso il font definito con \font\eulb pi� sopra. 
]   


\setuphead[subsection] % equivalente a H3
[
	number=no,
	page=no,
	align={flushleft, nothyphenated, verytolerant},
	before={\testpage[10][5cm]},
	after={\blank[2pt]},	
	style=\hiii
]   

\setuphead[subsubsection] % equivalente a H4
[
	align={flushleft, nothyphenated, verytolerant},
	before={\testpage[10][5cm]},	
	after={\blank[2pt]},	
	number=no,
	style=\hiv
] 

\definelist[chapter]  % come definisco la lista dei capitoli, sottocapitoli, sottotsottocapitoli
\setuplist[chapter][alternative=c,before=\blank,after=\blank,style=\eulbig]

\setuplist[subject][alternative=d]

\definelist[subsubject]
\setuplist[subsubject][alternative=d]


\definecombinedlist
[contents][chapter,subject]
[level=subject]

\setuppagenumbering[command=\mypagenumber,location={inmargin,text},alternative=doublesided]

\setupframedtexts[
	background=color,
	backgroundcolor=grigino,
	frame=off]

\setuptables[
	bodyfont=8pt
]

\setuppagenumbering[alternative=doublesided]
\definepapersize[halfhybrid][width=5.5in,height=21cm]
\definepapersize[hybrid][width=21cm,height=11in]
\definepapersize[ultrabook][width=13.66in,height=7.68in]
\definepapersize[notebook][width=12.8in,height=8in]
\definepapersize[ipad][width=10.24in,height=7.68in]
\setuppapersize[hybrid][hybrid] % this is default and may be omitted 	
\setuplayout
[grid=yes,
leftmargin=17.5mm,
rightmargin=12.5mm,
leftmargindistance=2.5mm,
rightmargindistance=2.5mm,
backspace=20mm,
cutspace=15mm,
header=5mm,
footer=7mm,
height=middle,
footerdistance=5mm,
headerdistance=3mm,
topspace=8mm,
bottomspace=10mm,
width=middle]

\def\mypagenumber#1%
{{\eulbig #1}}

\definetextbackground[intro]
[backgroundcolor=grigino,
backgroundoffset=0.1cm,
offset=0.1cm,
frame=off,
location=paragraph,
color=black]

\language[en]

\definetextbackground[intro2]
[backgroundcolor=grigino2,
backgroundoffset=0.1cm,
offset=0.1cm,
frame=off,
location=paragraph,
color=black]

% \defineoverlay[log][{\externalfigure[sfondo_rene.pdf][width=\overlaywidth,height=\overlayheight]}] 
% \defineoverlay[log1][{\externalfigure[sfondo.pdf][width=\overlaywidth,height=\overlayheight]}] 
% \setupbackgrounds[page][background=log]


\setupcolors[state=start]
\definecolor[gray80][r=.8,g=.8,b=.8]
\definecolor[backx][r=.9,g=.9,b=.9]
\definecolor[backred][r=.9,g=.3,b=.0]

\setupTABLE[row][first][background=color,backgroundcolor=backx,bodyfont=8pt,style={\tfb\tfx}]
\setupTABLE[background=color,frame=off,framecolor=white,bodyfont=8pt,style=\tfx]
\setupTABLE[row][odd][background=color,backgroundcolor=backx,bodyfont=8pt,style=\tfx]
\setupTABLE[option=stretch,bodyfont=8pt,style=\tfx]

\define[1]\exampleitem{
\blank[2pt]
{\backred{#1}}
\blank[2pt]
}

\setuppagenumbering[alternative=doublesided]

\setupheadertexts[]
\setupheadertexts[\setups{text a}][][][\setups{text b}]

\startsetups[text a]

  \rlap{\em \getmarking[chapter]}
  \hfill
  \llap{\em \getmarking[section]}
\stopsetups

\startsetups[text b]
  \rlap{\em \getmarking[section]}
  \hfill
  \llap{\em \getmarking[chapter]}
\stopsetups

\setupfootertexts[subsubject]

\widowpenalty = 5000
\clubpenalty = 5000


%----------------pagebackground----------------------------------------
\defineoverlay[bgright][{\externalfigure[right_background.jpg][width=\overlaywidth,height=\overlayheight]}] 
\defineoverlay[bgleft][{\externalfigure[left_background.jpg][width=\overlaywidth,height=\overlayheight]}] 
\defineoverlay[bgsbright][{\externalfigure[right_background.jpg][width=\overlaywidth,height=\overlayheight]}] 
\defineoverlay[bgsbleft][{\externalfigure[left_background.jpg][width=\overlaywidth,height=\overlayheight]}] 
\defineoverlay[bgintro][{\externalfigure[background-intro.jpg][width=\overlaywidth,height=\overlayheight]}] 
\defineoverlay[booktitle][{\externalfigure[booktitle.jpg][width=\overlaywidth,height=\overlayheight]}] 
%----------------pagebackground----------------------------------------



%----------------chapter thumb-----------------------------------
% questo e' codice TeX di alto livello, scritto da Hans Hagen di
% ConTeXt. L'ho preso e lievemente modificato

\unprotected \def\rawsectionnumber#1{\countervalue{\??se#1}} % will be in kernel

\setuppagenumbering
   [alternative=doublesided]

\setupcolors
   [state=start]


\defineoverlay
   [thumbs]
   [\setups{thumbs}]

\def\NOfChapters{45} % i need to make this one available

\startsetups thumbs
     \definemeasure[thumbheight][\dimexpr\paperheight/\NOfChapters\relax]
     \vbox{ 			%definisco un vbox per contenere pi� elementi verticali in uno.
     \vskip 7cm		%passo di 7cm
     \hbox{				
     \vbox to \paperheight \bgroup
         \vskip \rawsectionnumber{section-2} \measure{thumbheight}
         \vskip -1                           \measure{thumbheight}
         \doifoddpageelse\hfill\donothing
         \framed
             [background=color,
              backgroundcolor=darkred,
              frame=off,
              foregroundcolor=white,
              height=\measure{thumbheight},width=1cm]
             {\eulbig \rawsectionnumber{section-2}}
         \doifoddpageelse\donothing\hfill
         \vss
     \egroup
}}
\stopsetups


\defineparagraphs[MonsterName][n=1,frame=off,offset=none,width=broad,style={\ssbf},strut=no,before={\blank[4pt]},after={\nowhitespace}]
\defineparagraphs[MonsterDescription][n=1,frame=off,offset=none,width=broad,style={\rmit},strut=no,before={\nowhitespace},after={\nowhitespace}]
\defineparagraphs[MonsterQualities][n=1,frame=off,offset=none,width=broad,style={\rmbf},strut=no,before={\nowhitespace},after={\nowhitespace}]
\defineparagraphs[BasicMoveName][n=1,frame=off,offset=none,width=broad,style={\ssbf},strut=no,before={\nowhitespace},after={\nowhitespace}]
\defineparagraphs[Instruction][n=1,frame=off,offset=none,width=broad,style={\ss},strut=no,before={\nowhitespace},after={\nowhitespace}]
\defineparagraphs[Instructions][n=1,frame=off,offset=none,width=broad,style={\ss},strut=no,before={\nowhitespace},after={\nowhitespace}]
\defineparagraphs[InstructionsAfterHeader][n=1,frame=off,offset=none,width=broad,style={\ss},strut=no,before={\nowhitespace},after={\nowhitespace}]
\defineparagraphs[SpellName][n=1,align={raggedleft},frame=off,offset=none,width=broad,style={\ssbf},strut=no,before={\blank[4pt]},after={\nowhitespace}]
\defineparagraphs[SpellDescription][n=1,frame=off,offset=none,width=broad,style={\rmit},strut=no,before={\nowhitespace},after={\nowhitespace}]
\defineparagraphs[Requirements][n=1,frame=off,offset=none,width=broad,style={\rmit},strut=no,before={\nowhitespace},after={\nowhitespace}]
\defineframedtext[Example][frame=on,offset=4pt,width=broad,style={\rmit},strut=no]

\def\CMTags#1{{\ss{\BigColorTwo #1}}}
\def\CMSpellName#1{{\ssbf{\BigColorTwo #1}}}
%\setupindenting[medium,first,yes]
\parskip=4pt
