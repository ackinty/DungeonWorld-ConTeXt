\chapter{Fronts}
 \index{Fronts} \index{Fronts}
 



Fronts are secret tomes of GM knowledge. With the exception of a few sneaky PC tricks Fronts are your purview and are a place where you’ll build the adversaries, organizations, and other misfortune that the characters will come in conflict with. A Front is a collection of linked Dangers—threats to the characters specifically and to the people, places, and things the characters care about. It also includes one or more Impending Dooms, the horrible things that will happen without the players' intervention. “Fronts” comes, of course, from “fighting on two fronts” which is just where you want the characters to be—surrounded by threats, danger and adventure.

 

Fronts are built outside of active play. They’re the solo fun that you get to have between games—rubbing your hands and cackling evilly to yourself as you craft the foes with which to challenge your PCs. You may tweak or adjust the Fronts during play (who knows when inspiration will strike?) but the meat of them comes from preparation between sessions.

 

Fronts are designed to help you organize your thoughts on PC opposition. They’re here to contain your notes, ideas, and plans for these opposing forces. When you’re in a bind your Fronts are where you’re going to turn and say “oh, so {\em that} ’s what I should do”. Consider them an organizational tool, as inspiration for present and future mayhem.

 

When you’re building Fronts, think about all the creepy dungeon denizens, the rampaging hordes and ancient cults that you’d like to see in your game. Think in broad strokes at first and then, as you build Dangers into your Fronts, you’ll be able to narrow those ideas down. When you write your Campaign Front, think about session—to—session trends. When you write your Adventure Fronts, think about what’s important right here and right now. When you’re done writing a few Fronts you’ll be equipped with all the tools you’ll need to challenge your players and ready to run Dungeon World.

 
\subsection{Campaign and Adventure Fronts}  \index{Campaign and Adventure Fronts} \index{Campaign} \index{Adventure} \index{Fronts}
 

At their core, all Fronts contain the same components. They sort and gather your Dangers into easy—to—use clusters. There are, however, two different {\em kinds}  of Fronts available to you. On the session—to—session level there’s your Adventure Fronts. These Fronts will see use for 3 or 4 sessions each. They’re tied to one problem and will be dealt with or cast aside as the characters wander the dungeon or uncover the plot at hand. Think of them as episodic content: “Today, on Dungeon World…”

 

Tying your Adventure Fronts together is your Campaign Front. While the Adventure Fronts will contain immediate Dangers—the Orcs in Hargrosh Pass, say—the Campaign Front contains the Dark God Grishkar who drives the Orcs to their pillaging. The Campaign Front is the unifying element that spans your all the sessions of your Dungeon World game. It will have slower—burning Portents but they’ll be bigger in scope and have a deeper impact on the world. Most importantly they'll be scarier if they're allowed to resolve.

 

When a Danger from an Adventure Front goes without resolution you’ll have to make a decision. If the Danger is something you like and feel has a place in the larger story of your game don’t hesitate to move it to the Campaign Front. You’re able to make smaller Dangers that went unresolved into bigger Dangers some day later on. You can move Dangers from the Campaign Fronts to the an Adventure Front if you’re ready for the big showdown, too.

 
\section{Creating Fronts}  \index{Creating Fronts} \index{Fronts}
 

Here’s how a Front comes together:

 
\startitemize[1,packed]

\item Choose Campaign Front or Adventure Front

 
\item Create 2–3 Dangers

 
\item Choose their Impending Doom

 
\item (For an Adventure Front, 1–3 Grim Portents)

 
\item (For the Campaign Front, 3–5 Grim Portents)

 
\item Write 1–3 stakes questions

 
\item List the general cast of the Front


\stopitemize
 
\section{Creating Dangers}  \index{Creating Dangers} \index{Dangers}
 

Not every single element of your game will warrant a Danger—traps, some roving monsters and other bits of ephemera may just be there to add flavour but aren't important enough to warrant inclusion. That’s okay. Fronts are here to keep you appraised of the bigger picture. Dangers are divided into a handful of categories, each with its own name and {\bf impulse} .

 

Every Danger has a crucial motivation that drives it called its "impulse." The impulse exists to help you understand that Danger. What pushes it to fulfill its Impending Doom? Impulse can help you translation the Danger into action.

 

When creating Dangers for your Front, think about how each one interacts as a facet of the Front as a whole. Keep in mind the people, places, and things that might be a part of the threat to the world that the Front represents. How does each Danger contribute to the Front?

 
\startExample
Let's say we have an idea for a Front—an ancient portal has been discovered in the icy north. We'll call our Front "The Opening of the White Gate".
\stopExample
 

The easiest place to start is with people and monsters. Cultists, ogre chieftains, demonic overlords and the like are all excellent Dangers. These are the creatures that have risen above mere monster status to become serious threats on their own. Groups of monsters, too, can be Dangers—goblin tribes or a rampaging centaur khanate, for example.

 
\startExample
For the Front we're creating, we can pick a few different groups or people who might be interested in the gate. The College of Arcanists, perhaps. There's a golem, too, we've decided, that protects the forgotten portal. The golem is just an obstacle, so we won't make him a Danger.
\stopExample
 

Thinking more broadly, less obvious elements of the world can be Dangers. Blasted landscapes, intelligent magical items, ancient spells woven into the fabric of time. These things fulfill the same purposes as a mad necromancer—they're part of the Front, a Danger to the world.

 
\startExample
For our Front, we'll add the Gate itself.
\stopExample
 

Lastly, if we think ahead, we can include some overarching Dangers. The sorts of things that are in play outside the realm of the obvious—godly patrons, hidden conspiracies and cursed prophecies waiting to be fulfilled.

 
\startExample
Perhaps the White Gate was carved in the ancient past, hidden by a race of Angels until the Day of Judgement. We'll add the "Argent Seraphim" to our Front.
\stopExample
 

Of course, there's so much more I could add to my Front, but there's two reasons not to go overboard: firstly, I want to leave room for discovery. Like a map, blank spaces can always be filled in later. Leaving room for player contribution and future inspiration means I'll have freedom to alter the Front and make it fit the game as the story emerges. Secondly, not every bad thing that could happen deserves Danger—hood. If you're uncertain, think about it this way: Dangers can always get worse.

  
\startExample
A barbarian tribe near the Gate, the frozen tundra itself, a band of rival adventurers; all these things could be dangerous elements of the game but they're not important enough just yet to deserve to be Dangers.
\stopExample
 

Creating Dangers is a way to slice up your overall Front concept into smaller, easier to manage pieces. Dangers are a tool for adding detail to the right parts of the Front and for making the Front easier to manage in the long run.

 

Once you've named and added a Danger to the Front you need to choose a type for that Danger from the list below. Alternately you can use the list of types to inspire Dangers: with your Front in mind, peruse the list and pick one or two that fit.

 
\startExample
For our three Dangers (The College of Arcanists, The White Gate and the Argent Seraphim) we've selected Cabal, Dark Portal and Choir of Angels, respectively.
\stopExample
 
\section{Types of Dangers}  \index{Types of Dangers} \index{Types} \index{Dangers}
 
\startitemize[1,packed]

\item Ambitious Organizations

 
\item Planar Forces

 
\item Arcane Enemies

 
\item Hordes

 
\item Cursed Places


\stopitemize
 
\subsection{Ambitious Organizations}  \index{Ambitious Organizations} \index{Ambitious} \index{Organizations}
 
\startitemize[1,packed]

\item Misguided Good ({\em impulse: to do “what is right” no matter the cost} )

 
\item Thieves Guild ({\em impulse: to take by subterfuge} )

 
\item Cult ({\em impulse: to infest from within} )

 
\item Religious Organization ({\em impulse: to establish and follow doctrine} )

 
\item Corrupt Government ({\em impulse: to maintain the status quo} )

 
\item Cabal ({\em impulse: to absorb those in power, to grow} )


\stopitemize
 
\subsubsection{GM Moves for Ambitious Organizations}  \index{GM Moves for Ambitious Organizations} \index{Gm} \index{Moves} \index{Ambitious} \index{Organizations}
 
\startitemize[1,packed]

\item Attack someone by stealthy means (kidnapping, etc.)

 
\item Attack someone directly (with a gang or single assailant)

 
\item Absorb or buy out an someone important (an ally, perhaps)

 
\item Influence a body of control (change a law, manipulate doctrine)

 
\item Establish a new rule (within the organization)

 
\item Claim territory or resources

 
\item Negotiate a deal

 
\item Observe a potential foe in great detail


\stopitemize
 
\subsection{Planar Forces}  \index{Planar Forces} \index{Planar} \index{Forces}
 
\startitemize[1,packed]

\item God ({\em impulse: to gather worshippers} )

 
\item Demon Prince ({\em impulse: to open the gates of Hell} )

 
\item Elemental Lord ({\em impulse: to tear down creation to its component parts} )

 
\item Force of Chaos ({\em impulse: to destroy all semblance of order} )

 
\item Choir of Angels ({\em impulse: to pass judgement} )

 
\item Construct of Law ({\em impulse: to eliminate perceived disorder} )


\stopitemize
 
\subsubsection{GM Moves for Planar Forces}  \index{GM Moves for Planar Forces} \index{Gm} \index{Moves} \index{Planar} \index{Forces}
 
\startitemize[1,packed]

\item Turn an organization (corrupt or infiltrate with influence)

 
\item Give dreams of prophecy

 
\item Lay a Curse on a foe

 
\item Extract a promise in exchange for a boon

 
\item Attack indirectly, through intermediaries

 
\item Rarely, when the stars are right, attack directly

 
\item Foster rivalries with other, similar powers

 
\item Expose someone to a Truth, wanted or otherwise


\stopitemize
 
\subsection{Arcane Enemies}  \index{Arcane Enemies} \index{Arcane} \index{Enemies}
 
\startitemize[1,packed]

\item Lord of the Undead ({\em impulse: to seek true immortality} )

 
\item Power-mad Wizard ({\em impulse: to seek magical power} )

 
\item Sentient Artifact ({\em impulse: to find a worthy wielder} )

 
\item Ancient Curse ({\em impulse: to ensnare} )

 
\item Chosen One ({\em impulse: to fulfill or resent their destiny} )

 
\item Dragon ({\em impulse: to hoard gold and jewels, to protect the clutch} )


\stopitemize
 
\subsubsection{GM Moves for Arcane Enemies}  \index{GM Moves for Arcane Enemies} \index{Gm} \index{Moves} \index{Arcane} \index{Enemies}
 
\startitemize[1,packed]

\item Learn forbidden knowledge

 
\item Cast a spell over time and space

 
\item Attack a foe with magic, directly or otherwise

 
\item Spy on someone with a scrying spell

 
\item Recruit a follower or toady

 
\item Tempt someone with promises

 
\item Demand a sacrifice


\stopitemize
 
\subsection{Hordes}  \index{Hordes} \index{Hordes}
 
\startitemize[1,packed]

\item Wandering Barbarians ({\em impulse: to grow strong, to drive their enemies before them} )

 
\item Humanoid Vermin ({\em impulse: to breed – to multiply and consume} )

 
\item Underground Dwellers ({\em impulse: to defend the complex from outsiders} )

 
\item Plague of the Undead ({\em impulse: to spread} )


\stopitemize
 
\subsubsection{GM Moves for Hordes}  \index{GM Moves for Hordes} \index{Gm} \index{Moves} \index{Hordes}
 
\startitemize[1,packed]

\item Assault a bastion of civilization

 
\item Embrace internal chaos

 
\item Change direction suddenly

 
\item Overwhelm a weaker force

 
\item Display a show of dominance

 
\item Abandon an old home, find a new one

 
\item Grow in size by breeding or conquest

 
\item Appoint a champion

 
\item Declare war and act upon that declaration without hesitation or deliberation


\stopitemize
 
\subsection{Cursed Places}  \index{Cursed Places} \index{Cursed} \index{Places}
 
\startitemize[1,packed]

\item Abandoned Tower ({\em impulse: to draw in the weak-willed} )

 
\item Unholy Ground ({\em impulse: to spawn evil} )

 
\item Elemental Vortex ({\em impulse: to grow, to tear apart reality} )

 
\item Dark Portal ({\em impulse: to disgorge demons} )

 
\item Shadowland ({\em impulse: to corrupt or consume the living} )

 
\item Place of Power ({\em impulse: to be controlled or tamed} )


\stopitemize
 
\subsubsection{GM Moves for Cursed Places}  \index{GM Moves for Cursed Places} \index{Gm} \index{Moves} \index{Cursed} \index{Places}
 
\startitemize[1,packed]

\item Vomit forth a lesser monster

 
\item Spread to an adjacent place

 
\item Draw the attention of a curious party

 
\item Grow in intensity or depth

 
\item Leave a lingering effect on an inhabitant or visitor

 
\item Hide something from sight

 
\item Offer power

 
\item Dampen magic or increase its effects

 
\item Confuse or obfuscate truth or direction

 
\item Corrupt a natural law


\stopitemize
 
\subsection{Description and Cast}  \index{Description and Cast} \index{Description} \index{Cast}
 

Write up something short to remind you just what this Danger is about; something to describe it in a nutshell. Don’t worry about where it’s going or what could happen—Grim Portents and the Impending Doom will handle that for you, you'll get to those in a bit. If there are multiple people involved in the Danger (an orc warlord and his clansmen, a hateful God and his servants) go ahead and give them a name and a detail or two now. Leave yourself some space as you'll be adding to this section as you play.

 
\subsection{Custom Moves}  \index{Custom Moves} \index{Custom} \index{Moves}
 

Sometimes, a Danger will require some particular move that might not exist yet. Write one or two you think you might need, now. They may be player moves or GM moves, as you see fit. Of course, if you're writing a player move, keep your (the GMs) hands off the dice and keep in mind the basic structure of a move. A 10+ is a complete success; a 7–9 is a partial success. On a miss, maybe the custom move does something specific, or maybe not—maybe you just get to make a move or work towards fulfilling a Grim Portent. The formatting of these moves varies from move to move. See the Advanced Delving chapter for details on how to create your own. 

 
\startExample
For the Opening of the White Gate, I just know some fool PC is going to end up in the light that spills from the gate, so I'm writing a move to show what might occur.
\stopExample
 
\subsection{Grim Portents}  \index{Grim Portents} \index{Grim} \index{Portents}
 

Grim Portents are dark designs for what could happen if a Danger goes unchecked. Imagine yourself a kind of diviner working some scrying spell into the future of your campaign or the adventure that the characters are undertaking. Think about what would happen if the Danger existed in the world but the PCs didn’t. If all these awful things you’ve conjured up had their run of the world. Scary, huh? The Grim Portents are your way to codify the plans and machinations of your Dangers. A Grim Portent can be a single interesting event or a chain of steps. When you’re not sure what to do next push your Danger towards resolving a Grim Portent.

 

More often than not, each Portent relies on its predecessors to resolve. The Orcs tear down the city only after the peace talks fail, for example. A simple Front will progress from bad to worse to much worse in a clear path forward. Sometimes, Grim Portents are unconnected pathways to the Impending Doom. The early manifestations of Danger might not all be related. It's up to you how complex your Front will be. Whenever a Danger comes to pass, check the other Dangers in the Front. In a complex Front, you may need to cross off or alter the Grim Portents. That's fine, you're allowed.

 

Think of your Grim Portents as possible moves waiting in the wings. When the time is right, unleash them on the world.

 
\startExample
I've chosen a few Grim Portents for my new Front. 
\startitemize[1,packed]

\item The College sends an expedition to the Gate

 
\item The Key is discovered

 
\item The First Trumpet sounds

 
\item A Champion is chosen

 
\item The Second Trumpet sounds

 
\item The Herald appears

 
\item The Gate is Opened


\stopitemize

\stopExample
 

Grim Portents are the sword of inevitability that hangs over the characters. The struggles that they fight every day on the field of battle, in the courts of society and in the deepest dungeons of the world. Think about how the Dangers contribute to the Front you're creating, both externally (the effect of the Danger on the world around it) and internally (the politics or struggles of its parts). Keep scale in mind, too. Grim Portents don’t all have to be world—shaking. They can simply represent a change in direction for a Danger. Some new way for it to cause trouble in the world.

 

When a Grim Portent comes to pass, check it off—the Portent is a part of the world, now. The prophecy has come true! A Portent that has come to pass might have ramifications for your other Fronts, too. Have a quick look when your players aren’t demanding your attention and feel free to make changes. One small Portent may resound across the whole Campaign in subtle ways.

 

You can advance a Grim Portent descriptively or prescriptively. Descriptively means that, through play, you've seen the change happen, so you mark it off. Maybe the players sided with the goblin tribes against their lizardmen enemies—now the goblins control the tunnels. Lo and behold, this was the next step in a Grim Portent. Prescriptive is when, due to a failed player move or a golden opportunity, you advance the Grim Portent as your hard move. That step comes to pass, show its effects and keep on asking "what do you do, now?"

 
\subsection{Impending Doom}  \index{Impending Doom} \index{Doom}
 

At the end of every Danger's path is an Impending Doom. This is the final toll of the bell that signals the Danger’s triumphant resolution. When a Grim Portent comes to pass the Impending Doom grows stronger, more apparent and present in the world. These are the Very Bad Things that every Danger, in some way, seeks to bring into effect. Choose one of the types of Impending Dooms and give it a concrete vision in your Front. These may change in play—often they will, as the characters meddle in the affairs of the world. Don't fret, you can change them later.

 
\startitemize[1,packed]

\item Tyranny (of the strong over the weak or the few over the many)

 
\item Pestilence (the spread of sickness and disease, the end of wellness)

 
\item Destruction (apocalypse, ruin and woe)

 
\item Usurpation (the chain of order comes apart, someone rightful is displaced)

 
\item Impoverishment (enslavement, the abandonment of goodness and right)

 
\item Rampant Chaos (laws of reality, of society, or any order is dissolved)


\stopitemize
 

When all of the Grim Portents of a danger come to pass, the Impending Doom sets in. The Danger is then resolved but the setting has changed in some drastic way—even on a small level. This will almost certainly change the Front at large, as well. Making sure that these effects are felt and significant to the NPCs, places, and life of the campaign world is a big part of making them feel real.

 
\subsection{Stakes}  \index{Stakes} \index{Stakes}
 

Your stakes questions are 1–3 questions about people, places, or groups that you're interested in. People include characters and NPCs, your choice. Remember that your agenda includes "Play to find out what happens?" Well this is a way of reminding yourself what you want to find out.

 

Stakes are concrete and clear. Don't write stakes about vague feelings or incremental changes. Stakes are about important changes that affect the PCs and the world. A good stakes question is one that, when it's resolved, means that things will never be the same again.

 

The most important thing about stakes is that you find them interesting. Your stakes should be things that you genuinely want to know, but that you're also willing to not decide. Once you've written it as a stake, it's out of your hands, you don't get to just make it up anymore. Now you have to play to find out.

 

Playing to find out is one of the biggest rewards of playing Dungeon World. You've written down something tied to events happening in the world that you want to find out about—now you get to do just that.

 

Once you have your stakes your front is ready to play.

 
\startExample
My stakes questions include, as tailored to my group: 
\startitemize[1,packed]

\item Who will be the Champion?

 
\item How will Lux respond to the Light from Beyond?

 
\item Will the College be able to recruit Avon?


\stopitemize

\stopExample
 
\section{Resolving a Front}  \index{Resolving a Front} \index{Front}
 

Often a Front will be resolved in a simple and straightforward manner. A Front representing a single dungeon may have its Dangers killed, turned to good, or overcome by some act of heroism. In this case the Front is dissolved and set aside. Maybe there are elements of the Front—Dangers that go unresolved or leftover members of a Danger that’s been cleared—that lives on. Maybe they move to the Campaign Front as brand new Dangers?

 

The Campaign Front will need a bit more effort to resolve. It’ll be working slowly and subtly as the course of the Campaign rolls along. You won’t introduce or resolve it all at once, but in pieces. The characters work towards defeating the various minions of the Big Bad that lives in your Campaign Front. In the end, though, you’ll know that the Campaign Front is resolved when the Dark God is confronted, the Undead Plague is wiped clear, and the heroes emerge bloodied but victorious. Campaign Fronts take longer to deal with, but in the end they’re the most satisfying to resolve.

 

When a Front is resolved, take some extra time to sit down and look at the aftermath. Did any Grim Portents come to pass? Even if a Danger is stopped, if one or two Grim Portents are fulfilled, the world is changed. Keep this in mind when you write your future Fronts. Is there anyone who could be moved from the now—defeated Front somewhere else? Anyone get promoted or reduced in stature? The resolution of a Front is an important event!

 

When you resolve an Adventure Front, usually that means the adventure itself has been resolved. This is a great time to take a break and look at The Campaign Front you have. Let it inspire your next Adventure Front. Write up a new Adventure Front or polish off one you’ve been working on, draw a few maps to go with it and get ready for the next big thing.

 
\section{Multiple Adventure Fronts}  \index{Multiple Adventure Fronts} \index{Multiple} \index{Adventure} \index{Fronts}
 

As you start your campaign you're likely to have a lightly-detailed Campaign Front and a single, detailed Adventure Front. Characters may choose, part-way through an Adventure, to pursue some other course. You might end up with a handful of partly-resolved Adventure Fronts. Not only is this okay, it's a great way to explore a world that feels alive and organic. Always remember, Fronts continue along apace no matter whether the characters are there to see it or not. Think offscreen, especially where Fronts are concerned.

 

When running two Adventure Fronts at the same time they can be intertwined or independent. The Anarchists corrupting the city from the inside are a different Front from the orcs massing outside the walls, but they'd both be in play at once. On the other hand one dungeon could have multiple Fronts at play within its walls: the powers and effects of the cursed place itself and the warring humanoid tribes that inhabit it.

 

A situation warrants multiple Adventure Fronts when there are multiple Impending Dooms, all equally potent but not necessarily related. The Impending Doom of the Anarchists is chaos in the city, the Impending Doom of the orcs is its utter ruination. They are two separate Fronts with their own Dangers. They'll deal with each other, as well, so there's some room for the players choosing sides or attempting to turn the Dangers of one Front against the other.

 

When dealing with multiple Adventure Fronts the players are likely to prioritize. The cult needs attention now, the orcs can wait, or vice versa. These decisions lead to the slow advancement of the neglected Front, eventually causing more problems for the players and leading to new adventures. This can get complex once you've got three or four Fronts in play. Take care not to get overwhelmed.

 
