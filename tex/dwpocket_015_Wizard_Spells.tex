\chapter{Wizard Spells}
 \index{Wizard Spells} \index{Wizard} \index{Spells}
 


\section{Cantrips}  \index{Cantrips} \index{Cantrips}
 
\startSpellName
\CMSpellName{Light} 	Cantrip
\stopSpellName
 

An item you touch glows with arcane light, about as bright as a torch. It gives off no heat or sound and requires no fuel, but it is otherwise like a mundane torch. You have complete control of the color of the flame. The spell lasts as long as it is in your presence.

 
\startSpellName
\CMSpellName{Unseen Servant} 	Cantrip
\stopSpellName
 

You conjure a simple invisible construct that can do nothing but carry items. It has Load 2 and carries anything you hand to it. It cannot pick up items on its own and can only carry those you give to it. Items carried by an unseen servant appear to float in the air a few paces behind you. An unseen servant that takes damage or leaves your presence is immediately dispelled.

 
\startSpellName
\CMSpellName{Prestidigitation} 	Cantrip
\stopSpellName
 

You perform minor tricks of true magic. If you touch an item as part of the casting you can make cosmetic changes to it: clean it, soil it, cool it, warm it, flavor it, or change its color. If you cast the spell without touching an item you can instead create minor illusions no bigger than yourself. Prestidigitation illusions are crude and clearly illusions; they won't fool anyone, but they might entertain them.



 


\section{1st Level Spells}  \index{1st Level Spells} \index{1st} \index{Level} \index{Spells}
 
\startSpellName
\CMSpellName{Contact Spirits} 	Level 1	Summoning
\stopSpellName
 

Name the spirit you wish to contact (or leave it to the GM). You pull that creature through the planes, just close enough to speak to you. It is bound to answer any one question you ask to the best of its ability.

 
\startSpellName
\CMSpellName{Detect Magic} 	Level 1	Divination
\stopSpellName
 

One of your senses is briefly attuned to magic. The GM will tell you what here is magical.

 
\startSpellName
\CMSpellName{Telepathy} 	Level 1	Divination
\stopSpellName
 

You form a telepathic bond, allowing you to speak to the person you touch with this spell through your thoughts. You can only have one telepathic bond at a time.

 
\startSpellName
\CMSpellName{Charm Person} 	Level 1	Enchantment
\stopSpellName
 

The person (not beast or monster) you touch while casting this spell counts you as a friend until they take damage or you prove otherwise.

 
\startSpellName
\CMSpellName{Invisibility} 	Level 1	Illusion
\stopSpellName
 

Touch an ally: nobody can see them. They're invisible! The spell persists until the target attacks or you dismiss the effect. While the spell is ongoing, you can't cast another spell.

 
\startSpellName
\CMSpellName{Magic Missile} 	Level 1	Evocation
\stopSpellName
 

Projectiles of pure magic spring from your fingers. Deal 2d4 damage to one target.

 
\startSpellName
\CMSpellName{Alarm} 	Level 1
\stopSpellName
 

Walk a wide circle. Until you prepare spells again your magic will alert you if a creature crosses that circle. Even if you are asleep, the spell will shake you from your slumber.



 


\section{3rd Level Spells}  \index{3rd Level Spells} \index{3rd} \index{Level} \index{Spells}
 
\startSpellName
\CMSpellName{Dispel Magic} 	Level 3
\stopSpellName
 

Choose a spell or magic effect in your presence: this spell rips it apart. Lesser spells are ended, powerful magic is just reduced or dampened so long as you are nearby.

 
\startSpellName
\CMSpellName{Visions Through Time} 	Level 3	Divination
\stopSpellName
 

Cast this spell and gaze into a reflective surface to see into the depths of time. The GM will reveal the details of a Grim Portent to you—a bleak event that will come to pass if not directly stopped. They'll tell you something useful about how you can interfere with the Grim Portent's dark outcomes. Rare is the Portent that claims "You'll live happily ever after." Sorry.

 
\startSpellName
\CMSpellName{Fireball} 	Level 3	Evocation
\stopSpellName
 

You evoke a mighty ball of flame that envelops your target and everyone nearby, inflicting 2d6 damage which ignores armor.

 
\startSpellName
\CMSpellName{Mimic} 	Level 3
\stopSpellName
 

You take the form of someone you touch while casting this spell. Your physical characteristics match theirs exactly but your behavior may not. This change persists until you take damage or choose to return to your own form. While this spell is ongoing, you lose access to all your wizard moves.

 
\startSpellName
\CMSpellName{Mirror Image} 	Level 3	Illusion
\stopSpellName
 

You create an illusory image of yourself. The next attack against you effects the illusory image, not you. The image then dissipates.

 
\startSpellName
\CMSpellName{Sleep} 	Level 3	Enchantment
\stopSpellName
 

1d4 enemies you can see of the GM's choice fall asleep. Only creatures capable of sleeping are effected. They awake as normal: loud noises, jolts, pain.



 


\section{5th Level Spells}  \index{5th Level Spells} \index{5th} \index{Level} \index{Spells}
 
\startSpellName
\CMSpellName{Cage} 	Level 5	Evocation
\stopSpellName
 

The target is held in a cage of magical force. Nothing can get in or out of the cage. The cage remains until you cast another spell or dismiss it. While the spell is ongoing, the caged creature can hear your thoughts and you cannot leave sight of the cage.

 
\startSpellName
\CMSpellName{Contact Other Plane} 	Level 5	Divination
\stopSpellName
 

You send a request to another plane. Specify what you'd like to contact by location, type of creature, name, or title. You open a two-way communication with that creature. Your communication can be cut off at any time by you or the creature you contacted.

 
\startSpellName
\CMSpellName{Polymorph} 	Level 5	Enchantment
\stopSpellName
 

Your touch reshapes a creature entirely, they stay in the form you craft until you Cast a Spell. Describe to the GM the new shape you craft, including any stat changes, significant adaptations, or major weaknesses. The GM will then tell you one or more of these:

 
\startitemize[1,packed]

\item The form will be unstable and temporary

 
\item The creature's mind will be altered as well

 
\item The form has an unintended benefit or weakness


\stopitemize
 
\startSpellName
\CMSpellName{Summon Monster} 	Level 5	Summoning
\stopSpellName
 

A monster appears and aids you as best it can. Treat it as your character, but with access to only the basic moves. It has +1 modifier for all stats and 1 HP. Choose the type of monster by choosing 1d6 statements from the list below. The GM will tell you the type of monster you get based on your choices:

 
\startitemize[1,packed]

\item The monster has +2 instead of +1 to one stat

 
\item The monster is not reckless

 
\item The monster does 1d8 damage

 
\item The monster's bond to your plane is strong, +3 HP for each level you have

 
\item The monster has some useful adaptation


\stopitemize
 

The creature remains on this plane until it dies or you dismiss it. While the spell is ongoing, you take -1 to Cast a Spell.



 


\section{7th Level Spells}  \index{7th Level Spells} \index{7th} \index{Level} \index{Spells}
 
\startSpellName
\CMSpellName{Dominate} 	Level 7	Enchantment
\stopSpellName
 

Your touch pushes your mind into someone else's. You gain 1d4 hold. Spend one hold to make the target take one of these actions:

 
\startitemize[1,packed]

\item Speak a few words of your choice

 
\item Give you something they hold

 
\item Make a concerted attack on a target of your choice

 
\item Truthfully answer one question


\stopitemize
 

If you run out of hold the spell ends. If the target takes damage you lose 1 hold. While the spell is ongoing you cannot Cast a Spell.

 
\startSpellName
\CMSpellName{True Seeing} 	Level 7	Divination
\stopSpellName
 

You see all things as they truly are. This effect persists until you tell a lie or dismiss the spell. While this spell is ongoing, you take -1 to Cast a Spell.

 
\startSpellName
\CMSpellName{Shadow Walk} 	Level 7	Illusion
\stopSpellName
 

The shadows you target with this spell become a portal for you and your allies. Name a location, describing it with a number of words up to your level. Stepping through the portal deposits you and any allies present when you cast the spell at the location you described. The portal may only be used once by each ally.

 
\startSpellName
\CMSpellName{Contingency} 	Level 7	Evocation
\stopSpellName
 

Choose a 5th level or lower spell you know. Describe a trigger condition using a number of words equal to your level. The chosen spell is held until you choose to unleash it or the trigger condition is met, whichever happens first. You don't have to roll for the held spell, it just takes effect. While you have a contingent spell, you can't gain another one.

 
\startSpellName
\CMSpellName{Cloudkill} 	Level 7	Summoning
\stopSpellName
 

A cloud of fog drifts into this realm from beyond the Black Gates of Death, filling the immediate area. Whenever a creature in the area takes damage it takes an extra 1d6 damage which ignores armor. This spell persists so long as you can see the affected area, or until you dismiss it.



 


\section{9th Level Spells}  \index{9th Level Spells} \index{9th} \index{Level} \index{Spells}
 
\startSpellName
\CMSpellName{Antipathy} 	Level 9	Enchantment
\stopSpellName
 

Choose a target and describe a type of creature or an alignment. Creatures of the specified type or alignment cannot come within sight of the target. If a creature of the specified type does find itself within site of the target, it immediately flees. This effect continues until you leave the target's presence or you dismiss the spell. While the spell is ongoing, you take -1 to Cast a Spell.

 
\startSpellName
\CMSpellName{Alert} 	Level 9	Divination
\stopSpellName
 

Describe an event. The GM will tell you when that event occurs, no matter where you are or how far away the event is. If you choose, you can view the location of the event as though you were there in person. You can only have one Alert active at a time.

 
\startSpellName
\CMSpellName{Soul Gem} 	Level 9
\stopSpellName
 

You trap the soul of a dying creature within a gem. The trapped creature is aware of its imprisonment but can still be manipulated through spells, Parley, and other effects. All moves against the trapped creature are at +1. You can free the soul at any time but it can never be recaptured once freed.

 
\startSpellName
\CMSpellName{Shelter} 	Level 9	Evocation
\stopSpellName
 

You create a structure out of pure magical power. It can be as large as a castle or as small as a hut, but is impervious to all non-magical damage. The structure endures until you leave it or you end the spell.

 
\startSpellName
\CMSpellName{Perfect Summons} 	Level 9	Summoning
\stopSpellName
 

You teleport a creature to your presence. Name a creature or give a short description of a type of creature. If you named a creature, that creature appears before you. If you described a type of creature, a creature of that type appears before you.








