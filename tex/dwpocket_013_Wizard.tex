\chapter{Wizard}
 \index{Wizard} \index{Wizard}
            

         

Dungeon World has rules. Not the laws of men or the rule of some petty tyrant. Bigger, better rules. You drop something—it falls. You can’t make something out of nothing. The dead stay dead, right?

         

Oh, the things we tell ourselves to feel better about the long, dark nights.

         

You’ve spent so very long poring over those tomes of yours. The experiments that nearly drove you mad and all the botched summonings that endangered your very soul. For what? For power. What else is there? Not just the power of King or Country but the power to boil a man's blood in his veins. To call on the thunder of the sky and the churn of the roiling earth. To shrug off the rules the world holds so dear.

         

Let them cast their sidelong glances. Let them call you “warlock” or “diabolist.” Who among them can hurl fireballs from their eyes?

         

Yeah. We didn’t think so.

       

       
\section{Names}  \index{Names} \index{Names}
       

         

           {\em Elf} : Galadiir, Fenfaril, Lilliastre, Phirosalle, Enkirash, Halwyr

         

           {\em Human} : Avon, Morgan, Rath, Ysolde, Ovid, Vitus, Aldara, Xeno, Uri

       

       
\section{Look}  \index{Look}
       

         

Choose one for each:

         

Haunted, Sharp, or Crazy Eyes

         

Styled Hair, Wild Hair, or Pointed Hat

         

Worn, Stylish, or Strange Robes

         

Pudgy, Creepy, or Thin Body

       

       
\section{Stats}  \index{Stats} \index{Stats}
       

         

Assign these scores to your stats:

         

17 (+2), 15 (+1), 13 (+1), 11 (+0), 9 (+0), 8 (-1)

         

You start with 4+Constitution HP.

       

       

Your base damage is d4.

       
\section{Starting Moves}  \index{Starting Moves} \index{Moves}
       

         
\startInstructionsAfterHeader
Choose a racial move:
\stopInstructionsAfterHeader
         

           
\subsection{Elf}  \index{Elf} \index{Elf}
           

Choose one cleric spell, you can cast it as if it was a wizard spell.

           
\subsection{Human}  \index{Human} \index{Human}
           

When you cast a Summoning spell take +1.

         

         

           
\startInstructions
You start with these moves:
\stopInstructions
           
\subsection{Spellbook}  \index{Spellbook} \index{Spellbook}
           

You have mastered several spells and inscribed them in your spellbook. You start out with three first level spells in your spellbook as well as the cantrips. Whenever you gain a level, you add a new spell of your level or lower to your spellbook. You spellbook is 1 weight.

           
\subsection{Prepare Spells}  \index{Prepare Spells} \index{Prepare} \index{Spells}
           

When you spend uninterrupted time (an hour or so) in quiet contemplation of your spellbook, you lose any spells you already have prepared and prepare new spells of your choice from your spellbook whose total levels don't exceed your own+1. You also prepare your cantrips; they don't count against your limit.

           
\subsection{Cast A Spell (Int)}  \index{Cast A Spell (Int)} \index{Cast} \index{Spell} \index{(int)}
           

When you release a spell you've prepared, roll+Int. On a 10+, the spell is successfully cast and you do not forget the spell—you may cast it again later. On a 7-9, the spell is cast, but choose one:

           
\startitemize[1,packed]
             
\item You draw unwelcome attention or put yourself in a spot (the GM will describe it)

             
\item The spell disturbs the fabric of reality as it is cast—take -1 ongoing to Cast a Spell until you Prepare Spells again.

             
\item After it is cast, the spell is forgotten. You cannot cast the spell again until you Prepare Spells.

           
\stopitemize
           
\subsection{Spell Defense}  \index{Spell Defense} \index{Spell} \index{Defense}
           

When you craft an ongoing spell into a makeshift shield of arcane energy to deflect an attack, the spell is ended and you subtract the spell's level from the damage done to you.

           
\subsection{Ritual}  \index{Ritual} \index{Ritual}
           

When you draw on a place of power to create a magical effect, tell the GM what you're trying to achieve. The GM will tell you "yes, you can do that, but..." and then 1 to 4 of the following:

           
\startitemize[1,packed]
             
\item It's going to take days/weeks/months

             
\item First you must \_\_\_\_

             
\item You'll need help from \_\_\_\_

             
\item It will require a lot of money

             
\item The best you can do is a lesser version, unreliable and limited

             
\item You and your allies will risk danger from \_\_\_\_

             
\item You'll have to disenchant \_\_\_\_ to do it

           
\stopitemize
         

       

       
\section{Alignment}  \index{Alignment} \index{Alignment}
       
\startInstructionsAfterHeader
Choose an alignment:
\stopInstructionsAfterHeader
       

         
\subsection{Good}  \index{Good} \index{Good}
         

Use magic to directly aid another

         
\subsection{Neutral}  \index{Neutral} \index{Neutral}
         

Discover something about a magical mystery

         
\subsection{Evil}  \index{Evil} \index{Evil}
         

Use magic to cause terror and fear

       

       
\section{Gear}  \index{Gear} \index{Gear}
       

         

Your Load is 5+Str. You start with your spellbook (1 weight) and dungeon rations (1 weight, 5 uses). Choose your defenses:

         
\startitemize[1,packed]
           
\item Leather armor (1 armor, 1 weight)

           
\item Bag of books (5 uses, 2 weight) and 3 healing potions

         
\stopitemize
         

Choose your weapon:

         
\startitemize[1,packed]
           
\item Dagger (Hand, 1 weight)

           
\item staff (Close, two-handed, 1 weight)

         
\stopitemize
         

Choose one:

         
\startitemize[1,packed]
           
\item healing potion

           
\item three antitoxin

         
\stopitemize
       

       
\section{Bonds}  \index{Bonds} \index{Bonds}
       

         

Fill in the name of one of your companions in at least one:

         

\thinrules[n=2] will play an important role in the events to come. I have foreseen it!

         

\thinrules[n=2] is keeping an important secret from me.

         

\thinrules[n=2] is woefully misinformed about the world; I will teach them all that I can.

       

       
\section{Advanced Moves}  \index{Advanced Moves} \index{Advanced} \index{Moves}
       

         
\startInstructionsAfterHeader
When you gain a level from 2-5, choose from these moves. You also add a new spell to your spellbook at each level.
\stopInstructionsAfterHeader
         
\subsection{Prodigy}  \index{Prodigy} \index{Prodigy}
         

Choose a spell. You prepare that spell as if it were one level lower.

         
\subsection{Empowered Magic}  \index{Empowered Magic} \index{Empowered} \index{Magic}
         

When you Cast a Spell, on a 10+ you have the option of choosing from the 7-9 list. If you do, you may choose one of these as well:

         
\startitemize[1,packed]
           
\item The spell's effects are maximized

           
\item The spell's targets are doubled

         
\stopitemize
         
\subsection{Fount of Knowledge}  \index{Fount of Knowledge} \index{Fount} \index{Knowledge}
         

When you Spout Lore about something no one else has any clue about, take +1.

         
\subsection{Know-It-All}  \index{Know-It-All} \index{Know-it-all}
         

When another player's character comes to you for advice and you tell them what you think is best, they get +1 forward when following your advice and you mark experience if they do.

         
\subsection{Expanded Spellbook}  \index{Expanded Spellbook} \index{Expanded} \index{Spellbook}
         

Add a new spell from any class to your spellbook.

         
\subsection{Eldritch Touch}  \index{Eldritch Touch} \index{Eldritch} \index{Touch}
         

When you touch a living creature, skin to skin, you can ask the GM what the last spell to affect that creature was.

         
\subsection{Logical}  \index{Logical} \index{Logical}
         

When you use strict deduction to analyze your surroundings, you can Discern Realities with Int instead of Wis.

         
\subsection{Arcane Ward}  \index{Arcane Ward} \index{Arcane} \index{Ward}
         

As long as you have at least one prepared spell, you have +2 armor.

         
\subsection{Counterspell}  \index{Counterspell} \index{Counterspell}
         

When you are affected by arcane magic you may attempt to counter the spell. Stake one of your prepared spells of equal or higher level on the defense and roll+Int. On a 10+, the spell is countered and has no effect on you. On a 7-9, the spell is countered and you forget the spell you staked. If the spell has other targets they are effected as usual.

         
\subsection{Quick Study}  \index{Quick Study} \index{Quick} \index{Study}
         

When you see the effects of an arcane spell, ask the GM the name of the spell and its effects. You take +1 when acting on the answers.

         
\startInstructions
When you gain a level from 6-10, choose from these moves or the level 2-5 moves.
\stopInstructions
         
\subsection{Master}  \index{Master} \index{Master}
         

Requires: Prodigy

         

Choose a spell. You prepare that spell as if it were one level lower.

         
\subsection{Greater Empowered Magic}  \index{Greater Empowered Magic} \index{Greater} \index{Empowered} \index{Magic}
         

Replaces: Empowered Magic

         

When you Cast a Spell, on a 10-11 you have the option of choosing from the 7-9 list. If you do, you may choose one of these effects as well. On a 12+ you get to choose one of these effects for free.

         
\startitemize[1,packed]
           
\item The spell’s effects are doubled

           
\item The spell’s targets are doubled

         
\stopitemize
         
\subsection{Eldritch Connection}  \index{Eldritch Connection} \index{Eldritch} \index{Connection}
         

Replaces: Eldritch Touch

         

When you touch a living creature, skin to skin, you can ask the GM what the last spell to affect that creature was and one other question about their past of your choice.

         
\subsection{Highly Logical}  \index{Highly Logical} \index{Highly} \index{Logical}
         

Replaces: Logical

         

When you use strict deduction to analyze your surroundings, you can Discern Realities with Int instead of Wis. On a 12+ you get to ask the GM any three questions, not limited by the list.

         
\subsection{Arcane Armor}  \index{Arcane Armor} \index{Arcane} \index{Armor}
         

Replaces: Arcane Ward

         

As long as you have at least one prepared spell, you have +4 armor.

         
\subsection{Protective Counter}  \index{Protective Counter} \index{Protective} \index{Counter}
         

Requires: Counterspell

         

When an ally within sight of you is affected by an arcane spell, you can counter it as if it effected you. If the spell affects multiple allies you must counter for each ally separately.

         
\subsection{Ethereal Tether}  \index{Ethereal Tether} \index{Ethereal} \index{Tether}
         

When you have time with a willing or helpless subject you can craft an ethereal tether with them. You perceive what they perceive and can Discern Realities about someone tethered to you or their surroundings no matter the distance. Someone willingly tethered to you can communicate with you over the tether as if you were in the room with them.

         
\subsection{Mystical Puppet Strings}  \index{Mystical Puppet Strings} \index{Mystical} \index{Puppet} \index{Strings}
         

When you use magic to control a person's actions they have no memory of what you had them do and bear you no ill will.

         
\subsection{Spell Augmentation}  \index{Spell Augmentation} \index{Spell} \index{Augmentation}
         

When you deal damage to a creature you can shunt a spell's energy into them—end one of your ongoing spells and add the spell's level to the damage dealt.

         
\subsection{Self-Powered}  \index{Self-Powered} \index{Self-powered}
         

When you have time, arcane materials, and a safe space, you can create your own place of power. Describe to the GM what kind of power it is and how you're binding it to this place, the GM will tell you one kind of creature that will have an interest in your workings.

       

                
\section{Wizard Spells}  \index{Wizard Spells} \index{Wizard} \index{Spells}
     

       
\subsection{Cantrips}  \index{Cantrips} \index{Cantrips}
       
\startSpellName
         \CMSpellName{Light} 	Cantrip
\stopSpellName
       

An item you touch glows with arcane light, about as bright as a torch. It gives off no heat or sound and requires no fuel, but it is otherwise like a mundane torch. You have complete control of the color of the flame. The spell lasts as long as it is in your presence.

       
\startSpellName
         \CMSpellName{Unseen Servant} 	Cantrip
\stopSpellName
       

You conjure a simple invisible construct that can do nothing but carry items. It has Load 2 and carries anything you hand to it. It cannot pick up items on its own and can only carry those you give to it. Items carried by an unseen servant appear to float in the air a few paces behind you. An unseen servant that takes damage or leaves your presence is immediately dispelled.

       
\startSpellName
         \CMSpellName{Prestidigitation} 	Cantrip
\stopSpellName
       

You perform minor tricks of true magic. If you touch an item as part of the casting you can make cosmetic changes to it: clean it, soil it, cool it, warm it, flavor it, or change its color. If you cast the spell without touching an item you can instead create minor illusions no bigger than yourself. Prestidigitation illusions are crude and clearly illusions; they won't fool anyone, but they might entertain them.

     

     

       
\subsection{1st Level Spells}  \index{1st Level Spells} \index{1st} \index{Level} \index{Spells}
       
\startSpellName
         \CMSpellName{Contact Spirits} 	Level 1	Summoning
\stopSpellName
       

Name the spirit you wish to contact (or leave it to the GM). You pull that creature through the planes, just close enough to speak to you. It is bound to answer any one question you ask to the best of its ability.

       
\startSpellName
         \CMSpellName{Detect Magic} 	Level 1	Divination
\stopSpellName
       

One of your senses is briefly attuned to magic. The GM will tell you what here is magical.

       
\startSpellName
         \CMSpellName{Telepathy} 	Level 1	Divination
\stopSpellName
       

You form a telepathic bond, allowing you to speak to the person you touch with this spell through your thoughts. You can only have one telepathic bond at a time.

       
\startSpellName
         \CMSpellName{Identify} 	Level 1	Divination
\stopSpellName
       

The history of the item you hold while casting this spell is made clear to you. The GM will tell you what it does, where it came from, and how it got here.

       
\startSpellName
         \CMSpellName{Invisibility} 	Level 1	Illusion
\stopSpellName
       

Touch an ally: nobody can see them. They're invisible! The spell persists until the target attacks or you dismiss the effect. While the spell is ongoing, you can't cast another spell.

       
\startSpellName
         \CMSpellName{Magic Missile} 	Level 1	Evocation
\stopSpellName
       

Projectiles of pure magic spring from your fingers. Deal 2d4 damage to one target.

       
\startSpellName
         \CMSpellName{Alarm} 	Level 1
\stopSpellName
       

Walk a wide circle. Until you prepare spells again your magic will alert you if a creature crosses that circle. Even if you are asleep, the spell will shake you from your slumber.

     

     

       
\subsection{3rd Level Spells}  \index{3rd Level Spells} \index{3rd} \index{Level} \index{Spells}
       
\startSpellName
         \CMSpellName{Dispel Magic} 	Level 3
\stopSpellName
       

Choose a spell or magic effect in your presence: this spell rips it apart. Lesser spells are ended, powerful magic is just reduced or dampened so long as you are nearby.

       
\startSpellName
         \CMSpellName{Visions Through Time} 	Level 3	Divination
\stopSpellName
       

Cast this spell and gaze into a reflective surface to see into the depths of time. The GM will reveal the details of a Grim Portent to you—a bleak event that will come to pass if not directly stopped. They'll tell you something useful about how you can interfere with the Grim Portent's dark outcomes. Rare is the Portent that claims "You'll live happily ever after." Sorry.

       
\startSpellName
         \CMSpellName{Fireball} 	Level 3	Evocation
\stopSpellName
       

You evoke a mighty ball of flame that envelops your target and everyone nearby, inflicting 2d6 damage which ignores armor.

       
\startSpellName
         \CMSpellName{Mimic} 	Level 3
\stopSpellName
       

You take the form of someone you touch while casting this spell. Your physical characteristics match theirs exactly but your behavior may not. This change persists until you take damage or choose to return to your own form. While this spell is ongoing, you lose access to all your wizard moves.

       
\startSpellName
         \CMSpellName{Mirror Image} 	Level 3	Illusion
\stopSpellName
       

You create an illusory image of yourself. The next attack against you effects the illusory image, not you. The image then dissipates.

       
\startSpellName
         \CMSpellName{Sleep} 	Level 3	Enchantment
\stopSpellName
       

1d4 enemies you can see of the GM's choice fall asleep. Only creatures capable of sleeping are effected. They awake as normal: loud noises, jolts, pain.

     

     

       
\subsection{5th Level Spells}  \index{5th Level Spells} \index{5th} \index{Level} \index{Spells}
       
\startSpellName
         \CMSpellName{Cage} 	Level 5	Evocation
\stopSpellName
       

The target is held in a cage of magical force. Nothing can get in or out of the cage. The cage remains until you cast another spell or dismiss it. While the spell is ongoing, the caged creature can hear your thoughts and you cannot leave sight of the cage.

       
\startSpellName
         \CMSpellName{Contact Other Plane} 	Level 5	Divination
\stopSpellName
       

You send a request to another plane. Specify what you'd like to contact by location, type of creature, name, or title. You open a two-way communication with that creature. Your communication can be cut off at any time by you or the creature you contacted.

       
\startSpellName
         \CMSpellName{Polymorph} 	Level 5	Enchantment
\stopSpellName
       

Your touch reshapes a creature entirely, they stay in the form you craft until you Cast a Spell. Describe to the GM the new shape you craft, including any stat changes, significant adaptations, or major weaknesses. The GM will then tell you one or more of these:

       
\startitemize[1,packed]
         
\item The form will be unstable and temporary

         
\item The creature's mind will be altered as well

         
\item The form has an unintended benefit or weakness

       
\stopitemize
       
\startSpellName
         \CMSpellName{Summon Monster} 	Level 5	Summoning
\stopSpellName
       

A monster appears and aids you as best it can. Treat it as your character, but with access to only the basic moves. It has +1 modifier for all stats and 1 HP. Choose the type of monster by choosing 1d6 statements from the list below. The GM will tell you the type of monster you get based on your choices:

       
\startitemize[1,packed]
         
\item The monster has +2 instead of +1 to one stat

         
\item The monster is not reckless

         
\item The monster does 1d8 damage

         
\item The monster's bond to your plane is strong, +3 HP for each level you have

         
\item The monster has some useful adaptation

       
\stopitemize
       

The creature remains on this plane until it dies or you dismiss it. While the spell is ongoing, you take -1 to Cast a Spell.

     

     

       
\subsection{7th Level Spells}  \index{7th Level Spells} \index{7th} \index{Level} \index{Spells}
       
\startSpellName
         \CMSpellName{Dominate} 	Level 7	Enchantment
\stopSpellName
       

Your touch pushes your mind into someone else's. You gain 1d4 hold. Spend one hold to make the target take one of these actions:

       
\startitemize[1,packed]
         
\item Speak a few words of your choice

         
\item Give you something they hold

         
\item Make a concerted attack on a target of your choice

         
\item Truthfully answer one question

       
\stopitemize
       

If you run out of hold the spell ends. If the target takes damage you lose 1 hold. While the spell is ongoing you cannot Cast a Spell.

       
\startSpellName
         \CMSpellName{True Seeing} 	Level 7	Divination
\stopSpellName
       

You see all things as they truly are. This effect persists until you tell a lie or dismiss the spell. While this spell is ongoing, you take -1 to Cast a Spell.

       
\startSpellName
         \CMSpellName{Shadow Walk} 	Level 7	Illusion
\stopSpellName
       

The shadows you target with this spell become a portal for you and your allies. Name a location, describing it with a number of words up to your level. Stepping through the portal deposits you and any allies present when you cast the spell at the location you described. The portal may only be used once by each ally.

       
\startSpellName
         \CMSpellName{Contingency} 	Level 7	Evocation
\stopSpellName
       

Choose a 5th level or lower spell you know. Describe a trigger condition using a number of words equal to your level. The chosen spell is held until you choose to unleash it or the trigger condition is met, whichever happens first. You don't have to roll for the held spell, it just takes effect. While you have a contingent spell, you can't gain another one.

       
\startSpellName
         \CMSpellName{Cloudkill} 	Level 7	Summoning
\stopSpellName
       

A cloud of fog drifts into this realm from beyond the Black Gates of Death, filling the immediate area. Whenever a creature in the area takes damage it takes an extra 1d6 damage which ignores armor. This spell persists so long as you can see the affected area, or until you dismiss it.

     

     

       
\subsection{9th Level Spells}  \index{9th Level Spells} \index{9th} \index{Level} \index{Spells}
       
\startSpellName
         \CMSpellName{Antipathy} 	Level 9	Enchantment
\stopSpellName
       

Choose a target and describe a type of creature or an alignment. Creatures of the specified type or alignment cannot come within sight of the target. If a creature of the specified type does find itself within site of the target, it immediately flees. This effect continues until you leave the target's presence or you dismiss the spell. While the spell is ongoing, you take -1 to Cast a Spell.

       
\startSpellName
         \CMSpellName{Alert} 	Level 9	Divination
\stopSpellName
       

Describe an event. The GM will tell you when that event occurs, no matter where you are or how far away the event is. If you choose, you can view the location of the event as though you were there in person. You can only have one Alert active at a time.

       
\startSpellName
         \CMSpellName{Soul Gem} 	Level 9
\stopSpellName
       

You trap the soul of a dying creature within a gem. The trapped creature is aware of its imprisonment but can still be manipulated through spells, Parley, and other effects. All moves against the trapped creature are at +1. You can free the soul at any time but it can never be recaptured once freed.

       
\startSpellName
         \CMSpellName{Shelter} 	Level 9	Evocation
\stopSpellName
       

You create a structure out of pure magical power. It can be as large as a castle or as small as a hut, but is impervious to all non-magical damage. The structure endures until you leave it or you end the spell.

       
\startSpellName
         \CMSpellName{Perfect Summons} 	Level 9	Summoning
\stopSpellName
       

You teleport a creature to your presence. Name a creature or give a short description of a type of creature. If you named a creature, that creature appears before you. If you described a type of creature, a creature of that type appears before you.

     

           
