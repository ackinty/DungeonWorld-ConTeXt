\chapter{The Wizard}
 \index{The Wizard} \index{Wizard}
 





Dungeon World has rules. Not the laws of men or the rule of some petty tyrant. Bigger, better rules. You drop something—it falls. You can’t make something out of nothing. The dead stay dead, right?

 

Oh, the things we tell ourselves to feel better about the long, dark nights.

 

You’ve spent so very long poring over those tomes of yours. The experiments that nearly drove you mad and all the botched summonings that endangered your very soul. For what? For power. What else is there? Not just the power of King or Country but the power to boil a man's blood in his veins. To call on the thunder of the sky and the churn of the roiling earth. To shrug off the rules the world holds so dear.

 

Let them cast their sidelong glances. Let them call you “warlock” or “diabolist.” Who among them can hurl fireballs from their eyes?

 

Yeah. We didn’t think so.



 
\section{Names}  \index{Names} \index{Names}
 



{\em Elf} : Galadiir, Fenfaril, Lilliastre, Phirosalle, Enkirash, Halwyr

  

{\em Human} : Avon, Morgan, Rath, Ysolde, Ovid, Vitus, Aldara, Xeno, Uri



 
\section{Look}  \index{Look}
 



Choose one for each:

 

Haunted Eyes, Sharp Eyes, or Crazy Eyes

 

Styled Hair, Wild Hair, or Pointed Hat

 

Worn Robes, Stylish Robes, or Strange Robes

 

Pudgy Body, Creepy Body, or Thin Body



 
\section{Stats}  \index{Stats} \index{Stats}
 



Assign these scores to your stats:

 

17 (+2), 15 (+1), 13 (+1), 11 (+0), 9 (+0), 8 (-1)

 

You start with 4+Constitution HP.



 

Your base damage is d4.

 
\section{Starting Moves}  \index{Starting Moves} \index{Moves}
 


\startInstructionsAfterHeader
Choose a racial move:
\stopInstructionsAfterHeader
 


\subsection{Elf}  \index{Elf} \index{Elf}
 

Magic is as natural as breath to you. Detect Magic is a rote for you.

 
\subsection{Human}  \index{Human} \index{Human}
 

Choose one cleric spell, you can cast it as if it was a wizard spell.



 


\startInstructions
You start with these moves:
\stopInstructions
 
\subsection{Spellbook}  \index{Spellbook} \index{Spellbook}
 

You have mastered several spells and inscribed them in your spellbook. You start out with three first level spells in your spellbook as well as the cantrips. Whenever you gain a level, you add a new spell of your level or lower to your spellbook. You spellbook is 1 weight.

 
\subsection{Prepare Spells}  \index{Prepare Spells} \index{Prepare} \index{Spells}
 

When you spend uninterrupted time (an hour or so) in quiet contemplation of your spellbook, you lose any spells you already have prepared and prepare new spells of your choice from your spellbook whose total levels don't exceed your own+1. You also prepare your cantrips; they don't count against your limit.

 
\subsection{Cast a Spell (Int)}  \index{Cast a Spell (Int)} \index{Cast} \index{Spell} \index{(int)}
 

When you release a spell you've prepared, roll+Int. On a 10+, the spell is successfully cast and you do not forget the spell—you may cast it again later. On a 7-9, the spell is cast, but choose one:

 
\startitemize[1,packed]

\item You draw unwelcome attention or put yourself in a spot (the GM will describe it)

 
\item The spell disturbs the fabric of reality as it is cast—take -1 ongoing to Cast a Spell until you Prepare Spells again.

 
\item After it is cast, the spell is forgotten. You cannot cast the spell again until you Prepare Spells.


\stopitemize
 
\subsection{Spell Defense}  \index{Spell Defense} \index{Spell} \index{Defense}
 

When you craft an ongoing spell into a makeshift shield of arcane energy to deflect an attack, the spell is ended and you subtract the spell's level from the damage done to you.

 
\subsection{Ritual}  \index{Ritual} \index{Ritual}
 

When you draw on a place of power to create a magical effect, tell the GM what you're trying to achieve. The GM will tell you "yes, you can do that, but..." and then 1 to 4 of the following:

 
\startitemize[1,packed]

\item It's going to take days/weeks/months

 
\item First you must \_\_\_\_

 
\item You'll need help from \_\_\_\_

 
\item It will require a lot of money

 
\item The best you can do is a lesser version, unreliable and limited

 
\item You and your allies will risk danger from \_\_\_\_

 
\item You'll have to disenchant \_\_\_\_ to do it


\stopitemize




 
\section{Alignment}  \index{Alignment} \index{Alignment}
 
\startInstructionsAfterHeader
Choose an alignment:
\stopInstructionsAfterHeader
 


\subsection{Good}  \index{Good} \index{Good}
 

Use magic to directly aid another

 
\subsection{Neutral}  \index{Neutral} \index{Neutral}
 

Discover something about a magical mystery

 
\subsection{Evil}  \index{Evil} \index{Evil}
 

Use magic to cause terror and fear



 
\section{Gear}  \index{Gear} \index{Gear}
 



Your Load is 5+Str. You start with your spellbook (1 weight) and dungeon rations (1 weight, 5 uses). Choose your defenses:

 
\startitemize[1,packed]

\item Leather armor (1 armor, 1 weight)

 
\item Bag of books (5 uses, 2 weight) and 3 healing potions


\stopitemize
 

Choose your weapon:

 
\startitemize[1,packed]

\item Dagger (Hand, 1 weight)

 
\item staff (Close, two-handed, 1 weight)


\stopitemize
 

Choose one:

 
\startitemize[1,packed]

\item healing potion

 
\item three antitoxin


\stopitemize


 
\section{Bonds}  \index{Bonds} \index{Bonds}
 



Fill in the name of one of your companions in at least one:

 

\thinrules[2] will play an important role in the events to come. I have foreseen it!

 

\thinrules[2] is keeping an important secret from me.

 

\thinrules[2] is woefully misinformed about the world; I will teach them all that I can.



 
\section{Advanced Moves}  \index{Advanced Moves} \index{Advanced} \index{Moves}
 


\startInstructionsAfterHeader
When you gain a level from 2-5, choose from these moves. You also add a new spell to your spellbook at each level.
\stopInstructionsAfterHeader
 
\subsection{Prodigy}  \index{Prodigy} \index{Prodigy}
 

Choose a spell. You prepare that spell as if it were one level lower.

 
\subsection{Empowered Magic}  \index{Empowered Magic} \index{Empowered} \index{Magic}
 

When you Cast a Spell, on a 10+ you have the option of choosing from the 7-9 list. If you do, you may choose one of these as well:

 
\startitemize[1,packed]

\item The spell's effects are maximized

 
\item The spell's targets are doubled


\stopitemize
 
\subsection{Fount of Knowledge}  \index{Fount of Knowledge} \index{Fount} \index{Knowledge}
 

When you Spout Lore about something no one else has any clue about, take +1.

 
\subsection{Know-It-All}  \index{Know-It-All} \index{Know-it-all}
 

When another player's character comes to you for advice and you tell them what you think is best, they get +1 forward when following your advice and you mark experience if they do.

 
\subsection{Expanded Spellbook}  \index{Expanded Spellbook} \index{Expanded} \index{Spellbook}
 

Add a new spell from any class to your spellbook.

 
\subsection{Enchanter}  \index{Enchanter} \index{Enchanter}
 

When you have time and safety with a magic item you may ask the GM what it does, the GM will answer you.

 
\subsection{Logical}  \index{Logical} \index{Logical}
 

When you use strict deduction to analyze your surroundings, you can Discern Realities with Int instead of Wis.

 
\subsection{Arcane Ward}  \index{Arcane Ward} \index{Arcane} \index{Ward}
 

As long as you have at least one prepared spell, you have +2 armor.

 
\subsection{Counterspell}  \index{Counterspell} \index{Counterspell}
 

When you are affected by arcane magic you may attempt to counter the spell. Stake one of your prepared spells of equal or higher level on the defense and roll+Int. On a 10+, the spell is countered and has no effect on you. On a 7-9, the spell is countered and you forget the spell you staked. If the spell has other targets they are effected as usual.

 
\subsection{Quick Study}  \index{Quick Study} \index{Quick} \index{Study}
 

When you see the effects of an arcane spell, ask the GM the name of the spell and its effects. You take +1 when acting on the answers.

 
\startInstructions
When you gain a level from 6-10, choose from these moves or the level 2-5 moves.
\stopInstructions
 
\subsection{Master}  \index{Master} \index{Master}
 

Requires: Prodigy

 

Choose a spell. You prepare that spell as if it were one level lower.

 
\subsection{Greater Empowered Magic}  \index{Greater Empowered Magic} \index{Greater} \index{Empowered} \index{Magic}
 

Replaces: Empowered Magic

 

When you Cast a Spell, on a 10-11 you have the option of choosing from the 7-9 list. If you do, you may choose one of these effects as well. On a 12+ you get to choose one of these effects for free.

 
\startitemize[1,packed]

\item The spell’s effects are doubled

 
\item The spell’s targets are doubled


\stopitemize
 
\subsection{Enchanter's Soul}  \index{Enchanter's Soul} \index{Enchanter's} \index{Soul}
 

Requires: Enchanter

 

When you have time and safety with a magic item in a place of power you can empower that item so that the next time you use it its effects are amplified, the GM will tell you exactly how.

 
\subsection{Highly Logical}  \index{Highly Logical} \index{Highly} \index{Logical}
 

Replaces: Logical

 

When you use strict deduction to analyze your surroundings, you can Discern Realities with Int instead of Wis. On a 12+ you get to ask the GM any three questions, not limited by the list.

 
\subsection{Arcane Armor}  \index{Arcane Armor} \index{Arcane} \index{Armor}
 

Replaces: Arcane Ward

 

As long as you have at least one prepared spell, you have +4 armor.

 
\subsection{Protective Counter}  \index{Protective Counter} \index{Protective} \index{Counter}
 

Requires: Counterspell

 

When an ally within sight of you is affected by an arcane spell, you can counter it as if it effected you. If the spell affects multiple allies you must counter for each ally separately.

 
\subsection{Ethereal Tether}  \index{Ethereal Tether} \index{Ethereal} \index{Tether}
 

When you have time with a willing or helpless subject you can craft an ethereal tether with them. You perceive what they perceive and can Discern Realities about someone tethered to you or their surroundings no matter the distance. Someone willingly tethered to you can communicate with you over the tether as if you were in the room with them.

 
\subsection{Mystical Puppet Strings}  \index{Mystical Puppet Strings} \index{Mystical} \index{Puppet} \index{Strings}
 

When you use magic to control a person's actions they have no memory of what you had them do and bear you no ill will.

 
\subsection{Spell Augmentation}  \index{Spell Augmentation} \index{Spell} \index{Augmentation}
 

When you deal damage to a creature you can shunt a spell's energy into them—end one of your ongoing spells and add the spell's level to the damage dealt.

 
\subsection{Self-Powered}  \index{Self-Powered} \index{Self-powered}
 

When you have time, arcane materials, and a safe space, you can create your own place of power. Describe to the GM what kind of power it is and how you're binding it to this place, the GM will tell you one kind of creature that will have an interest in your workings.










