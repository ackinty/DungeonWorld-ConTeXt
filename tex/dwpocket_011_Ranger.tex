\chapter{Ranger}
 \index{Ranger} \index{Ranger}
            

         

These city-born folk you travel with. Have they heard the call of the wolf? Felt the winds howl in the bleak deserts of the East? Have they hunted their prey with the bow and the knife like you? Hell no. That’s why they need you.

         

Guide. Hunter. Creature of the wilds. You are these things and more. Your time in the wilderness may have been solitary until now, the call of some greater thing – call it fate if you like, has cast your lot with these folk. Brave, they may be. Powerful and strong, too. You know the secrets of the spaces-between, though.

         

Without you, they’d be lost. Blaze a trail through the blood and dark, strider.

       

       
\section{Names}  \index{Names} \index{Names}
       

         

           {\em Elf} : Throndir, Elrosine, Aranwe, Celion, Dambrath, Lanethe

         

           {\em Human} : Jonah, Halek, Brandon, Emory, Shrike, Nora, Diana

       

       
\section{Look}  \index{Look}
       

         

Choose one for each:

         

Wild Eyes, Sharp Eyes, Animal Eyes

         

Hooded Head, Wild Hair, Bald

         

Cape, Camouflage, Traveling Clothes

         

Lithe Body, Wild Body, Sharp Body

       

       
\section{Stats}  \index{Stats} \index{Stats}
       

         

Assign these scores to your stats:

         

17 (+2), 15 (+1), 13 (+1), 11 (+0), 9 (+0), 8 (-1)

         

You start with 8+Constitution HP.

       

       

Your base damage is d8.

       
\section{Starting Moves}  \index{Starting Moves} \index{Moves}
       

         
\startInstructionsAfterHeader
Choose a racial move:
\stopInstructionsAfterHeader
         

           
\subsection{Elf}  \index{Elf} \index{Elf}
           

When you undertake a Perilous Journey through wilderness whatever role you take you succeed as if you rolled a 10+.

           
\subsection{Human}  \index{Human} \index{Human}
           

When you Make Camp in a dungeon or city, you don't need to consume a ration.

         

         

           
\startInstructions
You start with these moves:
\stopInstructions
           
\subsection{Hunt and Track (Wis)}  \index{Hunt and Track (Wis)} \index{Hunt} \index{Track} \index{(wis)}
           

When you follow a trail of clues left behind by passing creatures, roll+Wis. On a hit, you follow the creature's trail until there's a significant change in its direction or mode of travel. On a 10+, you also choose 1:

           
\startitemize[1,packed]
             
\item Gain a useful bit of information about your quarry, the GM will tell you what

             
\item Determine what caused the trail to end

           
\stopitemize
           
\subsection{Called Shot}  \index{Called Shot} \index{Called} \index{Shot}
           

When you attack a defenseless or surprised enemy at range, you can choose to deal your damage or name your target and roll+Dex.

           
\startitemize[1,packed]
             
\item Head – 10+: As 7–9, plus your damage; 7-9: They do nothing but stand and drool for a few moments.

             
\item Arms – 10+: As 7-9, plus your damage; 7-9: They drop anything they're holding.

             
\item Legs – 10+: As 7-9, plus your damage; 7-9: They're hobbled and slow moving.

           
\stopitemize
           
\subsection{Animal Companion}  \index{Animal Companion} \index{Animal} \index{Companion}
           

You have a supernatural connection with a loyal animal. You can't talk to it per se but it always acts as you wish it to. Name your animal companion and choose a species:

           
\startExample
Wolf, cougar, bear, eagle, dog, hawk, cat, owl, pigeon, rat, mule
\stopExample
           

Choose a base:

           
\startitemize[1,packed]
             
\item Ferocity 2d4, Cunning +1, 1 Armor, Instinct +1

             
\item Ferocity 2d4, Cunning +2, 0 Armor, Instinct +1

             
\item Ferocity 1d4, Cunning +2, 1 Armor, Instinct +1

             
\item Ferocity 2d4, Cunning +1, 2 Armor, Instinct +2

           
\stopitemize
           

Choose as many strengths as its ferocity:

           
\startExample
Fast, burly, huge, calm, adaptable, quick reflexes, tireless, camouflage, ferocious, intimidating, keen senses, stealthy
\stopExample
           

Your animal companion is trained to fight humanoids. Choose as many additional trainings as its cunning:

           
\startExample
Hunt, search, scout, guard, fight monsters, perform, labor, travel
\stopExample
           

Choose as many weaknesses as its instinct:

           
\startExample
Flighty, savage, slow, broken, frightening, forgetful, stubborn, lame
\stopExample
           
\subsection{Command}  \index{Command} \index{Command}
           

When you work with your animal companion on something it's trained in…

           
\startitemize[1,packed]
             
\item …and you attack the same target, add its ferocity to your damage

             
\item …and you track, add its cunning to your roll

             
\item …and you take damage, add its armor to your armor

             
\item …and you discern realities, add its cunning to your roll

             
\item …and you parley, add its cunning to your roll

             
\item …and someone interferes with you, add its instinct to your roll

           
\stopitemize
         

       

       
\section{Alignment}  \index{Alignment} \index{Alignment}
       
\startInstructionsAfterHeader
Choose an alignment:
\stopInstructionsAfterHeader
       

         
\subsection{Chaotic}  \index{Chaotic} \index{Chaotic}
         

Free someone from literal or figurative bonds

         
\subsection{Good}  \index{Good} \index{Good}
         

Endanger yourself to combat an unnatural threat

         
\subsection{Neutral}  \index{Neutral} \index{Neutral}
         

Help an animal or spirit of the wild

       

       
\section{Gear}  \index{Gear} \index{Gear}
       

         

Your Load is 6+Str. You start with dungeon rations (1 weight, 5 uses), leather armor (1 armor, 1 weight), and a bundle or arrows (3 ammo, 2 weight). Choose your armament:

         
\startitemize[1,packed]
           
\item Hunter's bow (Near, Far, 1 weight) and short sword (Close, 1 weight)

           
\item Hunter's bow (Near, Far, 1 weight) and spear (Reach, 1 weight)

         
\stopitemize
         

Choose one:

         
\startitemize[1,packed]
           
\item Adventuring gear (1 weight) and dungeon rations (1 weight)

           
\item adventuring gear (1 weight) and bundle of arrows (3 ammo, 2 weight)

         
\stopitemize
       

       
\section{Bonds}  \index{Bonds} \index{Bonds}
       

         

Fill in the name of one of your companions in at least one:

         

I have guided \thinrules[n=2] before and they owe me for it.

         

\thinrules[n=2] is a friend of nature, so I will be their friend as well.

         

\thinrules[n=2] has no respect for nature, so I have no respect for them.

         

\thinrules[n=2] does not understand life in the wild, so I will teach them.

       

       
\section{Advanced Moves}  \index{Advanced Moves} \index{Advanced} \index{Moves}
       

         
\startInstructionsAfterHeader
Take this move only if it is your first advancement
\stopInstructionsAfterHeader
         
\subsection{Half-Elven}  \index{Half-Elven} \index{Half-elven}
         

Somewhere in your lineage lies mixed blood and it begins to show its presence. You gain the Elf starting move if you took the Human one at character creation or vice versa.

         
\startInstructions
When you gain a level from 2-5, choose from these moves.
\stopInstructions
         
\subsection{Wild Empathy}  \index{Wild Empathy} \index{Wild} \index{Empathy}
         

You can speak with and understand animals.

         
\subsection{Familiar Prey}  \index{Familiar Prey} \index{Familiar} \index{Prey}
         

When you Spout Lore about a monster you use Wis instead of Int.

         
\subsection{Dual Wield}  \index{Dual Wield} \index{Dual} \index{Wield}
         

When strike an enemy with two weapons at once, add an extra 1d4 damage for your off-hand strike.

         
\subsection{Camouflage}  \index{Camouflage} \index{Camouflage}
         

When you're still in natural surroundings, enemies never spot you until you make a movement.

         
\subsection{Man's Best Friend'}  \index{Man's Best Friend'} \index{Man's} \index{Friend'}
         

When you take damage and you allow your animal companion to take the blow the damage is negated and your animal companion's Ferocity becomes 0. If its Ferocity was already 0 you can't use this ability. When you have a few hours of rest with your animal companion its Ferocity returns to normal.

         
\subsection{Quarry}  \index{Quarry} \index{Quarry}
         

You take +1 forward against someone whose trail you follow.

         
\subsection{Well Trained}  \index{Well Trained} \index{Trained}
         

Choose another training for your animal companion.

         
\subsection{God Amidst the Wastes}  \index{God Amidst the Wastes} \index{God} \index{Wastes}
         

You gain the Commune and Cast a Spell Cleric move. Your level for the purposes of that move is 1 + the levels you've gained since you took God Amidst the Wastes.

         
\subsection{Follow Me}  \index{Follow Me} \index{Follow}
         

When you Undertake a Perilous journey you can take two roles. You make a roll for each.

         
\subsection{A Safe Place}  \index{A Safe Place} \index{Safe} \index{Place}
         

When you set the watch for the night everyone takes +1 to Take Watch.

         
\startInstructions
When you gain a level from 6-10, choose from these moves or the level 2-5 moves.
\stopInstructions
         
\subsection{Wild Speech}  \index{Wild Speech} \index{Wild} \index{Speech}
         

Replaces: Wild Empathy

         

You can speak with and understand any non-magical non-planar creature.

         
\subsection{Hunter's Prey}  \index{Hunter's Prey} \index{Hunter's} \index{Prey}
         

Replaces: Familiar Prey

         

When you Spout Lore about a monster you use Wis instead of Int. On a 12+ you get to ask the GM any one question about the subject.

         
\subsection{Dual Wield}  \index{Dual Wield} \index{Dual} \index{Wield}
         

Replaces: Dual Strike

         

When strike an enemy with two weapons at once, add an extra 1d8 damage for your off-hand strike.

         
\subsection{Unescapable Quarry}  \index{Unescapable Quarry} \index{Unescapable} \index{Quarry}
         

Replaces: Quarry

         

You take +1 forward against someone whose trail you follow. You cannot be surprised, ambushed, or otherwise at a disadvantage against someone whose trail you follow.

         
\subsection{Strider}  \index{Strider} \index{Strider}
         

Replaces: Follow Me

         

When you Undertake a Perilous journey you can take two roles. Roll twice and use the better result for both roles.

         
\subsection{A Safer Place}  \index{A Safer Place} \index{Safer} \index{Place}
         

Replaces: A Safe Place

         

When you set the watch for the night everyone takes +1 to Take Watch. After a night in camp when you set the watch everyone takes +1 forward.

         
\subsection{Observant}  \index{Observant} \index{Observant}
         

When you Hunt and Track, on a hit you may also ask one question about the creature you are tracking from the Discern Realities list for free.

         
\subsection{Special Trick}  \index{Special Trick} \index{Special} \index{Trick}
         

Choose a move from another class. So long as you are working with your animal companion you have access to that move.

         
\subsection{Unnatural Ally}  \index{Unnatural Ally} \index{Unnatural} \index{Ally}
         

Your animal companion is a monster, not an animal. Describe it. Give it +1d4 Ferocity and +1 Instinct, plus a new training.

       

                
