\chapter{Guild Members}
 \index{Guild Members} \index{Guild} \index{Members}
 





 



Since you've read this book it's likely at some point you'll be teaching the game to others, either experienced roleplayers or those new to the hobby. Throughout the design process we've had many chances to play with lots of different gamers from different backgrounds and there are a few things we've found work well for teaching the game.

 
\section{Pitch It}  \index{Pitch It} \index{Pitch}
 

Before you play you'll likely be explaining the game to your new players (don't just spring it on them, that's not cool). We call that the pitch: it's explaining why you want to play Dungeon World and why you think they'll like it.

 

First and most importantly: put it in your own words. We can't give you a script because the best way to get people excited about the game is to share your honest excitement. There are, however, some things you might want to touch on.

 

With first-time roleplayers it's best to focus on what roleplaying means in Dungeon World. Tell them what they'll be doing (portraying a character) and what you'll be doing (portraying the world around them). Mention the general conceit (adventurers and adventure). It's usually a good idea to mention the role of the rules too, how they're there to drive the action forward in interesting ways.

 

With folks who've played RPGs before, especially those who've played other fantasy adventure games, you can focus more on what makes Dungeon World different from other similar games. Ease-of-play, the way the rules just step in at the right times, and the fast pace are all things that experienced roleplayers often appreciate.

 

No matter the audience, don't just pitch Dungeon World, pitch the game you're going to run. If this is going to be a trip into the city sewers, tell them that right up front. If there's an evil cult to be stopped that should be part of your description. The interaction between you, the players, and the rules will create all kinds of interesting secrets later on, your pitch should honestly portray the game you intend to run.

 
\section{Present the Classes}  \index{Present the Classes} \index{Present} \index{Classes}
 

Once everybody's on board for a game of Dungeon World and you've sat down to play start by presenting the character sheets. Give a short description of each, making sure to mention what each does and their place in the world. You can also read out the descriptions for each class, those all include something about both what the class does and how that fits into the big picture.

 

If anybody has questions about the rules, answer them, but for now focus on describing what the classes do in plain terms. If someone asks about the fighter it's more useful to tell them that the fighter has a signature weapon that's one of a kind then to go into detail about how the Signature Weapon move works.

 
\section{Create Characters}  \index{Create Characters} \index{Create} \index{Characters}
 

Go through the character creation rules step-by-step. The process of creating a character is also a great introduction to the basic concepts: the players will encounter stats, moves, HP, and damage all in an order that makes sense. Don't bother trying to front-load the rules explanations. There aren't really any wrong choices.

 

Each player will encounter the rules that are important to their class. The Fighter, for example, will see moves about weapon ranges and piercing and ask about them, explain them as needed. If the Fighter player doesn't ask you what piercing is, don't worry about it. They're happy to choose based on the fiction, which is all the stats and tags reflect anyway.

 

If your players are particularly worried about making their characters 'right' just give them the option of changing them later. Trying to cover every rule and give them all the context now will just slow the game down. In particular, don't go over the basic moves in detail yet. Leave them out so that the players can read them and ask questions, but don't waste time by explaining each. They'll come up as needed.

 

As the players introduce their characters and start setting bonds move from answering questions to asking them. Ask about why they chose what they did and what that means for their character. Ask about details established by their bonds. Let their choices establish the world around them. Take special note of anything that you think you might be able to make moves with (like an estranged teacher or a simmering war).

 
\subsection{Start Play}  \index{Start Play} \index{Start} \index{Play}
 

Start play by concretely describing the world around them. Keep it brief and evocative, use plenty of details, and end with something that demands action. Then ask them what they do.

 

Ending with something that demands action is important. Don't presume that new players will already know what they want to do. Giving them something to react to right away means you get straight to playing.

 

Especially for new players make sure that the action they're thrown into is something they have the tools to deal with. A fight is a good choice, as is a tense negotiation (which can easily become a fight). Keep it simple and let the complexity build.

 

Even in a fight keep to simple monsters: things that bleed, don't have too much armor, and don't have piercing. Give them a chance to get used to their armor and dealing damage before you start using the exceptions to those rules, like piercing and ignoring armor. Of course if the fiction dictates ignoring armor or piercing or a certain monster, use it, but don't lead with those.

 

For new players make liberal use of your Show Signs of Doom move. New players, or those used to a different type of fantasy adventure, may have different assumptions about what's lethal and when they're in danger, so make sure to show them danger clearly. Once they've started to pick up on what's dangerous you can give them a little less warning.

 

If you're GMing for the first time focus on a few moves: Show Signs of Doom, Deal Damage, Put Someone in a Spot. Only look at your moves sheet if you're pretty sure none of those three apply. Eventually you'll build up familiarity with the whole range of GM moves and using them will seem like second nature.

 
\subsection{Continuing Play}  \index{Continuing Play} \index{Play}
 

After an hour or two of play the players will likely have everything down. As a first time GM you may take a bit longer to pick up all your moves, maybe a session or two. Just roll with it.

 

If you find yourself struggling in the first session consider it a pilot, like the first episode of a TV show. Feel free to start over or retroactively change things. If a player decides that the Thief just isn't what they thought it was let them switch classes (either remaking the same character or introducing someone new). If your first adventure wasn't working too well scrap it and start something new.

 

While Dungeon World works great for one-shots the longer cycles of levels and bonds don't really kick in for a bit. If your first one or two sessions go well consider scheduling out enough time for 5–10 more. Knowing that you're planning to play that much longer gives you some space to plan out your fronts and resolve them.

 





 



There isn’t always time for prep. People aren’t entirely committed to a game—you just want to test it out or you’ve got a four-hour slot at a convention that you want to fill and you’ve never met the players before in your life. Maybe prep isn’t something you care about or you think it’s more fun to just take a map and run with it. Even better, maybe you’ve got a favorite old-school adventure module and you’d love to run through using the Dungeon World rules. In this appendix, we’ll cover how to convert and adapt material from other games into Dungeon World and give you the same flexibility to run your favorite adventures using the rules in this book.

 
\section{Overview}  \index{Overview} \index{Overview}
 

The first step in preparing an adventure for use with Dungeon World is reading through that adventure, and through the Dungeon World rules. For this book, you’ll want to be familiar with all the basic rules, as well as familiar with the section on Fronts and on the GM principles. The former will be guiding you in adapting the framework of the adventure and the latter will help keep your mind going in the right direction—so that gameplay stays true to the style and rules set out in this book. You’ll want to read through the module next, paying close attention to the four following topics as you go.

 
\startitemize[1,packed]

\item Maps

 
\item Monsters

 
\item Magic Items

 
\item NPCs and Organizations


\stopitemize
 

Flip through the adventure, make some notes as you go, but don’t feel you need to memorize the whole thing. Areas that focus particularly on statistics are likely to end up ignored, and you’ll want to leave blanks in the adventure for you and the players to discover as you go.

 

When you’ve finished, you’ll have a broad understanding about what the adventure is about—the power groups at play in it, the special or cool monsters the adventure contains, the threats and dangers that its cast present to the world and the kinds of things the PCs might be interested in. Set aside the adventure for now, and refer to the Fronts section of Dungeon World. This is where the majority of your work is going to take place.

 
\section{Fronts}  \index{Fronts} \index{Fronts}
 

The core of any standard adventure, scenario or game session in Dungeon World flows outward from the Fronts to the players; the Fronts have their Impending Dooms, the players react, and in the space between, you play the game to find out what happens. The same is true when presenting a converted adventure. Reading through the module, you’ll have noticed things—NPCs, places of interest, special monsters and organizations that might have an impact on the world or some agenda to carry out. Depending on the size of the adventure, there may be just one or a few of these. Take a look through the list of Front types and create one for each group.

 
\startExample
I’m going to convert an old adventure I love; I’ve run it a dozen times in a bunch of different systems and I think it’d be a blast to run my Dungeon World group through. I’ve given myself a quick read through to remind myself what the adventure is all about. In this case, there’s a town being menaced in secret by a wicked cult who worships a squamous reptile god. Sounds like fun! The adventure has a secret dungeon, a corrupt religious order, a bunch of smelly troglodytes and some very helpless adventurers. I’ve decided that the Fronts in this adventure are The Cultists and The Troglodyte Clan
\stopExample
 
\startExample
Now, I could make the sorcerous Naga that lives in the caverns her own Front, if I wanted to, or I could add in a Campaign Front for the Reptile God itself, but I think I’ll only be running this game a few sessions, so I’m going to stay focused. The two Fronts I have work together in some ways, but are unique and operate independently, so I’ve separated them.
\stopExample
 

Create these Fronts like you would normally, choosing dangers, impending dooms, and grim portents. Ask one or two stakes questions but be sure to leave yourself lots of room—that’s where you can really tie in the characters. Normally, you’d be pulling these things straight out of the inspiration of your brain, but in this case, you’ve got the module to guide you. Think about the Fronts as themes, and the Dangers as elements from the pages of your module. Look at the kinds of things your Fronts are said to be doing in the adventure and how that might go if the PCs were never there to stop it. What’s the worst that could happen if the Fronts were able to run rampant? This kind of reading-between-the-lines will give you ammunition for making your hard moves as you play through the adventure. This step is where you’ll turn those stat-block NPCs into either full-fledged dangers themselves, or members of the Front’s cast.

 

If there are any traps, curses or general effects in the adventure you’d like to write custom moves for, do it now. A lot of old adventures will have elements that call for a “saving throw” to avoid some noisome effect—these can often simply be a cause for a Defy Danger roll, or can have whole, separate custom moves if necessary. The key here is to capture the intent of the adventure—the spirit of the thing—rather than translate some mechanical element perfectly.

 

When you’re done, you’ll have a set of Fronts that cover the major threats and dangers the characters will face.

 
\section{Monsters}  \index{Monsters} \index{Monsters}
 

Most published adventures contained one or two unique monsters not seen anywhere else—custom creatures and denizens of the deeps that could threaten players in some way they hadn’t encountered before. Take a look through the adventure and make sure you’ve caught them all. Many monsters will already have statistics noted in Dungeon World and you can, if you’re happy with them, just make a note of what page they’re on in your Fronts and move on from there. If you want to further customize the monsters, or need to create your own, use the rules to do so. In this step, try to avoid thinking about “balancing” the monsters or concerning yourself too much with how many HP a monster has or whether its armor rating matches what you expect. Think more about how the monster is meant to participate in the adventure. Is it there to scare the pcs? Is it there to bar their way or pose a riddle? What is its purpose in the greater ecology of the dungeon or adventure at large? Translating the spirit of the thing will always give you better, more engaging results. If the monster has a cool power or neat trick you want to write a custom move for, do so! Custom moves are what make Dungeon World feel unique from group to group, so take advantage of them where you can.

 
\startExample
In my adventure, the monsters run the gamut. I’ve got a scary naga with some mind-controlling powers, an evil priest with divine snake-god magic, a bunch of ruffian cultists, a dragon turtle and a few miscellaneous lizards, crocodiles and snakes. Most of these I can pull from the Monster Settings, but I’ll create custom stats for the Naga and the cultist leader, at least. I want them to feel new and different and have some cool ideas for how that might look. I use the monster creation rules to put them together.
\stopExample
 
\subsection{Direct Conversion}  \index{Direct Conversion} \index{Direct} \index{Conversion}
 

If you run across a monster that you haven't already created and which you don't know well enough to convert using the monster creation rules you can instead convert them directly.

 
\subsubsection{Damage}  \index{Damage} \index{Damage}
 

If the monster's damage is a single dice with a bonus of up to +10 keep it as-is. If the monster's damage uses multiple dice of the same size roll the listed dice and take the highest result. If the monster uses multiple dice of different sizes roll only the largest and take the highest result.

 
\subsubsection{HP}  \index{HP} \index{Hp}
 

If the monster's HP is listed as Hit Dice take the maximum value of the first HD and add one for each additional hit dice. If the monster's HP is listed as a number with no Hit Dice divide the HP by 4.

 
\subsubsection{Armor}  \index{Armor} \index{Armor}
 

If the monster's AC is average give it 1 armor. If the monster's AC is low, give it 0 armor. If the monster's AC is high give it 2 armor, 3 armor for beasts that are all about defense. If it's nearly invulnerable, 4 armor. +1 armor if it's defenses are magical.

 
\subsubsection{Moves and Instinct}  \index{Moves and Instinct} \index{Moves} \index{Instinct}
 

Look at the special abilities or attacks listed for the monster, these form the basis for its moves.

 
\section{Maps}  \index{Maps} \index{Maps}
 

One of the biggest differences between Dungeon World and many other fantasy RPGs is the concept of maps and mapping. In many games, you’ll see a square-by-square map denoting precisely what goes where, often presented to give as much detail as possible and leave little to the imagination save the description of the location in question. Dungeon World often leans the opposite direction—maps marked with empty space and a one or two word description like “blades” or “scary.” To adapt an existing adventure for use in Dungeon World, simply keep in mind your Principles and Agenda. Primarily, keep in mind that as the GM, it’s your job to “draw maps, leave blanks” and to “ask questions and use the answers.”

  

To that end, it’s often best to re-draw the map entirely, if you have time. Don’t copy it inch-by-inch but redraw it freehand, leaving spaces and drawing out new rooms, if you’d like. Don’t stick to the map exactly as written, but give yourself some creative license. The idea here is to give yourself room to expand—to allow the players’ reaction to the adventure to surprise and inspire you. If you’ve got the whole map nailed down in advance, there’s nowhere to go you don’t already know about, is there? Pick a few rooms that don’t interest you and wipe out their inhabitants. Draw a new tunnel or two. This will give you some space to play around once you get into the game itself.

 

If you don’t have the time or inclination to re-draw the map, don’t worry. Just take the original map, make a few notes about what might go where and leave the rest blank. When the players go into that room marked “4f” don’t look it up, just make a guess at what might be there based on your notes and what else has been happening. You’ll find a comfortable balance between freely playing out what happens and consulting your prep as you go along.

 
\startExample
The maps that come with my adventure are a good mix of fun and cool and sort of boring fluff. I’ll keep most of what the dungeon describes under the city—the lair of the troglodytes and the secret caves where the captive villagers are being kept—but I’m going to throw away a lot of the stuff about the village itself and just leave blank spaces. It’ll give me room to use the answers to questions like “who do you already know, here?” and “who lives in the abandoned hut up the road?” I’ve made some notes about where the map and my Fronts intersect, but mostly I’ve just given myself room to explore.
\stopExample
 
\section{Magic \& Treasure}  \index{Magic \& Treasure} \index{Magic} \index{\&} \index{Treasure}
 

Two things that are, traditionally, a “big deal” in published modules are treasure and magic items. This is less relevant in Dungeon World (as the reward cycle for characters is more about “doing” than about “having”) but it’s still fun to drudge through a dungeon or explore lost ruins and come up with cool magic items and piles of gold! Like the map, it’s useful to get an idea of the kinds of stuff that might be found in the adventure—anything particularly called out in the text as relevant to the adventure itself (a magic sword that can be used to wound the golem on level 4, or a pendant belonging to the prince captured in room 3) is particularly important. Like monsters, it’s better to look at magic items in terms of what purpose they fulfill: what they’re “for” rather than the damage or armor bonus they might give. Dungeon World isn’t built on balancing treasure against character level, for example, so just look through the adventure for items that seem cool or fun or interesting and create new magic items (with custom moves as necessary) wherever you think it’s needed. This is possibly the easiest step of conversion. Again, you can leave yourself exploratory room, here. Make notes to yourself like “the wizard has a magic staff, what does it do?” and find that out in play. Ask the players about it, see what they have to say. Let Spout Lore do some work for you. “You’ve heard that the wizard here has a strange magical staff. What rumors have you heard of its origins?”

 
\section{Introductory Moves}  \index{Introductory Moves} \index{Introductory} \index{Moves}
 

This step is entirely optional, but can be really useful when running through an adventure for a convention group or other group where running through a full “first session” process just isn’t possible. You can take variables of the adventure and create “hooks” for that adventure, writing custom moves to be made after character creation but before play starts. These moves will serve to engage the characters in the fiction and give them something special to prepare them or hook them into what’s about to happen. You can write one for each class, or bundle them together, if you like. Here’s an example:

 
\startExample
Fighter, someone who loves you gave you a gift before you left for a life of adventure. Roll + CHA and tell us how much they love you. On a 10+ pick two heirlooms, on a 7-9 pick one . On a miss, well, good intentions count for something, right?
\stopExample
 
\startitemize[1,packed]

\item a vial of antivenom

 
\item a shield that glows with silver light

 
\item a rusted old key in the shape of a lizard


\stopitemize
 

These sorts of moves can give the players the sense that their characters are tied to the situation at hand, and open the door for more lines of question-and-answer play that can fill the game world with life. Think about the Fronts, the things they endanger, the riches they might protect and their impact on the world. Let these intro moves flow from that understanding, creating a great kick-start to the adventure.

  





 



Sometimes the players will come across someone who becomes important in the moment. When the ritual goes wrong and a poor captive gets the power cosmic what does that villager do with it? Who were they?

 

When you need a quick NPC all you need is an instinct and some way to pursue it. We call that a Knack, it can be anything from a skill to a title to a debt owed. Combine the two and you have an NPC who has something they want and a way to try to get it, you're ready to go.

 
\section{100 Instincts}  \index{100 Instincts} \index{100} \index{Instincts}
 
\startitemize[n,packed]

\item To avenge

 
\item To spread the good word

 
\item To reunite with a loved one

 
\item To make money

 
\item To make amends

 
\item To explore a mysterious place

 
\item To uncover a hidden truth

 
\item To locate a lost thing

 
\item To kill a hated foe

 
\item To conquer a far-away land

 
\item To cure an illness

 
\item To craft a masterwork

 
\item To survive just one more day

 
\item To earn affection

 
\item To prove a point

 
\item To be smarter, faster and stronger

 
\item To heal an old wound

 
\item To extinguish an evil forever

 
\item To hide from a shameful fact

 
\item To evangelise

 
\item To spread suffering

 
\item To prove worth

 
\item To rise in rank

 
\item To be praised

 
\item To discover the truth

 
\item To make good on a bet

 
\item To get out of an obligation

 
\item To convince someone to do their dirty work

 
\item To steal something valuable

 
\item To overcome a bad habit

 
\item To commit an atrocity

 
\item To earn renown

 
\item To accumulate power

 
\item To save someone from a monstrosity

 
\item To teach

 
\item To settle down

 
\item To get just one more haul

 
\item To preserve the law

 
\item To discover

 
\item To devour

 
\item To restore the family name

 
\item To live a quiet life

 
\item To help others

 
\item To atone

 
\item To prove their worth

 
\item To gain honor

 
\item To expand their land

 
\item To gain a title

 
\item To retreat from society

 
\item To escape

 
\item To party

 
\item To return home

 
\item To serve

 
\item To reclaim what was taken

 
\item To do what must be done

 
\item To be a champion

 
\item To avoid notice

 
\item To help a family member

 
\item To perfect a skill

 
\item To travel

 
\item To overcome a disadvantage

 
\item To play the game

 
\item To establish a dynasty

 
\item To improve the realm

 
\item To retire

 
\item To recover a lost memory

 
\item To battle

 
\item To become a terror to criminals

 
\item To raise dragons

 
\item To live up to expectations

 
\item To become someone else

 
\item To do what can't be done

 
\item To be remembered in song

 
\item To be forgotten

 
\item To find true love

 
\item To lose their mind

 
\item To indulge

 
\item To make the best of it

 
\item To find the one

 
\item To destroy an artifact

 
\item To show them all

 
\item To bring about unending summer

 
\item To fly

 
\item To find the six-fingered man

 
\item To wake the ancient sleepers

 
\item To entertain

 
\item To follow an order

 
\item To die gloriously

 
\item To be careful

 
\item To show kindness

 
\item To not screw it all up

 
\item To uncover the past

 
\item To go where no man has gone before

 
\item To do good

 
\item To become a beast

 
\item To spill blood

 
\item To live forever

 
\item To hunt the most dangerous game

 
\item To hate

 
\item To run away


\stopitemize
 
\section{100 Knacks}  \index{100 Knacks} \index{100} \index{Knacks}
 
\startitemize[n,packed]

\item Criminal connections

 
\item Muscle

 
\item Skill with a specific weapon

 
\item Hedge wizardry

 
\item Comprehensive local knowledge

 
\item Noble blood

 
\item A one-of-a-kind item

 
\item Special destiny

 
\item Unique perspective

 
\item Hidden knowledge

 
\item Magical awareness

 
\item Abnormal parentage

 
\item Political leverage

 
\item A tie to a monster

 
\item A secret

 
\item True love

 
\item An innocent heart

 
\item A plan for the perfect crime

 
\item A one-way ticket to paradise

 
\item A mysterious ore

 
\item Money, money, money

 
\item Divine blessing

 
\item Immunity from the law

 
\item Prophecy

 
\item Secret martial arts techniques

 
\item A ring of power

 
\item A much-needed bag of taters

 
\item A heart

 
\item A fortified position

 
\item Lawmaking

 
\item Tongues

 
\item A discerning eye

 
\item Endurance

 
\item A safe place

 
\item Visions

 
\item A beautiful mind

 
\item A clear voice

 
\item Stunning looks

 
\item A catchy tune

 
\item Invention

 
\item Baking

 
\item Brewing

 
\item Smelting

 
\item Woodworking

 
\item Writing

 
\item Immunity to fire

 
\item Cooking

 
\item Storytelling

 
\item Ratcatching

 
\item Lying

 
\item Utter unremarkableness

 
\item Mind-bending sexiness

 
\item Undefinable coolness

 
\item A way with knots

 
\item Wheels of polished steel

 
\item A magic carpet

 
\item Endless ideas

 
\item Persistence

 
\item A stockpile of food

 
\item A hidden path

 
\item Piety

 
\item Resistance to disease

 
\item A library

 
\item A silver tongue

 
\item Bloodline

 
\item An innate spell

 
\item Balance

 
\item Souls

 
\item Speed

 
\item A sense of right and wrong

 
\item Certainty

 
\item An eye for detail

 
\item Heroic self-sacrifice

 
\item Sense of direction

 
\item A big idea

 
\item A hidden entrance to the city

 
\item The love of someone powerful

 
\item Unquestioning loyalty

 
\item Exotic fruit

 
\item Poison

 
\item Perfect memory

 
\item The language of birds

 
\item A key to an important door

 
\item Metalworking

 
\item Mysterious benefactors

 
\item Steely nerves

 
\item Bluffing

 
\item A trained wolf

 
\item A long-lost sibling, regained

 
\item An arrow with your name on it

 
\item A true name

 
\item Luck

 
\item The attention of supernatural powers

 
\item Kindness

 
\item Strange tattoos

 
\item A majestic beard

 
\item A book in a strange language

 
\item Power overwhelming

 
\item Delusions of grandeur

 
\item The wind at his back and a spring in his step


\stopitemize
 




\stopcolumns
\page
\setupbodyfont[rm,8pt]
\startcolumns[n=2]
\completeindex


