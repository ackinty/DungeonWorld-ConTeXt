\chapter{Twisted Experiments}
 \index{Twisted Experiments} \index{Twisted} \index{Experiments}
 


\startMonsterName
Bulette	\CMTags{Solitary, Huge, Construct} 
\stopMonsterName
 

Bite (d10+5 damage 3 piercing)	20 HP	3 Armor

 

\CMTags{Close, Forceful} 

 
\startMonsterQualities
{\bf Special Qualities:}  Burrowing
\stopMonsterQualities
 
\startMonsterDescription
A seasoned caravan guard learns to listen for the calls of a scout or sentry with a keen ear. A few extra seconds after the alarm is raised can mean life or death. Different cries mean different responses, too—a call of “orcs!” means draw your sword and steady for blood but a call of “bandits!” says you might be able to bargain. One alarm from the scouts that always, always means it’s time to pack up, whip your horse and run for the hills? “LAND SHARK!” {\em Instinct} : To devour
\stopMonsterDescription
 
\startitemize[1,packed]

\item Drag prey into rough tunnels

 
\item Burst from the earth

 
\item Swallow whole


\stopitemize
 
\startMonsterName
Chimera	\CMTags{Solitary, Large, Construct} 
\stopMonsterName
 

Bite (d10+1 damage)	16 HP	1 Armor

 

\CMTags{Reach} 

 
\startMonsterDescription
Well known and categorized, the chimera is a perfected creature. From the codices of the Mage’s Guild to the famous pages of Cullaina’s Creature Compendium, there’s no confusion about what Chimera means. Two parts lioness, one part serpent, head of a she-goat, and all the vicious magic one can muster. The actual ritual might vary, as might a detail or two—more creative sorcerers switch the flame breath for acid, perhaps. Used as a guardian, an assassin or merely an instrument of chaos unchained, it matters little. The chimera is the worst sort of abomination: an intentional affront to all natural life. {\em Instinct} : To do as commanded
\stopMonsterDescription
 
\startitemize[1,packed]

\item Belch forth flame

 
\item Run them over

 
\item Poison them


\stopitemize
 
\startMonsterName
Derro	\CMTags{Horde, Devious, Intelligent, Organized} 
\stopMonsterName
 

Pickaxe (2d6·w damage)	3 HP	2 Armor

 

\CMTags{Close} 

 
\startMonsterQualities
{\bf Special Qualities:}  Telepathy
\stopMonsterQualities
 
\startMonsterDescription
It’s typical to think that all the malignant arcane monsters made in this world are birthed by wizards, sorcerers, and their ilk. That the colleges and towers of Dungeon World are womb to every bleak experiment. There are mistakes made in the depths of the earth, too. These ones, the Derro, are the mistakes of a long-forgotten dwarven alchemist. The derro don’t forget, though. Twisted and hateful, the Derro can be spotted by their swollen skulls, brain-matter grown too large. They do not speak except in thoughts to one another and plot in the silent dark to extract sweetest revenge—that of the created on the creator. {\em Instinct} : To replace dwarves
\stopMonsterDescription
 
\startitemize[1,packed]

\item Fill a mind with foreign thoughts

 
\item Take control of a beast's mind


\stopitemize
 
\startMonsterName
Digester	\CMTags{Solitary, Large, Construct} 
\stopMonsterName
 

Acid (d10+1 damage ignores armor)	16 HP	1 Armor

 

\CMTags{Close, Reach} 

 
\startMonsterQualities
{\bf Special Qualities:}  Digest acid secretion
\stopMonsterQualities
 
\startMonsterDescription
It’s okay, magical experimentation is a messy science. For every beautiful pegasus there’s a half-done creature that wasn’t quite right. We understand. The goblin-elephant you thought was such a great idea. The Gelatinous Drake. Just examples. No judgement here. Anyway, we’ve got something for that. We call it the Digester. Yes, just like it sounds. Strange looking, I know, and the smell isn’t the best, but this thing—it’ll eat magic like Svenloff the Stout drinks ale. Next time one of these unfortunate accidents occurs, just point the Digester at it and all your troubles drain away. Just keep an eye on it. Damn thing ate my wand last week. {\em Instinct} : To digest
\stopMonsterDescription
 
\startitemize[1,packed]

\item Eat away at something

 
\item Draw sustenance


\stopitemize
 
\startMonsterName
Ethereal Filcher	\CMTags{Solitary, Devious, Planar} 
\stopMonsterName
 

Stolen dagger (2d8·w damage)	12 HP	1 Armor

 

\CMTags{Close, Reach} 

 
\startMonsterQualities
{\bf Special Qualities:}  Burrowing
\stopMonsterQualities
 
\startMonsterDescription
Things go missing. A sock, a silver spoon, your dead mother’s bones. We blame the maid, or bad luck, or just a moment of stupid forgetfulness and we move on. We never get to see the real cause of these problems. The spidery thing with human hands and eyes as blue as the deep ethereal from whence the creature comes. We never see the nest it makes of astral silver webbing and stolen objects arranged in some madness pattern. We never watch it assemble its collection of halfling finger-bones, stolen from the hands of the sleeping. We’re lucky, that way. {\em Instinct} : To steal
\stopMonsterDescription
 
\startitemize[1,packed]

\item Take something important to it's planar lair

 
\item Retreat to the Ethereal plane

 
\item Use an item from its lair


\stopitemize
 
\startMonsterName
Ettin	\CMTags{Solitary, Large, Construct} 
\stopMonsterName
 

Club (d10+3 damage)	16 HP	1 Armor

 

\CMTags{Close, Reach, Forceful} 

 
\startMonsterQualities
{\bf Special Qualities:}  Two heads
\stopMonsterQualities
 
\startMonsterDescription
What could possibly be better than an idiotic angry hill giant? One with two heads. Fantastic idea, really. Grade A stuff. {\em Instinct} : To smash
\stopMonsterDescription
 
\startitemize[1,packed]

\item Attack two enemies at once

 
\item Defend its creator


\stopitemize
 
\startMonsterName
Girallon	\CMTags{Solitary, Huge} 
\stopMonsterName
 

Rend (d10+5 damage)	20 HP	1 Armor

 

\CMTags{Reach, Forceful} 

 
\startMonsterQualities
{\bf Special Qualities:}  Many arms
\stopMonsterQualities
 
\startMonsterDescription
The pounding of the jungle drums calls to it. The slab of meat on the sacrificial stone to lure in the Great Ape. Girallon, they call it—a name from the long-forgotten tongue of the kings who bred the beast. Taller than a building, some say. Cloaked in ivory fur with tusks as long as scimitars. Four arms? Six? The rumors are hard to verify. Every year it is the same: some explorer visits the jungle villages seeking the Ape and return, never quite the same, never with a trophy. The pounding of the drums goes on. {\em Instinct} : To rule
\stopMonsterDescription
 
\startitemize[1,packed]

\item Answer the call of sacrifice

 
\item Drive them from the jungle

 
\item Throw someone


\stopitemize
 
\startMonsterName
Iron Golem	\CMTags{Group, Large, Construct} 
\stopMonsterName
 

Slam (d8+5 damage)	10 HP	3 Armor

 

\CMTags{Close, Reach, Forceful} 

 
\startMonsterQualities
{\bf Special Qualities:}  Metal
\stopMonsterQualities
 
\startMonsterDescription
A staple of the enchanters art. Every golemist and mechano-thaumaturge in the Kingdoms knows this. Iron is a misnomer, though. These guardians are crafted of any metal, really. Steel, copper, or even gold, in some small cases. As much an art as a science, the crafting of a fine golem is as respected in the Kingdoms as a bridge newly built or a castle erected in the mountains. Unceasing watchdogs, stalwart defenders, the iron golem lives to serve, following its orders eternally. Any enchanter worth his salt can craft one, if he can afford the materials. If not… {\em Instinct} : To serve
\stopMonsterDescription
 
\startitemize[1,packed]

\item Follow orders implacably

 
\item Use a special tool or adaptation built-in


\stopitemize
 
\startMonsterName
Flesh Golem	\CMTags{Horde} 
\stopMonsterName
 

Claw (d6+2 damage)	3 HP	0 Armor

 

\CMTags{Close, Forceful} 

 
\startMonsterQualities
{\bf Special Qualities:}  Many body parts
\stopMonsterQualities
 
\startMonsterDescription
Stolen bits and pieces in the night. Graveyards stealthily uprooted and maybe tonight an arm. A leg. Another head (the last one came apart too soon). Even the humblest hedge-enchanter can make due with what he can and, with a little creativity, well—it’s not only the College that can make life, hmm? We’ll show them. {\em Instinct} : To live
\stopMonsterDescription
 
\startitemize[1,packed]

\item Follow orders

 
\item Detach a body part


\stopitemize
 
\startMonsterName
Kraken	\CMTags{Solitary, Huge} 
\stopMonsterName
 

Slam (d10+5 damage)	20 HP	2 Armor

 

\CMTags{Reach, Forceful} 

 
\startMonsterQualities
{\bf Special Qualities:}  Aquatic
\stopMonsterQualities
 
\startMonsterDescription
A cephalo-what? No, boy. Not “a kraken” but “the kraken”. I don’t know what nonsense they taught you at that school you say you’re from, but here, we know to respect the Hungerer. Right, that’s what we call it, The Hungerer in the Deep to be more proper. Ain’t no god, though we’ve got those, too. It’s a squid! A mighty squid with tentacles thicker ‘round than a barrel and eyes the size of the full moon. Smart, too, the Hungerer. Knows just when to strike—when you’re all too drunk or too tired or run out of clean water, that’s when he gets you. No, I ain’t ever seen him. I’m alive, aren’t I? {\em Instinct} : To rule the ocean
\stopMonsterDescription
 
\startitemize[1,packed]

\item Drag a person or ship to a watery grave

 
\item Wrap them in tentacles


\stopitemize
 
\startMonsterName
Manticore	\CMTags{Solitary, Large, Construct} 
\stopMonsterName
 

Sting (d10+1 damage 1 piercing)	16 HP	3 Armor

 

\CMTags{Close, Reach, Messy} 

 
\startMonsterQualities
{\bf Special Qualities:}  Wings
\stopMonsterQualities
 
\startMonsterDescription
If the chimera is the first step down a dark path, the manticore is a door that can’t be closed once its been opened. A lion, a scorpion, the wings of a drake. All difficult to obtain but not impossible and just animals, anyway. The last component, the hissing hateful face of the beast, is the ingredient that makes a manticore so cruel. Young or old, man or woman, it matters not but that they are human, living and breathing, married to the creature with twisted magic. All sense of who they are is lost, and maybe that’s a blessing, but the beast is born from human suffering. No wonder, then, that they’re all so eager to kill. {\em Instinct} : To kill
\stopMonsterDescription
 
\startitemize[1,packed]

\item Poison them

 
\item Rip something apart


\stopitemize
 
\startMonsterName
Owlbear	\CMTags{Solitary, Construct} 
\stopMonsterName
 

Claw (d10 damage)	12 HP	2 Armor

 

\CMTags{Close} 

 
\startMonsterDescription
Body of a bear. Feathers of an owl. Beak, claws, and excellent night vision. What’s not to love? {\em Instinct} : To hunt
\stopMonsterDescription
 
\startitemize[1,packed]

\item Strike from darkness


\stopitemize
 
\startMonsterName
Pegasus	\CMTags{Group, Construct} 
\stopMonsterName
 

Stomp (d8 damage)	10 HP	1 Armor

 

\CMTags{Close} 

 
\startMonsterQualities
{\bf Special Qualities:}  Wings
\stopMonsterQualities
 
\startMonsterDescription
Don’t go thinking that every creature not natural-born is a horrible abomination. Don’t imagine for a second that they’re all tentacles and screaming and blood or whatever. Take this noble beast, for example. Lovely thing, isn’t it? A fine white horse with the wings of a swan. Don’t look like it ought to be able to fly, but it does. The elves work miracles, in their own way. They breed true—that’s the purity of elf-magic at work. Hatching from little crystal eggs and bonded with their riders for life. There’s still some beauty in the world, mark my words. {\em Instinct} : To carry aloft
\stopMonsterDescription
 
\startitemize[1,packed]

\item Carry a rider into the air

 
\item Give their rider an advantage


\stopitemize
 
\startMonsterName
Rust Monster	\CMTags{Group, Construct} 
\stopMonsterName
 

Corrode (d8 damage ignores armor)	6 HP	3 Armor

 

\CMTags{Close} 

 
\startMonsterQualities
{\bf Special Qualities:}  Corrosive touch
\stopMonsterQualities
 
\startMonsterDescription
A very distinctive-looking creature. Something like a reddish cricket, I think. Long crickety legs, anyhow. Blind, too, as I understand it—they feel their way around with those long moth-looking tendrils. Feed that way, too. Sift through piles of metal for the choicest bits. That’s what they eat, don’t matter the type, neither. Their merest touch turns it all to rusted flakes. Magic lasts longer but under the scrutiny of a rust monster, it’s a foregone conclusion. Only the gods know where they came from, but they’re a curse if you value your belongings. {\em Instinct} : To decay
\stopMonsterDescription
 
\startitemize[1,packed]

\item Turn metal to rust

 
\item Gain strength from consuming metal


\stopitemize
 
\startMonsterName
Xorn	\CMTags{Solitary, Large, Construct} 
\stopMonsterName
 

Maw (d10 damage)	12 HP	2 Armor

 

\CMTags{Close, Reach} 

 
\startMonsterQualities
{\bf Special Qualities:}  Burrowing
\stopMonsterQualities
 
\startMonsterDescription
Dwarf-made elemental garbage muncher. Shaped like a trash bin with a radius of arms to feed excess rock and stone into its gaping maw. They eat stone and excrete light and heat. Perfect for operating a mine or digging out a quarry. Once one gets lost in the sewers below a city, though, or in the foundation of a castle? You’re in deep trouble. They’ll eat and eat until you’ve got nothing left but to collapse the place down on it and move somewhere else. Ask Burrin, Son of Fjornnvald, exile from his clan. I bet he could tell you a story about a Xorn. {\em Instinct} : To eat
\stopMonsterDescription
 
\startitemize[1,packed]

\item Consume stone

 
\item Give off a burst of light and heat


\stopitemize
 






