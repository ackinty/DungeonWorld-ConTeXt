\chapter{The Cleric}
 \index{The Cleric} \index{Cleric}
            

         

The lands of Dungeon World are a gods-forsaken mess. They’re lousy with the walking dead, beasts of all sorts, and the vast unnatural spaces between safe and temple-blessed civilizations. It is a godless world out there. That’s why it needs you.

         

Bringing the glory of your god to the heathens isn’t just in your nature—it’s your calling. It falls to you to proselytize with sword and mace and spell. To cleave deep into the witless heart of the wilds and plant the seed of divinity there. Some say that it is best to keep god close to your heart. You know that’s rubbish. God lives at the edge of a blade. Show the world who is lord.

       

       
\section{Names}  \index{Names} \index{Names}
       

         

           {\em Dwarf} : Durga, Aelfar, Gerda, Rurgosh, Bjorn, Drummond, Helga, Siggrun, Freya

         

           {\em Human} : Wesley, Brinton, Jon, Sara, Hawthorn, Elise, Clarke, Lenore, Piotr, Dahlia, Carmine

       

       
\section{Look}  \index{Look}
       

         

Choose one for each:

         

Kind, Sharp, or Sad Eyes

         

Tonsure, Strange Hair, or Bald

         

Flowing Robes, Habit, or Common Garb

         

Thin, Knobby, or Flabby Body

       

       
\section{Stats}  \index{Stats} \index{Stats}
       

         

Assign these scores to your stats:

 		 		         

17 (+2), 15 (+1), 13 (+1), 11 (+0), 9 (+0), 8 (-1)

         

You start with 8+Constitution HP.

       

       

Your base damage is d6.

       
\section{Starting Moves}  \index{Starting Moves} \index{Moves}
       

         
\startInstructionsAfterHeader
Choose a racial move:
\stopInstructionsAfterHeader
         

           
\subsection{Dwarf}  \index{Dwarf} \index{Dwarf}
           

You are one with stone. When you Commune you are also granted a special version of Words of the Unspeaking which only works on stone as a Rote.

           
\subsection{Human}  \index{Human} \index{Human}
           

Your faith is diverse. Choose one Wizard spell. You can cast and be granted that spell as if it was a Cleric spell.

         

         

           
\startInstructions
You start with these moves:
\stopInstructions
           

             
\subsection{Deity}  \index{Deity} \index{Deity}
             

You serve and worship some deity or power which grants you spells. Give your god a name (maybe Helferth, Sucellus, or Zorica) and choose your deity’s domain:

             
\startitemize[1,packed]
               
\item Healing and Restoration

               
\item Bloody Conquest

               
\item Civilization

               
\item Knowledge and Hidden Things

               
\item The Downtrodden and Forgotten

               
\item What Lies Beneath

             
\stopitemize
             

Choose one precept of your religion:

             
\startitemize[1,packed]
               
\item Your religion preaches the sanctity of suffering, add Petition: Suffering

               
\item Your religion is cultish and insular, add Petition: Gaining Secrets

               
\item Your religion has important sacrificial rights, add Petition: Offering

               
\item Your religion believes in trial by combat, add Petition: Personal Victory

             
\stopitemize
           

           

             
\subsection{Divine Guidance}  \index{Divine Guidance} \index{Divine} \index{Guidance}
             

When you fulfill your religion’s petition your deity grants you some useful knowledge or boon related to their domain. The GM will tell you what.

           

           

             
\subsection{Turn Undead}  \index{Turn Undead} \index{Turn} \index{Undead}
             

When you hold your holy symbol aloft and pray aloud for protection, roll+Wis. On a hit so long as you continue to pray and brandish your holy symbol no undead may come within reach of you. On a 10+ intelligent undead are momentarily dazed by the radiance of your god and mindless undead flee. If you move aggressively towards an undead creature while Turning them it break the effects and they are able to act as normal. Intelligent undead, vampires and so on, may still find ways to harry you from afar. They're clever like that.

           

           

             
\subsection{Commune}  \index{Commune} \index{Commune}
             

When you spend uninterrupted time (an hour or so) in quiet communion with your deity, you lose any spells already granted to you and are granted new spells of your choice whose total levels don't exceed your own+1. You also prepare your rotes; they don't count against your limit. You cannot be granted spells with tags such as Death or Damaging unless you have a move which allows it. You can't prepare spells that are higher level than you.

           

           

             
\subsection{Cast A Spell}  \index{Cast A Spell} \index{Cast} \index{Spell}
             

When you unleash a spell granted to you by your deity, roll+Wis. On a 10+, the spell is successfully cast and your deity does not revoke the spell, so you may cast it again. On a 7-9, the spell is cast, but choose one:

             
\startitemize[1,packed]
               
\item You draw unwelcome attention or put yourself in a spot (the GM will describe it).

               
\item Your casting distances you from your deity—take -1 ongoing to Cast a Spell until you Commune.

               
\item After you cast it, the spell is revoked by your deity. You cannot cast the spell again until you Commune and have it granted to you.

             
\stopitemize
           

         

       

       
\section{Alignment}  \index{Alignment} \index{Alignment}
       
\startInstructionsAfterHeader
Choose an alignment:
\stopInstructionsAfterHeader
       

         
\subsection{Good}  \index{Good} \index{Good}
         

Endanger yourself to heal another

         
\subsection{Lawful}  \index{Lawful} \index{Lawful}
         

Endanger yourself following the precepts of your church or god

         
\subsection{Evil}  \index{Evil} \index{Evil}
         

Harm another to prove the superiority of your church or god

       

       
\section{Gear}  \index{Gear} \index{Gear}
       

         

Your Load is 7+Str. You carry dungeon rations (1 weight, 5 uses) and some symbol of the divine, describe it (weight 0). Choose your defenses:

         
\startitemize[1,packed]
           
\item Chainmail (1 armor, 1 weight)

           
\item Shield (+1 armor, 2 weight)

         
\stopitemize
         

Choose your armament:

         
\startitemize[1,packed]
           
\item Warhammer (Close, 1 weight)

           
\item Mace (Close, 1 weight)

           
\item Staff (Close, Two-handed, 1 weight) and bandages

         
\stopitemize
         

Choose one:

         
\startitemize[1,packed]
           
\item Adventuring gear (1 weight) and dungeon rations (1 weight)

           
\item Healing potion (1 weight)

         
\stopitemize
       

       
\section{Bonds}  \index{Bonds} \index{Bonds}
       

         

Fill in the name of one of your companions in at least one:

         

\thinrules[n=2] has insulted my deity; I do not trust them.

         

\thinrules[n=2] is a good and faithful person; I trust them implicitly.

         

\thinrules[n=2] is in constant danger, I will keep them safe.

         

I am working on converting \thinrules[n=2] to my faith.

       

       
\section{Advanced Moves}  \index{Advanced Moves} \index{Advanced} \index{Moves}
       

         
\startInstructionsAfterHeader
When you gain a level from 2-5, choose from these moves.
\stopInstructionsAfterHeader
         
\subsection{Chosen One}  \index{Chosen One} \index{Chosen}
         

Choose one spell. You are granted that spell as if it was one level lower.

         
\subsection{Inquisitor}  \index{Inquisitor} \index{Inquisitor}
         

Your deity gives you domain over pain and suffering. You can cast spells with the Damaging tag.

         
\subsection{The Scales of Life and Death}  \index{The Scales of Life and Death} \index{Scales} \index{Life} \index{Death}
         

Your deity gives you domain over the dead and undead. You can cast spells with the Death tag.

         
\subsection{Serenity}  \index{Serenity} \index{Serenity}
         

You are able to divide your power effectively. When you cast a spell you ignore the first -1 penalty from ongoing spells.

         
\subsection{First Aid}  \index{First Aid} \index{Aid}
         

Cure Light Wounds does not count against your limit of granted spells.

         
\subsection{Divine Intervention}  \index{Divine Intervention} \index{Divine} \index{Intervention}
         

When you Commune you get 1 hold and lose any hold you already had. Spend that hold when you or an ally takes damage to call on your deity, they intervene in an appropriate idiom (a sudden gust of wind, a lucky slip, a burst of light) and negate the damage.

         
\subsection{Penitent}  \index{Penitent} \index{Penitent}
         

When you take damage and embrace the pain, you may take +1d4 damage (ignoring armor). If you do, take +1 forward to Cast a Spell.

         
\subsection{Empower}  \index{Empower} \index{Empower}
         

When you Cast a Spell, on a 10+ you have the option of choosing from the 7-9 list. If you do, you may choose one of these effects as well:

         
\startitemize[1,packed]
           
\item The spell’s effects are doubled

           
\item The spell’s targets are doubled

         
\stopitemize
         
\subsection{Orison for Guidance}  \index{Orison for Guidance} \index{Orison} \index{Guidance}
         

When you sacrifice something of value to your deity and pray for guidance your deity tells you what it would have you do. If you do it, mark experience.

         
\subsection{Divine Protection}  \index{Divine Protection} \index{Divine} \index{Protection}
         

When you wear no armor or shield you get 2 armor.

         
\subsection{Devoted Healer}  \index{Devoted Healer} \index{Devoted} \index{Healer}
         

When you heal someone else of damage, heal +your level damage.

         
\startInstructions
When you gain a level from 6-10, choose from these moves or the level 2-5 moves.
\stopInstructions
         
\subsection{Anointed}  \index{Anointed} \index{Anointed}
         

Requires: Chosen One

         

Choose one spell. You are granted that spell as if it was one level lower.

         
\subsection{Vicious}  \index{Vicious} \index{Vicious}
         

Requires: Inquisitor

         

When you do damage with a spell, you deal +1d4 damage.

         
\subsection{Reaper}  \index{Reaper} \index{Reaper}
         

Requires: The Scales of Life and Death

         

When you take time after a conflict to dedicate your victory to your deity and deal with the dead, take +1 forward.

         
\subsection{Providence}  \index{Providence} \index{Providence}
         

Replaces: Serenity

         

You ignore the -1 penalty from two spells you maintain.

         
\subsection{Greater First Aid}  \index{Greater First Aid} \index{Greater} \index{Aid}
         

Requires: First Aid

         

Cure Moderate Wounds does not count against your limit of granted spells.

         
\subsection{Divine Invincibility}  \index{Divine Invincibility} \index{Divine} \index{Invincibility}
         

Replaces: Divine Intervention

         

When you Commune you get 2 hold and lose any hold you already had. Spend that hold when you or an ally takes damage to call on your deity, they intervene in an appropriate idiom (a sudden gust of wind, a lucky slip, a burst of light) and negate the damage.

         
\subsection{Martyr}  \index{Martyr} \index{Martyr}
         

Replaces: Penitent

         

When you take damage and embrace the pain, you may take +1d4 damage (ignoring armor). If you do, take +1 forward to Cast a Spell and add your level to any damage done or healed by the spell.

         
\subsection{Divine Armor}  \index{Divine Armor} \index{Divine} \index{Armor}
         

Replaces: Divine Protection

         

When you wear no armor or shield you get 3 armor.

         
\subsection{Greater Empower}  \index{Greater Empower} \index{Greater} \index{Empower}
         

Replaces: Empower

         

When you Cast a Spell, on a 10-11 you have the option of choosing from the 7-9 list. If you do, you may choose one of these effects as well. On a 12+ you get to choose one of these effects for free.

         
\startitemize[1,packed]
           
\item The spell’s effects are doubled

           
\item The spell’s targets are doubled

         
\stopitemize
         
\subsection{Multiclass Dabbler}  \index{Multiclass Dabbler} \index{Multiclass} \index{Dabbler}
         

Get one move from another class. Treat your level as one lower for choosing the move.

       

                
\section{Cleric Spells}  \index{Cleric Spells} \index{Cleric} \index{Spells}
     

       
\subsection{Rotes}  \index{Rotes} \index{Rotes}
       

         
\startSpellName
           \CMSpellName{Light} 	Rote
\stopSpellName
         

An item you touch glows with divine light, about as bright as a torch. It gives off no heat or sound and requires no fuel but is otherwise like a mundane torch. You have complete control of the color of the flame. The spell lasts as long as it is in your presence.

       

       
\startSpellName
         \CMSpellName{Sanctify} 	Rote
\stopSpellName
       

Food or water you hold in your hands while you cast this spell is consecrated by your deity. In addition to now being holy or unholy the affected substance is purified of any mundane spoilage.

       
\startSpellName
         \CMSpellName{Guidance} 	Rote
\stopSpellName
       

The symbol of your deity appears before you and gestures towards the direction or course of action your deity would have you take then disappears. The message is through gesture only; your communication through this spell is severely limited.

     

     

       
\subsection{1st Level Spells}  \index{1st Level Spells} \index{1st} \index{Level} \index{Spells}
       
\startSpellName
         \CMSpellName{Bless} 	Level 1
\stopSpellName
       

You deity smiles on the target in combat. They take +1 ongoing so long as battle continues and they stand and fight.

       
\startSpellName
         \CMSpellName{Cure Light Wounds} 	Level 1
\stopSpellName
       

At your touch wounds scab and bones cease to ache. Heal an ally of 1d8 damage.

       
\startSpellName
         \CMSpellName{Detect Alignment} 	Level 1
\stopSpellName
       

When you cast this spell choose an alignment: Good, Evil, or Neutral. One of your senses is briefly able to detect that alignment. The GM will tell you what here is of that alignment.

       
\startSpellName
         \CMSpellName{Inflict Light Wounds} 	Level 1	{\em Damaging} 
\stopSpellName
       

You open wounds, channeling the wrath of your god. Deal 1d8 damage to the target ignoring armor.

       
\startSpellName
         \CMSpellName{Magic Weapon} 	Level 1	 {\em Ongoing} 
\stopSpellName
       

The weapon you hold while casting does +1d4 damage until you dismiss this spell. Until you dismiss this spell you take -1 to Cast a Spell.

       
\startSpellName
         \CMSpellName{Sanctuary} 	Level 1
\stopSpellName
       

You make an area holy to your deity. Walk the perimeter of the area. So long as you stay within that area you are alerted whenever someone acts with malice within the sanctuary (including entering with harmful intent). Anyone who receives healing within a Sanctuary heals +1d4 HP.

       
\startSpellName
         \CMSpellName{Speak With Dead} 	Level 1	{\em Death} 
\stopSpellName
       

A corpse converses with you briefly. It will answer any three questions you pose to it to the best of the knowledge it had in life and the knowledge it gained in death.

     

     

       
\subsection{3rd Level Spells}  \index{3rd Level Spells} \index{3rd} \index{Level} \index{Spells}
       
\startSpellName
         \CMSpellName{Animate Dead} 	Level 3	Death
\stopSpellName
       

You invoke a hungry spirit to possess a recently-dead body and act for you. This forms a zombie that follows your orders to the best of its limited abilities. Treat the zombie as your character, but with access to only the basic moves. It has a +1 modifier for all stats and 1 HP. You can only have one zombie at a time. You get 1d4 of these effects:

       
\startitemize[1,packed]
         
\item The zombie is talented. Give one stat a +2 modifier.

         
\item The zombie is durable. It has +2 HP for each level you have.

         
\item The zombie has a functioning brain and can complete complex tasks.

         
\item The zombie is restored by magic—it does not appear obviously dead, at least for a day or two.

       
\stopitemize
       
\startSpellName
         \CMSpellName{Cure Moderate Wounds} 	Level 3
\stopSpellName
       

You staunch bleeding and set bones through magic. Heal an ally of 2d8 damage.

       
\startSpellName
         \CMSpellName{Inflict Moderate Wounds} 	Level 3	{\em Damaging} 
\stopSpellName
       

You break bones and leave gushing wounds. Deal 1d10 damage ignoring armor.

       
\startSpellName
         \CMSpellName{Resurrection} 	Level 3
\stopSpellName
       

Tell the GM you would like to resurrect a corpse whose soul has not yet fully departed this world. The GM will tell you "yes, you can resurrect them, but first…" or "yes, you can resurrect them now, but it won't be permanent until…" and then one to all of the things from this list:

       
\startitemize[1,packed]
         
\item It's going to take days/weeks/months

         
\item You must \_\_\_\_

         
\item You must get help from \_\_\_\_

         
\item It will require a lot of money

         
\item You must sacrifice \_\_\_\_ to do it

       
\stopitemize
       
\startSpellName
         \CMSpellName{Hold Person} 	Level 3
\stopSpellName
       

Choose a creature you can see. Until you Cast a Spell or leave their presence they cannot act except to speak. If they're harmed this effect ends.

     

     

       
\subsection{5th Level Spells}  \index{5th Level Spells} \index{5th} \index{Level} \index{Spells}
       
\startSpellName
         \CMSpellName{Revelation} 	Level 5
\stopSpellName
       

Your deity answers your prayers with a moment of perfect understanding. The GM will explain the root cause of the current situation. When acting on the information, you take +1 Forward.

       
\startSpellName
         \CMSpellName{Cure Critical Wounds} 	Level 5
\stopSpellName
       

Heal an ally of 3d8 damage.

       
\startSpellName
         \CMSpellName{Divination} 	Level 5
\stopSpellName
       

Name a person, place, or thing you want to learn about. Your deity grants you visions of the target, as clear as if you were there.

       
\startSpellName
         \CMSpellName{Inflict Critical Wounds} 	Level 5	Damaging
\stopSpellName
       

You cause horrific harm. Deal 1d8+1d4 damage ignoring armor.

       
\startSpellName
         \CMSpellName{Words of the Unspeaking} 	Level 5
\stopSpellName
       

With a touch you speak to the spirits within things. The non-living object you touch responds to three questions you pose, answering them as best it can.

       
\startSpellName
         \CMSpellName{True Seeing} 	Level 5
\stopSpellName
       

For a brief moment your vision is opened to the true nature of everything you lay your eyes on. You pierce illusions and see things that have been hidden. The GM will describe the area before you ignoring any illusions and falsehoods, magical or otherwise.

       
\startSpellName
         \CMSpellName{Trap Soul} 	Level 5	Death
\stopSpellName
       

When cast in the presence of a ghost or recently dead body this spell traps the soul in a gem you provide. While trapped the soul answers every question posed to it and can not resist your requests. Once released the soul is likely to hold a grudge against its captor.

     

     

       
\subsection{7th Level Spells}  \index{7th Level Spells} \index{7th} \index{Level} \index{Spells}
       
\startSpellName
         \CMSpellName{Word of Recall} 	Level 7
\stopSpellName
       

Choose a word. The first time after casting this spell that you speak the chosen word, you and any allies touching you when you cast the spell are immediately returned to the exact spot you cast the spell at. Casting Word of Recall again before speaking the word replaces the earlier recall.

       
\startSpellName
         \CMSpellName{Heal} 	Level 7
\stopSpellName
       

Touch an ally and you may restore up to your maximum HP to them.

       
\startSpellName
         \CMSpellName{Harm} 	Level 7
\stopSpellName
       

Touch an enemy and strike them with divine wrath—deal 2d8 damage to them and 1d6 damage to yourself (ignores armor).

       
\startSpellName
         \CMSpellName{Sever} 	Level 7	Damaging
\stopSpellName
       

Choose an appendage on the target such as an arm, tentacle, or wing. The appendage is magically severed from their body, causing no damage but considerable pain. Missing an appendage may, for example, keep a winged creature from flying, or a bull from goring you on its horns. While you maintain the spell you take -1 to Cast a Spell.

       
\startSpellName
         \CMSpellName{Mark of Death} 	Level 7	Death
\stopSpellName
       

Choose a type of creature. You inscribe runes matching that type of creature. The next mortal creature of that type to look on the runes is immediately slain.

       
\startSpellName
         \CMSpellName{Control Weather} 	Level 7
\stopSpellName
       

Pray for rain—or sun, wind or snow. Within a day or so, your god will answer. The weather will change according to your will and last a handful of days.

     

     

       
\subsection{9th Level Spells}  \index{9th Level Spells} \index{9th} \index{Level} \index{Spells}
       
\startSpellName
         \CMSpellName{Dictum} 	Level 9
\stopSpellName
       

All non-Lawful creatures that can hear you stand paralyzed so long as you continue speaking.

       
\startSpellName
         \CMSpellName{Gibber} 	Level 9
\stopSpellName
       

All non-Chaotic creatures that can hear you stand paralyzed so long as you continue speaking.

       
\startSpellName
         \CMSpellName{Repair} 	Level 9
\stopSpellName
       

Choose one event in the target's past. All effects of that event, including damage, poison, disease, and magical effects, are ended and repaired. HP and diseases are healed, poisons are neutralized, magical effects are ended.

       
\startSpellName
         \CMSpellName{Divine Presence} 	Level 9	Damaging
\stopSpellName
       

Every creature must ask your leave to enter your presence, and you must speak permission for them to enter. Any creature that you deny permission takes an extra 1d10 damage whenever they take damage in your presence.

       
\startSpellName
         \CMSpellName{Consume Unlife} 	Level 9	Death
\stopSpellName
       

The mindless undead creature you touch is destroyed and you steal its death energy to heal yourself or the next ally you touch of damage equal to its current HP.

       
\startSpellName
         \CMSpellName{Peace} 	Level 9
\stopSpellName
       

Choose one traumatic memory in the target's past. The target's memory of that event is calmly erased. If the target is a PC, they must be willing.

     

           
