\chapter{The Cleric}
 \index{The Cleric} \index{Cleric}
 





The lands of Dungeon World are a gods-forsaken mess. They’re lousy with the walking dead, beasts of all sorts, and the vast unnatural spaces between safe and temple-blessed civilizations. It is a godless world out there. That’s why it needs you.

 

Bringing the glory of your god to the heathens isn’t just in your nature—it’s your calling. It falls to you to proselytize with sword and mace and spell. To cleave deep into the witless heart of the wilds and plant the seed of divinity there. Some say that it is best to keep god close to your heart. You know that’s rubbish. God lives at the edge of a blade. Show the world who is lord.



 
\section{Names}  \index{Names} \index{Names}
 



{\em Dwarf} : Durga, Aelfar, Gerda, Rurgosh, Bjorn, Drummond, Helga, Siggrun, Freya

 

{\em Human} : Wesley, Brinton, Jon, Sara, Hawthorn, Elise, Clarke, Lenore, Piotr, Dahlia, Carmine



 
\section{Look}  \index{Look}
 



Choose one for each:

 

Kind Eyes, Sharp Eyes, or Sad Eyes

 

Tonsure, Strange Hair, or Bald

 

Flowing Robes, Habit, or Common Garb

 

Thin Body, Knobby Body, or Flabby Body



 
\section{Stats}  \index{Stats} \index{Stats}
 



Assign these scores to your stats:

 

17 (+2), 15 (+1), 13 (+1), 11 (+0), 9 (+0), 8 (-1)

 

You start with 8+Constitution HP.



 

Your base damage is d6.

 
\section{Starting Moves}  \index{Starting Moves} \index{Moves}
 


\startInstructionsAfterHeader
Choose a racial move:
\stopInstructionsAfterHeader
 


\subsection{Dwarf}  \index{Dwarf} \index{Dwarf}
 

You are one with stone. When you Commune you are also granted a special version of Words of the Unspeaking which only works on stone as a Rote.

 
\subsection{Human}  \index{Human} \index{Human}
 

Your faith is diverse. Choose one Wizard spell. You can cast and be granted that spell as if it was a Cleric spell.



 


\startInstructions
You start with these moves:
\stopInstructions
 


\subsection{Deity}  \index{Deity} \index{Deity}
 

You serve and worship some deity or power which grants you spells. Give your god a name (maybe Helferth, Sucellus, or Zorica) and choose your deity’s domain:

 
\startitemize[1,packed]

\item Healing and Restoration

 
\item Bloody Conquest

 
\item Civilization

 
\item Knowledge and Hidden Things

 
\item The Downtrodden and Forgotten

 
\item What Lies Beneath


\stopitemize
 

Choose one precept of your religion:

 
\startitemize[1,packed]

\item Your religion preaches the sanctity of suffering, add Petition: Suffering

 
\item Your religion is cultish and insular, add Petition: Gaining Secrets

 
\item Your religion has important sacrificial rites, add Petition: Offering

 
\item Your religion believes in trial by combat, add Petition: Personal Victory


\stopitemize


 


\subsection{Divine Guidance}  \index{Divine Guidance} \index{Divine} \index{Guidance}
 

When you fulfill your religion’s petition your deity grants you some useful knowledge or boon related to their domain. The GM will tell you what.



 


\subsection{Turn Undead}  \index{Turn Undead} \index{Turn} \index{Undead}
 

When you hold your holy symbol aloft and pray aloud for protection, roll+Wis. On a hit so long as you continue to pray and brandish your holy symbol no undead may come within reach of you. On a 10+ intelligent undead are momentarily dazed by the radiance of your god and mindless undead flee. If you move aggressively towards an undead creature while Turning them it breaks the effects and they are able to act as normal. Intelligent undead, vampires and so on, may still find ways to harry you from afar. They're clever like that.



 


\subsection{Commune}  \index{Commune} \index{Commune}
 

When you spend uninterrupted time (an hour or so) in quiet communion with your deity, you lose any spells already granted to you and are granted new spells of your choice whose total levels don't exceed your own+1. You also prepare your rotes; they don't count against your limit. You can't prepare spells that are higher level than you.



 


\subsection{Cast a Spell}  \index{Cast a Spell} \index{Cast} \index{Spell}
 

When you unleash a spell granted to you by your deity, roll+Wis. On a 10+, the spell is successfully cast and your deity does not revoke the spell, so you may cast it again. On a 7-9, the spell is cast, but choose one:

 
\startitemize[1,packed]

\item You draw unwelcome attention or put yourself in a spot (the GM will describe it).

 
\item Your casting distances you from your deity—take -1 ongoing to Cast a Spell until you Commune.

 
\item After you cast it, the spell is revoked by your deity. You cannot cast the spell again until you Commune and have it granted to you.


\stopitemize






 
\section{Alignment}  \index{Alignment} \index{Alignment}
 
\startInstructionsAfterHeader
Choose an alignment:
\stopInstructionsAfterHeader
 


\subsection{Good}  \index{Good} \index{Good}
 

Endanger yourself to heal another

 
\subsection{Lawful}  \index{Lawful} \index{Lawful}
 

Endanger yourself following the precepts of your church or god

 
\subsection{Evil}  \index{Evil} \index{Evil}
 

Harm another to prove the superiority of your church or god



 
\section{Gear}  \index{Gear} \index{Gear}
 



Your Load is 7+Str. You carry dungeon rations (1 weight, 5 uses) and some symbol of the divine, describe it (weight 0). Choose your defenses:

 
\startitemize[1,packed]

\item Chainmail (1 armor, 1 weight)

 
\item Shield (+1 armor, 2 weight)


\stopitemize
 

Choose your armament:

 
\startitemize[1,packed]

\item Warhammer (Close, 1 weight)

 
\item Mace (Close, 1 weight)

 
\item Staff (Close, Two-handed, 1 weight) and bandages


\stopitemize
 

Choose one:

 
\startitemize[1,packed]

\item Adventuring gear (1 weight) and dungeon rations (1 weight)

 
\item Healing potion (1 weight)


\stopitemize


 
\section{Bonds}  \index{Bonds} \index{Bonds}
 



Fill in the name of one of your companions in at least one:

 

\thinrules[2] has insulted my deity; I do not trust them.

 

\thinrules[2] is a good and faithful person; I trust them implicitly.

 

\thinrules[2] is in constant danger, I will keep them safe.

 

I am working on converting \thinrules[2] to my faith.



 
\section{Advanced Moves}  \index{Advanced Moves} \index{Advanced} \index{Moves}
 


\startInstructionsAfterHeader
When you gain a level from 2-5, choose from these moves.
\stopInstructionsAfterHeader
 
\subsection{Chosen One}  \index{Chosen One} \index{Chosen}
 

Choose one spell. You are granted that spell as if it was one level lower.

 
\subsection{Invigorate}  \index{Invigorate} \index{Invigorate}
 

When you heal someone they take +2 damage forward.

 
\subsection{The Scales of Life and Death}  \index{The Scales of Life and Death} \index{Scales} \index{Life} \index{Death}
 

When someone takes their Last Breath in your presence they take +1 to the roll.

 
\subsection{Serenity}  \index{Serenity} \index{Serenity}
 

You are able to divide your power effectively. When you cast a spell you ignore the first -1 penalty from ongoing spells.

 
\subsection{First Aid}  \index{First Aid} \index{Aid}
 

Cure Light Wounds does not count against your limit of granted spells.

 
\subsection{Divine Intervention}  \index{Divine Intervention} \index{Divine} \index{Intervention}
 

When you Commune you get 1 hold and lose any hold you already had. Spend that hold when you or an ally takes damage to call on your deity, they intervene in an appropriate idiom (a sudden gust of wind, a lucky slip, a burst of light) and negate the damage.

 
\subsection{Penitent}  \index{Penitent} \index{Penitent}
 

When you take damage and embrace the pain, you may take +1d4 damage (ignoring armor). If you do, take +1 forward to Cast a Spell.

 
\subsection{Empower}  \index{Empower} \index{Empower}
 

When you Cast a Spell, on a 10+ you have the option of choosing from the 7-9 list. If you do, you may choose one of these effects as well:

 
\startitemize[1,packed]

\item The spell’s effects are doubled

 
\item The spell’s targets are doubled


\stopitemize
 
\subsection{Orison for Guidance}  \index{Orison for Guidance} \index{Orison} \index{Guidance}
 

When you sacrifice something of value to your deity and pray for guidance your deity tells you what it would have you do. If you do it, mark experience.

 
\subsection{Divine Protection}  \index{Divine Protection} \index{Divine} \index{Protection}
 

When you wear no armor or shield you get 2 armor.

 
\subsection{Devoted Healer}  \index{Devoted Healer} \index{Devoted} \index{Healer}
 

When you heal someone else of damage, heal +your level damage.

 
\startInstructions
When you gain a level from 6-10, choose from these moves or the level 2-5 moves.
\stopInstructions
 
\subsection{Anointed}  \index{Anointed} \index{Anointed}
 

Requires: Chosen One

 

Choose one spell. You are granted that spell as if it was one level lower.

 
\subsection{Vicious}  \index{Vicious} \index{Vicious}
 

Requires: Inquisitor

 

When you do damage with a spell, you deal +1d4 damage.

 
\subsection{Reaper}  \index{Reaper} \index{Reaper}
 

When you take time after a conflict to dedicate your victory to your deity and deal with the dead, take +1 forward.

 
\subsection{Providence}  \index{Providence} \index{Providence}
 

Replaces: Serenity

 

You ignore the -1 penalty from two spells you maintain.

 
\subsection{Greater First Aid}  \index{Greater First Aid} \index{Greater} \index{Aid}
 

Requires: First Aid

 

Cure Moderate Wounds does not count against your limit of granted spells.

 
\subsection{Divine Invincibility}  \index{Divine Invincibility} \index{Divine} \index{Invincibility}
 

Replaces: Divine Intervention

 

When you Commune you get 2 hold and lose any hold you already had. Spend that hold when you or an ally takes damage to call on your deity, they intervene in an appropriate idiom (a sudden gust of wind, a lucky slip, a burst of light) and negate the damage.

 
\subsection{Martyr}  \index{Martyr} \index{Martyr}
 

Replaces: Penitent

 

When you take damage and embrace the pain, you may take +1d4 damage (ignoring armor). If you do, take +1 forward to Cast a Spell and add your level to any damage done or healed by the spell.

 
\subsection{Divine Armor}  \index{Divine Armor} \index{Divine} \index{Armor}
 

Replaces: Divine Protection

 

When you wear no armor or shield you get 3 armor.

 
\subsection{Greater Empower}  \index{Greater Empower} \index{Greater} \index{Empower}
 

Replaces: Empower

 

When you Cast a Spell, on a 10-11 you have the option of choosing from the 7-9 list. If you do, you may choose one of these effects as well. On a 12+ you get to choose one of these effects for free.

 
\startitemize[1,packed]

\item The spell’s effects are doubled

 
\item The spell’s targets are doubled


\stopitemize
 
\subsection{Multiclass Dabbler}  \index{Multiclass Dabbler} \index{Multiclass} \index{Dabbler}
 

Get one move from another class. Treat your level as one lower for choosing the move.










