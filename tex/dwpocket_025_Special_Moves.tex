\chapter{Special Moves}
 \index{Special Moves} \index{Special} \index{Moves}
       

Special moves are moves that come up less often or in more specific situations. They're still the basis of what characters do in Dungeon World—particularly what they do between dungeon crawls and high-flying adventures.

       
\section{Last Breath}  \index{Last Breath} \index{Breath}
       

The Last Breath is the last moment that stands between life and death. Time stands still as Death comes to claim the living. Even those who stay will catch a glimpse of the other side as they fight for their life. Many are changed by this moment—even those who escape alive.

       

The deal offered by death is decided by the GM but it should always be a real choice with real consequences. If the GM offers something completely painless, the move is pointless. If the GM offers a ridiculous price, no one will take it. Think of ways that the character might be changed by the event: a new goal in life, a debt that must be paid, an obligation.

       
\startExample
Sparrow stands at Death's black gates. First the Gm describes what she sees beyond them: "In among the suffering souls you clearly see Lord Hywn. It appears his double dealing has caught up with him." Now for the bargain: "The shadowy form of Death itself steps between you and the gates. 'Here so soon? I enjoy seeing the souls you send me. I'll return you to the world so that you may serve me, but there is a cost: you will never be able to move under the sun again, or you will return to my realm immediately."
\stopExample
       
\section{Encumbrance}  \index{Encumbrance} \index{Encumbrance}
       

A PC's Load stat is determined by their class and Str. Being able to carry more is a clear benefit when trying to carry treasure out of a dungeon or just making sure you can bring along what you need.

       

This move only applies to things a person could walk around and still act with. Carrying a boulder on your back is not encumbrance—you can't really act or move much with it. It effects what moves you can make appropriately in the fiction.

       
\section{Carouse}  \index{Carouse} \index{Carouse}
       

Unless the PCs are particularly extravagant or generous Carousing doesn't cost any gold. If the players are paying someone else's tab or living the high life then it'll costs them appropriately.

       

You can only carouse when you return triumphant. That's what draws the crowd of revelers to surround adventurers as they celebrate their latest haul. If you don't claim your success or your failure is well known then who would want to party with you anyway?

       
\section{Undertake a Perilous Journey}  \index{Undertake a Perilous Journey} \index{Undertake} \index{Perilous} \index{Journey}
       

Distances in Dungeon World are measured in rations. A ration is the amount of supplies used up in a day. Journeys take more rations when they are long or when travel is slow.

       

A perilous journey is the whole way between two locations. You don't roll for one day's journey and then make camp only to roll for the next day's journey, too. Make one roll for the entire trip.

       

This move only applies when you know where you're going. Setting off to explore is not a perilous journey. It's wandering around looking for cool things to discover. Use up rations as you camp and the GM will give you details about the world as you discover them.

       
\section{Make Camp}  \index{Make Camp} \index{Camp}
       

You usually Make Camp so that you can do other things, like Prepare Spells or Dutiful Prayer. Or, you know, sleep soundly at night. Whenever you stop to catch your breath for more than an hour or so, you've probably Made Camp.

       

When camping in dangerous territory the selections made apply to the entire camp. Every PC camping out needs to roll. Camping with fewer than three characters, or without the Ranger, is dangerous—there will always be at least one option not selected.

       

What counts as dangerous territory is up to the GM. She should call for the move. When not Making Camp in dangerous territory the camp is in all ways unexceptional with neither benefits or dangers.

       

Staying a night in an inn or house is Making Camp is a safe location. Regain your hit points as usual, but only mark off a ration if you're eating from the food you carry, not paying for a meal or receiving hospitality.

       
\section{Outstanding Warrants}  \index{Outstanding Warrants} \index{Warrants}
       

This move is only for places where you've caused trouble, not every piece of civilization you enter. Being publicly caught up in someone else's trouble still triggers this move.

       

Civilization generally means the villages, towns and cities of humans, elves, dwarves, and halflings but it can also apply to any relatively lawful establishment of monstrous species, such as orcs or goblins. If the PCs have stayed there as part of the community it's civilization.

                       
