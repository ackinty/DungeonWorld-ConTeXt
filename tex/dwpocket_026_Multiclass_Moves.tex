\chapter{Multiclass Moves}
 \index{Multiclass Moves} \index{Multiclass} \index{Moves}
       
\subsection{Multiclass Dabbler}  \index{Multiclass Dabbler} \index{Multiclass} \index{Dabbler}
       
\subsection{Multiclass Initiate}  \index{Multiclass Initiate} \index{Multiclass} \index{Initiate}
       
\subsection{Multiclass Master}  \index{Multiclass Master} \index{Multiclass} \index{Master}
       

For the purposes of these multiclass moves the Cleric's Commune and Cast a Spell count as one move. Likewise for the Wizard's Spellbook, Prepare Spells, and Cast a Spell.

       

If a multiclass move grants you the ability to cast spells you prepare and cast spells as if you had one level in the casting class. Whenever you level up you increase the level you prepare and cast spells at too.

       
\startExample
When Ajax gains 3rd level he takes Multiclass Dabbler to get Commune and Cast a Spell from the Cleric class. He casts and prepares spells like a first level Cleric: first level spells and rotes only, a total of 2 levels of spells prepared.
\stopExample

.
       
\section{Bard Moves}  

\index{Bard Moves} \index{Bard} \index{Moves}
       
\subsection{Bardic Lore}  \index{Bardic Lore} \index{Bardic} \index{Lore}
       

Treat the areas of your lore like books. Is the upwards-flowing waterfall you just came across something important that would be covered in a book called "On Spells and Magicks?" If so, your Bardic Lore applies.

       

If you care enough to ask a question about it then it's probably important. Don't second guess yourself: if you care enough to want to know more about it then it has some importance.

       
\subsection{Charming and Open}  \index{Charming and Open} \index{Open}
       

Speaking frankly means you really are being open with them, not just giving the appearance of openness. It's your true sincerity that puts others at ease and lets you get information out of them; if you're trying to maintain a lie at the same time you won't get very far.

       
\subsection{It Goes To Eleven}  \index{It Goes To Eleven} \index{Eleven}
       

Of course the creature you effect must have some way of harming your target of choice. Spurring a wolf into a frenzy to attack the eagle lord circling above doesn't do any good, the wolf doesn't have a way to attack it.

       
\subsection{An Ear for Magic}  \index{An Ear for Magic} \index{Ear} \index{Magic}
       

Acting on the answers can mean acting against them or taking advantage of them. Either way you take +1 forward.

       
\section{Cleric Spells}  \index{Cleric Spells} \index{Cleric} \index{Spells}
       
\subsection{Guidance}  \index{Guidance} \index{Guidance}
       

It's up the the creativity of your deity (and the GM) to communicate as much as possible through the motions and gestures of you deity's symbol. You don't get visions or a voice from heaven, just some visual cue of what your deity would have you do (even if it's not in your best interest).

       
\subsection{Magic Weapon}  \index{Magic Weapon} \index{Magic} \index{Weapon}
       

Casting Magic Weapon on the same weapon again has no effect. No matter how many times you cast it on the same weapon it's still just magic +1d4 damage.

       

Magic though is nothing to be scoffed at. Having a magic weapon may give you an advantage against some of the stranger beasts of Dungeon World, ghosts and the sort. The exact effects depend on the monster and circumstances, so make the most of it.

       
\subsection{Animate Dead}  \index{Animate Dead} \index{Animate} \index{Dead}
       

Treating the zombie as your character means you make moves with it's ability scores based on the fiction, just like always. Unless it's brain is functioning on its own the zombie can't do much besides follow the last order it was given, so you'd better stay close. Even if its brain works it's still bound to follow your orders.

       
\section{Fighter Moves}  \index{Fighter Moves} \index{Fighter} \index{Moves}
       
\subsection{Signature Weapon}  \index{Signature Weapon} \index{Signature} \index{Weapon}
       

The base description you choose is just a description. Choosing a spear doesn't give you Close range, for example. You could choose a spear as the description, then Hand as the range. Your spear is something special, or your technique with it is different, just describe why your weapon has the tags you've chosen.

       
\subsection{Heirloom}  \index{Heirloom} \index{Heirloom}
       

The exact nature of the spirits (and therefore what knowledge they can offer to you) is up to you and the GM to decide. Maybe they're dead ancestors, or echoes of people you've slain, or a minor demon. Up to you.

       
\subsection{Armor Mastery}  \index{Armor Mastery} \index{Armor} \index{Mastery}
       

Armor and shields that are reduced to 0 armor are effectively destroyed. You'll pretty much be paying for a new one anyway, so you might as well drop them and haul out some gold instead.

       
\section{Paladin Moves}  \index{Paladin Moves} \index{Paladin} \index{Moves}
       
\subsection{Evidence of Faith}  \index{Evidence of Faith} \index{Evidence} \index{Faith}
       

Your +1 forward applies to anything you do based on your knowledge of the spell's effects: defying it, defending against it, using it to your advantage, etc.

       
\section{Ranger Moves}  \index{Ranger Moves} \index{Ranger} \index{Moves}
       
\subsection{Command}  \index{Command} \index{Command}
       

Your bonuses only applies when your animal is doing something it's trained in. An animal not trained to attack monsters won't be any help when you're attacking a otyugh.

       
\section{Thief Moves}  \index{Thief Moves} \index{Thief} \index{Moves}
       
\subsection{Poisoner}  \index{Poisoner} \index{Poisoner}
       

In order to make more doses of your chosen poison you need to be reasonably able to gather the required materials. If you're locked up at the top of a tower you're not going to be able to get the materials you need of course.

       
\subsection{Wealth and Taste}  \index{Wealth and Taste} \index{Wealth} \index{Taste}
       

In order to use this move it's really got to be your most valuable possession. It's the honest value you place on it that draws others, no lies.

       
\subsection{Disguise}  \index{Disguise} \index{Disguise}
       

Your disguise covers your appearance and any basics like accents and limps. It doesn't grant you any special knowledge of the target, so if someone asks you what your favorite color is you'd better think fast. Defying Danger with Cha is a common part of maintaining a Disguise.

       
\section{Wizard Moves}  \index{Wizard Moves} \index{Wizard} \index{Moves}
       
\subsection{Empowered Magic}  \index{Empowered Magic} \index{Empowered} \index{Magic}
       

Maximizing the effects of a spell is simple for spells that involve a roll: a maximized Magic Missile does 8 damage. In other cases it's down to the circumstances. A maximized Identify might result in far more information than expected. If there's no clear way to maximize it you can't choose that option.

       

Likewise for doubling the targets. If the spell doesn't have targets you can't choose to double them.

       
\section{Wizard Spells}  \index{Wizard Spells} \index{Wizard} \index{Spells}
       
\subsection{Dispel Magic}  \index{Dispel Magic} \index{Dispel} \index{Magic}
       

The exact effects depend on the circumstances. A goblin orkaster's spell might just be ended; a deity's consecration is probably just dimmed. The GM will tell you the likely effects of Dispeling a given effect before you cast.

       
\subsection{Fireball}  \index{Fireball} \index{Fireball}
       

"Nearby" means a few paces at most, depending on the circumstances.

       
\subsection{Polymorph}  \index{Polymorph} \index{Polymorph}
       

In some cases the GM may choose the last option more than once to list each unexpected benefit or weakness.

       
\subsection{Summon Monster}  \index{Summon Monster} \index{Summon} \index{Monster}
       

The exact type of monster you get is up to the GM, based on your choices. If you want a non-reckless swimming creature you might get a water elemental, a 1d8 damage +2 Str creature might be a barbed devil. Whatever the creature is you still get to play it.

                
