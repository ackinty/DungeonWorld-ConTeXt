\chapter{Moves in Detail}
 \index{Moves in Detail} \index{Moves} \index{Detail}
 


\subsection{Hack and Slash}  \index{Hack and Slash} \index{Hack} \index{Slash}
 

Hack and Slash is for attacking a prepared enemy plain and simple. If the enemy isn't prepared for your attack—if they don't know you're there or they're restrained and helpless—then that's not Hack and Slash. You just deal your damage or murder them outright, depending on the situation. Nasty stuff.

 

The enemy's counter-attack can be any GM move made directly with that creature. A goblin might just attack you back, or they might jam a poisoned needle into your veins. Life's tough, isn't it?

 

Note that an "attack" is some action that a player undertakes that has a chance of causing physical harm to someone else. Attacking a dragon with inch-thick metal scales full of magical energy with a typical sword is like swinging a meat cleaver at a tank: it just isn't going to cause any harm, so it's not an attack. Note that circumstances can change that: if you're in a position to stab the dragon on it's soft underbelly (good luck with getting there) it could hurt, so it's an attack.

 

If the action that triggers the move could reasonably hurt multiple targets roll once and apply damage to each target (they each get their armor).

 
\startExample
Jarl is up to his not-inconsiderable belly in slavering goblins. They have him surrounded, knives bared. "I've had enough of this!" he bellows "I wallop the closest goblin with my hammer." We agree that this is a combat situation and Jarl rolls the dice for Hack and Slash. He rolls an 11, so he has a choice. "Fear is for the weak! I deal extra damage—let the goblins come." "The goblin you strike certainly doesn't like that much" I say, "you smash your hammer into his shoulder and are rewarded with the crunch of goblin bones—and a deep knife wound as the goblin counter-attacks. He deals 4 damage to you."
\stopExample
 
\startExample
Cadeus has the drop on two orc warriors—he's lurking in the shadows as the orcs walk past. "I leap out and bring my sword down in a sweeping arc, like this!" he says, miming the strike. The orc wasn't ready to fight so I say "The orc is caught entirely off-guard and doesn't even have a chance to raise his patchwork shield. Deal your damage." Cadeus rolls his damage and it's enough to kill the orc. The other warrior is still standing, so I say "The other orc freezes in horror for a split second. Then he's smiling at you with his horrible tusked mouth as he raises a signal horn from his belt. What do you do?"
\stopExample
 
\startExample
Bartelby has disarmed a duelist and has him at sword point. "I'm not giving this guy another chance to attack! I run him through." Without thinking about it carefully I say "Oh, okay, sounds like Hack and Slash, roll+Str." Bartelby rolls and gets a 7. I try to make a move "You run him through, he's not able to defend himself, but, uh… Oh, wait, he's not really in melee with you, is he? He's helpless. Forget Hack and Slash. You run him through and he slumps to the ground coughing up blood."
\stopExample
 
\section{Volley}  \index{Volley} \index{Volley}
 

Volley covers the entire act of drawing, aiming, and firing a ranged weapon or throwing a thrown weapon. The advantage to using a ranged weapon over melee is that the attacker is less likely to be attacked back. Of course they do have to worry about ammunition and getting a clear shot though.

 

On a 7-9, read "danger" broadly. It can be bad footing or ending in the path of a sword or maybe just giving up your sweet sniper nest to your enemies. Whatever it is it's impending and it's always something that causes the GM to say "What do you do?" Quite often, the danger will be something that will then require you to dedicate yourself to avoiding it or force you to Defy Danger.

 

If you're throwing something that doesn't have ammo (maybe you've got a move that makes your shield throwable) you can't choose to mark off ammo. Choose from the other two options instead.

 
\startExample
Aranwe is on the floor of the ritualarium as the orc eyegouger chants his ritual from atop the pedestal. "Since Thelian has the other orcs busy, I take the opportunity to site down my bow and take a shot at the orc running the ritual." "Sounds like volley to me." She rolls an 8, plus her Dex makes 9. "Looks like you have a tough choice" I say. "Well, I'm almost out of arrows, and we need to get rid of him before the ritual finishes, so I'm going to take the danger." "Sure, that sounds good. Well, as the ritual progresses the flames around him have gotten higher and you have to move around to take the shot. You hit him dead on, roll your damage, but you had to step inside the ritual circle to do it. Everything outside the circle looks cloudy and unreal, all you can hear is the orc chanting. Thelian, you notice that Aranwe is inside the circle. What do you do?"
\stopExample
 
\startExample
Halek is firing on the advancing kobold mob. He rolls an 8 and decides to be put in danger. I think for a moment and then say "You have to duck and dodge to get the shot but you finally let it go and nail the lead kobold. You hear something behind you and turn to see that you're right next to the ogre. He smashes you with his club and deals you 12 damage." "All that? Just for getting put in danger? That seems like a lot more than danger." He's right, of course, so I say "Oh, you're right—danger's something that's about to happen. How about, instead, you turn around after firing the shot and the ogre's right in your face! He's about to swing his club right down on you. What do you do?"
\stopExample
 
\section{Defy Danger}  \index{Defy Danger} \index{Defy} \index{Danger}
 

You Defy Danger when you do something in the face of impending peril. This may seem like a catch-all. It is! Defy Danger is for those times when it seems like you clearly should be rolling but no other move applies.

 

Defy Danger also applies when you make another move despite danger not covered by that move. For example, Hack and Slash assumes that's you're trading blows in battle—you don't need to Defy Danger because of the monster you're fighting unless there's some specific danger that wouldn't be part of your normal attack. On the other hand, if you're trying to Hack and Slash while spikes shoot from hidden traps in the walls, you're ignoring a clear and present threat and need to Defy Danger.

 

Danger, here, is anything that requires resilience, concentration, or poise. This move will usually be called for by the GM. She'll tell you what the danger is as you make the move. Something like "You'll have to Defy Danger, first. The danger is the steep and icy floor you're running across. If you can keep your footing, you can make it to the door before the Necromancer's magic gets you."

 

Which stat applies depends on what action you take and your action has to trigger the move. That means you can't Defy the Danger of the steep and icy floor with a charming smile just so you can use Cha, since charmingly smiling at the ice floor does nothing to it. On the other hand, making a huge leap over the ice would be Str, placing your feet carefully would be Dex, and so on. Make the move to get the results.

 
\startExample
Emory is climbing a steep ravine. Unbeknownst to him, a cultist sorcerer lurks nearby. The sorcerer casts a spell of frost on the cliffside, covering it with ice. "As you reach for the next handhold, a terrible chill overcomes you. If you want to keep climbing, Defy Danger or risk slipping" I say, making sure to explain what the Danger is. "No way" Emory says, "I need to get the top of this ravine! I grit my teeth and hold tight even as my fingers go numb." He rolls Defy Danger, getting an 8 including his Con for enduring. Now it's time for a hard decision. "You make some progress but as your hands go numb you start slipping. The only way you can get any more traction is by jamming your dagger into the ice to pull yourself up the last few feet. If you do that, though, the dagger is going to be jammed in the face of the cliff until you get a chance to stop and pry it out."
\stopExample
 
\startExample
"The athach's third arm is swinging down on you with its crude club, what are you doing Valeria?" I've just made a move to establish an impending threat: the athach's strike. She says "I Hack and Slash it! I make a wide swing sideways, right into its legs." Sounds good to me, but she's not doing anything about the club coming at her. "Okay, you can do that, but you take the athach's damage from the club coming right down on your skull." "What? But trading blows is part of Hack and Slash, right?" "It is, but before you make your attack there's already a club coming at you, Hack and Slash doesn't cover that. Do you still want to Hack and Slash, or are you doing something about the club?"
\stopExample
 
\startExample
Octavia is locked in battle with an ogre. She says "I drop my shield and take up my hammer in both hands. I swing it at the ogre. That's Hack and Slash, right?" "Yeah, it will be but first you've gotta Defy Danger. The danger is the ogre's massive club." "Isn't that part of what Hack and Slash already is? I mean if he couldn't be smashing me with his club then I wouldn't be making the move at all because we wouldn't be in melee." "Oh yeah, you're totally right. Hack and Slash it is, make your roll!"
\stopExample
 
\section{Defend}  \index{Defend} \index{Defend}
 

Defending something means standing nearby and focusing on preventing attacks on that thing or stopping anyone from getting near it. When you're no longer nearby or you stop devoting your attention to incoming attacks then you lose any Hold you might have had.

 

You can only spend Hold when someone makes an attack on you or the thing you're Defending. The choices you can make depend on the attacker and the type of attack. In particular, you can't deal damage to an attacker who you can't reach with your weapon.

 

An attack is any action you can interfere with that has harmful effects. Swords and arrows are, of course, attacks but so are spells, grabs, and charges. 

 

If the attack doesn't deal damage then halving it means the attacker gets some of what they want but not all of it. It's up to you and the GM to work out what that means depending on the circumstances. If you're defending the Gem Eye of Oro-Uht and an orc tries to grab it from its pedestal then half effect might mean that the gem gets knocked to the floor but the orc doesn't get his hands on it, yet. Or maybe the orc gets ahold of it but so do you—now you're both fighting over it, tooth and nail. If you and the GM can't agree on a halved effect you can't choose that option.

 

Defending yourself is certainly an option. It amounts to giving up on making attacks and just trying to keep yourself safe.

 
\startExample
Avon is weaving a powerful spell to send the source of the Necromancer's power back to the Plane of Death. The spell takes time and concentration and there's zombies massing on all sides! Lux says "While Avon's casting his spell, it's my duty to keep him alive. I stand between him and the dead and slam my hammer against my shield—'You want him, you go through me!' I'm Defending Avon." That sounds good to me, so I say "roll+Con." She gets an 11 and Holds three. A few moments later Avon finishes his spell but, in rolling for it, has to make a tough choice and puts himself in danger. I say "you unleash the magic of your spell, sure enough. The magical disturbance draws the attention of the zombie horde—they sense your power and it drives their hunger! With a sudden burst of speed, they're right on top of you. What do you do?" Avon looks unsure for a moment, but Lux says "Let them come. I've got his back. I'm spending a point of hold to direct that attack to me. I push Avon back and swing a wide arc with my shield. I'll also spend a point of hold to halve the damage. To be safe, I follow up with my hammer and use one more hold to deal damage to the gang of zombies." "Wow, okay. They get a few feeble claws past your guard but you're mostly unscathed. That does it for your hold. Are you still defending him or are you doing something else?" "I don't think he'll live long without me. I yell at him to run without taking my eyes off the zombies, I'm not letting any of them past me." "Sounds like you're Defending again, roll+Con."
\stopExample
 
\startExample
Hadrian has been Defending Durga while she heals a badly wounded Willem. Willem's in fighting shape again so Durga has leaped forwards to drive back the troglodytes. Hadrian is still locked in battle with a deadly crocodilian. The troglodytes attack Durga and Hadrian reacts. "Wait! I still have one hold to Defend Durga. I'm doing to redirect that attack to myself." That doesn't sound quite right to me, they're spread out, now. "How are you doing that if she's over at the troglodyte camp and you're battling the crocodilian in the water?" "Oh yeah. I guess when I started doing something other than standing guard I lost that hold. Damn."
\stopExample
 
\section{Spout Lore}  \index{Spout Lore} \index{Spout} \index{Lore}
 

You Spout Lore any time you want to search your memory for knowledge or facts about something. You take a moment to ponder the things you know about the Orcish Tribes or the Tower of Ul'dammar and then reveal that knowledge.

 

The knowledge you get is like consulting a bestiary, travel guide, or library. You get facts about the subject matter. On a 10+ those facts the GM will show you how those facts can be immediately useful, on a 7–9 they're just facts.

 

On a miss the GM's move will often have to do with the time you take thinking. Maybe you miss that goblin moving around behind you, or the trip wire across the hallway. It's also a great chance to reveal an unwelcome truth.

 

Just in case it isn't clear: the answers are always true, even if the GM had to make them up on the spot. Always say what honesty demands.

 
\startExample
Fenfaril has had the misfortune of dropping through an illusory floor and now finds himself in a murky pit. A mottled, eyeless creature shambles towards him, mumbling in a strange tongue. "I'm a little freaked out—what is this thing? Is it going to attack me? I probably read about these things in a bestiary back in school." "Great, that's Spout Lore." I say. Fenfaril rolls and gets an 8. "Well of course you read about these. The name escapes you, but you clearly remember a drawing of a creature like this standing in front of a doorway, like a guard, with someone kneeling before it." On a strong hit I would have given some information on what makes the creature let people pass.
\stopExample
 
\startExample
Vitus has Spouted Lore on a gilded skull she found on a pedestal and gotten a 10. I begin by saying, mysteriously, "You're pretty sure you recognize the telltale signs of metal forged in the City of Dis, the living planar city." I catch myself and remember to be generous with the truth and make it useful. I add "You recognize some of the glyphs from your spellbook, actually: they're part of fire spells, but with other magic symbols smaller inside them. Casting a non-fire spell into the skull turns it into fire magic, based on the glyphs."
\stopExample
 
\section{Discern Realities}  \index{Discern Realities} \index{Discern} \index{Realities}
 

To Discern Realities you must closely observe your target. That usually means interacting with it or watching someone else do the same. You can't just stick your head in the doorway and Discern Realities about a room. You're not merely scanning for clues—you have to look under and around things, tap the walls and check for weird dust patterns on the bookshelves. That sort of thing.

 

Discerning Realities isn't just about noticing a detail, it's about figuring out the bigger picture. The GM always describes what the player characters experience honestly, so during a fight the GM will say that the kobold mage stays at the other end of the hall. Discerning Realities could reveal the reason behind that: the kobold's motions reveal that he's actually pulling energy from the room behind him, he can't come any close.

 

Just like Spout Lore the answers you get are always honest ones. Even if the GM has to figure it out on the spot. Once they answer, it's set in stone. You'll want to Discern Realities to find the truth behind illusions—magical or otherwise.

 

Unless a move says otherwise players can only ask questions from the list. If a player asks a question not on the list the GM can tell them to try again or answer a question from the list that seems equivalent.

 

Of course, some questions might have a negative answer, that's fine. If there really, honestly is nothing useful or valuable here, the GM will answer that question with "Nothing, sorry."

 
\startExample
Finding a strangely empty room guarded by deadly traps, Omar says "I don't trust this shifty room. I'm going to poke around a little. I take out my tools and start messing with stuff—pulling candlesticks, tapping the wall with my stone hammer. My usual tricks." I say to Omar, "Sounds like you're Discerning Realities?" Omar answers in the affirmative and makes his roll. He rolls a 12 and gets to ask questions. "What here is not as it appears to be?" I think for a second, look at my notes and tell him, "As you tap the walls you find that there's an odd, hollow space on the north side. The stones look newer too, this was added recently. It's actually a hidden room."
\stopExample
 
\startExample
Omar still has two more questions. His his next one is "Who sealed the hidden room?" That's not a question from the list but it sounds to me like he's really asking "what happened here recently". I answer that instead. "Looking at the stonework you notice the wall bends out a little. The work's definitely of goblin origin—shoddy and quick. The only way it could get bent out like that is if something pushed out on the stones from within." "So some goblins blocked it from the other side?" Omar says. "Yeah, exactly."
\stopExample
 
\section{Parley}  \index{Parley} \index{Parley}
 

Parley covers a lot of ground including old standbys like intimidation and diplomacy. You know you're using Parley when you're trying to get someone to do something for you by holding a promise or threat over them. Nice or not, the tone doesn't matter.

 

Merely asking someone politely isn't Parleying. That's just talking. You say "Can I have that magic sword?" and the King's knight says "Hell no, this is my blade, my father forged it and my mother enchanted it" and that's that. To Parley, you have to have leverage. Leverage is anything that could lure the target of your Parley to do something for you. Maybe it's something they want or something they don't want you to do. Like a sack of gold. Or punching them in the face. What counts as leverage depends on the people involved and the request being made. Threaten a long goblin with death and you have leverage. Threaten a goblin backed up by his gang with death and he might think he's better off in a fight.

 

On a hit they ask you for something related to whatever leverage you have. If your leverage is that you're standing before them sharpening your knife and insinuating about how much you'd like to shank them with it they might ask you to let them go. If your leverage is your position in court above them they might ask for a favor.

 

Whatever they ask for, on a 10+, you just have to promise it clearly and unambiguously. On a 7–9, that's not enough: you also have to give them some assurance, right now, before they do what you want. If you promise that you'll ensure their safety from the wolves if they do what you want and you roll a 7-9 they won't do their part until you bring a fresh wolf pelt, for example. It's worth noting that on a 10+ you don't actually have to keep your promise. Whether you'll follow up or not, well, that's up to you. Of course breaking promises leads to problems. People don't take kindly to oath-breakers and aren't likely to deal with them in the future.

 

In some cases when you state what you want you may include a possible promise for the creature to make, as in "Flee and I'll let you live." It's up to the target of the Parley if that's the promise they want or if they have something else in mind. They can say "yes, let me live and I'll go" (with assurances, if you rolled a 7–9) or "promise me you won't follow me."

 
\startExample
Leena is trying to convince Lord Hywn to vouch for her so that she is granted an audience with the Queen. She's laid out what she wants pretty well but I say "Lord Hywn obviously isn't convinced. Why should he help you?" She smirks a bit. "Oh. That. While I'm talking to him, I absentmindedly start playing with the signet ring from that assassin we killed. The one he hired to off the prince. I make a big show of it just to make sure he sees who's ring it is." That's perfect; now I know what to ask for. Leena's player hits her roll with an 8. "Once your little show sets in, Hywn just looks at you coldly. After a moment he says 'Enough being coy. You and I both know you murdered my hired man. Give me that ring and swear you'll speak no more about it, then I'll do as you ask." "Oh sure, I give it to him" she says "We can always dig up more dirt on this scumbag later."
\stopExample
 
\startExample
Pendrell's trying to get into the gambling den where One-Eye usually plays. He saunters up to the guards and says "Hey fellas, how's it going, care to open the door for me?" Pendrell's player says "I'm being all suave as I do it to; really cool so these guys will let me in. That's Parley! I roll+Cha." Something's not right here so I stop him "Wait a sec. All you've done is tell these guys what you want—you're just talking. The big smelly one on the right of the door steps in front of you, looks you in the eyes and says 'Sorry, private venue' like he's bored with keeping people out and he'd rather be inside himself. If you want to Parley him, you need some leverage. A bribe maybe?"
\stopExample
 
\section{Aid or Interfere}  \index{Aid or Interfere} \index{Aid} \index{Interfere}
 

Any time you feel like two players should be rolling against each other, the defender should be Interfering with the attacker. This doesn't always mean sabotaging them. It can mean anything from arguing against a Parley to just being a shifty person who's hard to Discern. It's about getting in the way of another players' success.

 

Always ask the person aiding or interfering how they are doing it. As long as they can answer that, they trigger the move. Sometimes, as the GM, you'll have to ask if interference is happening. Your players might not always notice they're interfering with each other.

 

Aid is a little more obvious. If a player can explain how they're helping in a roll and it makes sense, let them roll to aid.

 

No matter how many people aid or interfere with a given roll, the target only gets the +1 or -2 once. Even if a whole party of adventurers aid in attacking an ogre, the one who makes the final attack only gets +1.

 
\startExample
Ozruk stands alone and bloodied before a pack of angry hellhounds. Behind him, the Prince of Lescia weeps in fear. Ozruk says "I stand firm and lift my shield, despite certain doom. I'm defending the Prince." At the last moment, though (just as I'm about to have Ozruk roll Defend) Aronwe appears from the shadows, sword drawn. "Doom is not so certain, Dwarf" he says. "I'm standing beside him, helping Ozruk Defend by covering his sword arm." Aronwe rolls+bonds and, if he succeeds, Ozruk will be able to add a +1 to his Defend result.
\stopExample
 
